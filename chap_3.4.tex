\documentclass{article}
\usepackage[utf8]{inputenc}
\usepackage{latexsym}
\usepackage{amsmath}
\usepackage{amssymb}
\usepackage{bm}
\usepackage{multicol}
\usepackage{graphicx}
\usepackage{booktabs}
\usepackage{wrapfig}
\usepackage{fancybox}
\usepackage{bbm}
\usepackage{enumerate}
\pagestyle{plain}
\usepackage{yfonts}
\usepackage{ragged2e}


\usepackage{titling}
\setlength{\droptitle}{-7em} 


\def\Tiny{\fontsize{4pt}{4pt}\selectfont}
\newcommand*{\eqdef}{\ensuremath{\overset{\mathclap{\text{\Tiny def}}}{=}}}


\usepackage{geometry}
 \geometry{
 a4paper,
 total={160mm,257mm},
 left=24mm,
 top=20mm,
 }
 
 \usepackage{tcolorbox}
\newtcolorbox{mybox}[1]{colback=blue!5!white,colframe=blue!42!black,fonttitle=\bfseries,title=Problem #1}
 

\title{Title}
\author{Carlos Faz}
\date{ \ }

\begin{document}


\begin{flushleft}




\begin{mybox}{3.4.1}
Another set of Euler rotations in common use is
\begin{enumerate}[$(a)$]
\item a rotation about the $x_{3}$ -axis through an angle $\varphi$, counterclockwise,
\item a rotation about the $x_{1}'$ -axis through an angle $\theta,$ counterclockwise,
\item a rotation about the $x_{3}^{\prime \prime}$ -axis through an angle $\psi,$ counterclockwise. 
\end{enumerate}
\begin{multicols}{2}
If 
$$
\begin{array}{l}
\alpha=\varphi-\pi / 2 \\
\beta=\theta \\
\gamma=\psi+\pi / 2
\end{array}
$$
or
$$
\begin{array}{l}
\varphi=\alpha+\pi / 2 \\
\theta=\beta \\
\psi=\gamma-\pi / 2
\end{array}
$$
\end{multicols}
show that the final systems are identical.
\end{mybox}

$\boxed{\textbf{Solution}}$ The Euler rotations given in the text is:
\begin{enumerate}
\item a rotation about the $x_{3}-$ axis through an angle $\alpha,$ counterclockwise
\item a rotation about the $x_{2}'$ - axis through an angle $\beta$, counterclockwise
\item a rotation about the $x_{3}^{\prime \prime}$ -axis through an angle $\gamma$, counterclockwise.
\end{enumerate}
The Euler rotation defined here differ from those in the text in that the inclination of the polar axis is about that $x_{1}'-$axis rather than the $x_{2}'-$ axis. Therefore, to achieve the same polar orientation, we must place the $x_{1}'-$axis where the $x_{2}'-$axis was using the text rotation. This requires an additional first rotation of $\frac{\pi}{2}$. After inclining the polar axis, the rotational position
is now $\frac{\pi}{2}$ greater than form the text rotation, so the third Euler angle must be $\frac{\pi}{2}$ less than its original value.



\begin{mybox}{3.4.2}
Suppose the Earth is moved (rotated) so that the north pole goes to $30^{\circ}$ north, $20^{\circ}$ west (original latitude and longitude system) and the $10^{\circ}$ west meridian points due south (also in the original system).
(a) What are the Euler angles describing this rotation?
(b) Find the corresponding direction cosines.
\end{mybox}

$\boxed{\textbf{Solution}}$ No solution yet.


\begin{mybox}{3.4.3}
Verify that the Euler angle rotation matrix, Eq. (3.37), is invariant under the transformation
$$
\alpha \rightarrow \alpha+\pi, \quad \beta \rightarrow-\beta, \quad \gamma \rightarrow \gamma-\pi
$$
\end{mybox}

$\boxed{\textbf{Solution}}$ The Euler rotation matrix $\mathbf{S}(\alpha, \beta, \gamma)$ is :
$$
\mathbf{S}(\alpha, \beta, \gamma)=\begin{bmatrix}
\cos \gamma \cos \beta \cos \alpha-\sin \gamma \sin \alpha & \cos \gamma \cos \beta \sin \alpha+\sin \gamma \cos \alpha & -\cos \gamma \sin \beta \\
-\sin \gamma \cos \beta \cos \alpha-\cos \gamma \sin \alpha & -\sin \gamma \cos \beta \sin \alpha+\cos \gamma \cos \alpha & \sin \gamma \sin \beta \\
\sin \beta \cos \alpha & \sin \beta \sin \alpha & \cos \beta
\end{bmatrix}
$$
Using the transformation $\alpha \rightarrow \alpha+\pi, \beta \rightarrow-\beta, \gamma \rightarrow \gamma-\pi$ we get,
$$
\mathbf{S}(\alpha+\pi,-\beta, \gamma-\pi)=\begin{bmatrix}
\cos \gamma \cos \beta \cos \alpha-\sin \gamma \sin \alpha & \cos \gamma \cos \beta \sin \alpha+\sin \gamma \cos \alpha & -\cos \gamma \sin \beta \\
-\sin \gamma \cos \beta \cos \alpha-\cos \gamma \sin \alpha & -\sin \gamma \cos \beta \sin \alpha+\cos \gamma \cos \alpha & \sin \gamma \sin \beta \\
\sin \beta \cos \alpha & \sin \beta \sin \alpha & \cos \beta
\end{bmatrix}
$$
as $\cos \alpha \rightarrow-\cos \alpha, \sin \alpha \rightarrow-\sin \alpha ; \cos \beta \rightarrow \cos \beta, \sin \beta \rightarrow-\sin \beta ; \sin \gamma \rightarrow-\sin \gamma$, $\cos \gamma \rightarrow-\cos \gamma$
Thus, $\mathbf{S}(\alpha, \beta, \gamma)=\mathbf{S}(\alpha+\pi,-\beta, \gamma-\pi)$
Hence, $\mathbf{S}(\alpha, \beta, \gamma)$ is invariant under the transformation
$\alpha \rightarrow \alpha+\pi, \beta \rightarrow-\beta, \gamma \rightarrow \gamma-\pi$




\begin{mybox}{3.4.4}
Show that the Euler angle rotation matrix $\mathbf{S}(\alpha, \beta, \gamma)$ satisfies the following relations:
\begin{enumerate}[$(a)$]
\item $\mathbf{S}^{-1}(\alpha, \beta, \gamma)=\tilde{\mathbf{S}}(\alpha, \beta, \gamma)$
\item $\mathbf{S}^{-1}(\alpha, \beta, \gamma)=\mathbf{S}(-\gamma,-\beta,-\alpha)$
\end{enumerate}
\end{mybox}

$\boxed{\textbf{Solution}}$ For $(a)$ The three Euler rotations $\boldsymbol{S}_{1}(\alpha), \boldsymbol{S}_{2}(\beta), \boldsymbol{S}_{3}(\gamma)$ are an orthogonal matrix.
So, $\mathbf{S}(\alpha, \beta, \gamma)=\boldsymbol{S}_{3}(\gamma) \boldsymbol{S}_{2}(\beta) \boldsymbol{S}_{1}(\alpha)$ must also be orthogonal. Therefore $\mathbf{S}^{-1}(\alpha, \beta, \gamma)=\tilde{\mathbf{S}}(\alpha, \beta, \gamma)$, by the definition of an orthogonal matrix.


$\boxed{\textbf{Solution}}$ For $(b)$ we have 
$$
\mathbf{S}(\alpha, \beta, \gamma)=\begin{bmatrix}
\cos \gamma \cos \beta \cos \alpha-\sin \gamma \sin \alpha & \cos \gamma \cos \beta \sin \alpha+\sin \gamma \cos \alpha & -\cos \gamma \sin \beta \\
-\sin \gamma \cos \beta \cos \alpha-\cos \gamma \sin \alpha & -\sin \gamma \cos \beta \sin \alpha+\cos \gamma \cos \alpha & \sin \gamma \sin \beta \\
\sin \beta \cos \alpha & \sin \beta \sin \alpha & \cos \beta
\end{bmatrix}
$$
$$
\mathbf{S}(-\gamma,-\beta,-\alpha)=\begin{bmatrix}
\cos \gamma \cos \beta \cos \alpha-\sin \gamma \sin \alpha & -\sin \gamma \cos \beta \cos \alpha-\cos \gamma \sin \alpha & \sin \beta \cos \alpha \\
\cos \gamma \cos \beta \sin \alpha+\sin \gamma \cos \alpha & -\sin \gamma \cos \beta \sin \alpha+\cos \gamma \cos \alpha & \sin \beta \sin \alpha \\
-\cos \gamma \sin \beta & \sin \gamma \sin \beta & \cos \beta
\end{bmatrix}
$$

$$
\begin{aligned}
\mathbf{S}^{-1}(\alpha, \beta, \gamma) &=\tilde{\mathbf{S}}(\alpha, \beta, \gamma) \\
&=\begin{bmatrix}
\cos \gamma \cos \beta \cos \alpha-\sin \gamma \sin \alpha & -\sin \gamma \cos \beta \cos \alpha-\cos \gamma \sin \alpha & \sin \beta \cos \alpha \\
\cos \gamma \cos \beta \sin \alpha+\sin \gamma \cos \alpha & -\sin \gamma \cos \beta \sin \alpha+\cos \gamma \cos \alpha & \sin \beta \sin \alpha \\
-\cos \gamma \sin \beta & \sin \gamma \sin \beta & \cos \beta
\end{bmatrix}
\end{aligned}
$$

Thus, $\mathbf{S}^{-1}(\alpha, \beta, \gamma)=\mathbf{S}(-\gamma,-\beta,-\alpha)$



\begin{mybox}{3.4.5}
The coordinate system $(x, y, z)$ is rotated through an angle $\Phi$ counterclockwise about an axis defined by the unit vector $\hat{\mathbf{n}}$ into system $\left(x', y', z'\right) .$ In terms of the new coordinates the radius vector becomes
$$
\mathbf{r}'=\mathbf{r} \cos \Phi+\mathbf{r} \times \mathbf{n} \sin \Phi+\hat{\mathbf{n}}(\hat{\mathbf{n}} \cdot \mathbf{r})(1-\cos \Phi)
$$
\begin{enumerate}[$(a)$]
\item Derive this expression from geometric considerations.
\item Show that it reduces as expected for $\hat{\mathbf{n}}=\hat{\mathbf{e}}_{z} .$ The answer, in matrix form, appears in Eq. (3.35)
\item Verify that $r'^{2}=r^{2}$.
\end{enumerate}
\end{mybox}



$\boxed{\textbf{Solution}}$ For $(a)$ the projection of $r$ on the rotation axis is not changed by the rotation; it is $(\mathbf{r} \cdot \hat{\mathbf{n}}) \hat{\mathbf{n}}$.
The portion of $r$ perpendicular to the rotation axis can be written $r-(\mathbf{r} \cdot \hat{\mathbf{n}}) \hat{\mathbf{n}}$.
Upon rotation through an angle $\Phi$, this vector perpendicular to the rotation axis will consist of a vector in its original direction $(r-(\mathbf{r} \cdot \hat{\mathbf{n}}) \hat{\mathbf{n}}) \cos \Phi$ plus a vector perpendicular both to it and to $\hat{\mathbf{n}}$ given by $(r-(\mathbf{r} \cdot \hat{\mathbf{n}}) \hat{\mathbf{n}}) \sin \Phi \times \hat{\mathbf{n}} ;$ this reduces to $\mathbf{r} \times \hat{\mathbf{n}} \sin \Phi$
Adding these contributions, we get the required result.

$\boxed{\textbf{Solution}}$ For $(b)$ if $\hat{\mathbf{n}}=\hat{\mathbf{e}}_{z},$ the formula $\mathbf{r}'=\mathbf{r} \cos \Phi+\mathbf{r} \times n \sin \Phi+\hat{\mathbf{n}}(\hat{\mathbf{n}} \cdot \mathbf{r})(1-\cos \Phi)$ becomes
$$
\begin{aligned}
\mathbf{r}' &=\left(x \hat{\mathbf{e}}_{x}+y \hat{\mathbf{e}}_{y}+z \hat{\mathbf{e}}_{z}\right) \cos \Phi+\left(y \hat{\mathbf{e}}_{x}-x \hat{\mathbf{e}}_{y}\right) \sin \Phi+\hat{\mathbf{e}}_{z}\left(z \hat{\mathbf{e}}_{z}\right)(1-\cos \Phi) \\
&=\left(x \hat{\mathbf{e}}_{x}+y \hat{\mathbf{e}}_{y}+z \hat{\mathbf{e}}_{z}\right) \cos \Phi+\left(y \hat{\mathbf{e}}_{x}-x \hat{\mathbf{e}}_{y}\right) \sin \Phi+z(1-\cos \Phi) \hat{\mathbf{e}}_{z} \\
&=x \cos \Phi \hat{\mathbf{e}}_{x}+y \cos \Phi \hat{\mathbf{e}}_{y}+z \cos \Phi \hat{\mathbf{e}}_{z}+y \sin \Phi \hat{\mathbf{e}}_{x}-x \sin \Phi \hat{\mathbf{e}}_{y}+z(1-\cos \Phi) \hat{\mathbf{e}}_{z}
\end{aligned}
$$
as $r=x \hat{\mathbf{e}}_{x}+y \hat{\mathbf{e}}_{y}+z \hat{\mathbf{e}}_{z}, \quad \mathbf{r} \times n=\mathbf{r} \times \hat{\mathbf{e}}_{z}=y \hat{\mathbf{e}}_{x}-x \hat{\mathbf{e}}_{y}$ and
Simplifying, this reduces to
$$
\mathbf{r}'=(x \cos \Phi+y \sin \Phi) \hat{\mathbf{e}}_{x}+(y \cos \Phi-x \sin \Phi) \hat{\mathbf{e}}_{y}+z \hat{\mathbf{e}}_{z}
$$
This corresponds to the rotational transformation whose matrix form is
$$
\boldsymbol{S}_{1}(\alpha)=\begin{bmatrix}
\cos \alpha & \sin \alpha & 0 \\
-\sin \alpha & \cos \alpha & 0 \\
0 & 0 & 1
\end{bmatrix}
$$


$\boxed{\textbf{Solution}}$ For $(c)$ we expand $r'^{2}$, recognizing that the second term of

$$
\mathbf{r}'=\mathbf{r} \cos \Phi+\mathbf{r} \times n \sin \Phi+\hat{\mathbf{n}}(\hat{\mathbf{n}} \cdot \mathbf{r})(1-\cos \Phi)
$$
$$
\begin{aligned}
r'^{2} &=\mathbf{r}' \cdot \mathbf{r}' \\
&=(\mathbf{r} \cos \Phi+\mathbf{r} \times \hat{\mathbf{n}} \sin \Phi+\hat{\mathbf{n}}(\hat{\mathbf{n}} \cdot \mathbf{r})(1-\cos \Phi)) \cdot(\mathbf{r} \cos \Phi+\mathbf{r} \times \hat{\mathbf{n}} \sin \Phi+\hat{\mathbf{n}}(\hat{\mathbf{n}} \cdot \mathbf{r})(1-\cos \Phi)) \\
&=r^{2} \cos ^{2} \Phi+(\mathbf{r} \cdot \mathbf{r} \times \hat{\mathbf{n}}) \sin \Phi \cos \Phi+(\hat{\mathbf{n}} \cdot \mathbf{r})^{2}(1-\cos \Phi) \cos \Phi+(\mathbf{r} \times \hat{\mathbf{n}} \cdot \mathbf{r}) \sin \Phi \cos \Phi \\
&+(\mathbf{r} \times \hat{\mathbf{n}} \cdot \mathbf{r} \times \hat{\mathbf{n}}) \sin ^{2} \Phi+(\mathbf{r} \times \hat{\mathbf{n}} \cdot \hat{\mathbf{n}})(\hat{\mathbf{n}} \cdot \mathbf{r}) \sin \Phi(1-\cos \Phi)+(\hat{\mathbf{n}} \cdot \mathbf{r})^{2}(1-\cos \Phi) \cos \Phi \\
&+(\hat{\mathbf{n}} \cdot \mathbf{r} \times \hat{\mathbf{n}})(\hat{\mathbf{n}} \cdot \mathbf{r}) \sin \Phi(1-\cos \Phi)+(\hat{\mathbf{n}} \cdot \mathbf{r})^{2}(1-\cos \Phi)^{2}
\end{aligned}
$$
$$
r'^{2}=r^{2} \cos ^{2} \Phi+(\mathbf{r} \times \hat{\mathbf{n}} \cdot \mathbf{r} \times \hat{\mathbf{n}}) \sin ^{2} \Phi+(\hat{\mathbf{n}} \cdot \mathbf{r})^{2}(1-\cos \Phi)^{2}+2(\hat{\mathbf{n}} \cdot \mathbf{r})^{2}(1-\cos \Phi) \cos \Phi
$$
as $(\mathbf{r} \cdot \mathbf{r} \times \hat{\mathbf{n}})=(\mathbf{r} \times \hat{\mathbf{n}} \cdot \mathbf{r})=(\mathbf{r} \times \hat{\mathbf{n}} \cdot \hat{\mathbf{n}})=(\hat{\mathbf{n}} \cdot \mathbf{r} \times \hat{\mathbf{n}})(\hat{\mathbf{n}} \cdot \mathbf{r})=0$
$$
\begin{aligned}
r'^{2} &=r^{2} \cos ^{2} \Phi+(\mathbf{r} \times \hat{\mathbf{n}} \cdot \mathbf{r} \times \hat{\mathbf{n}}) \sin ^{2} \Phi+(\hat{\mathbf{n}} \cdot \mathbf{r})^{2}(1-\cos \Phi)^{2}+2(\hat{\mathbf{n}} \cdot \mathbf{r})^{2}(1-\cos \Phi) \cos \Phi \\
&=r^{2}+(\hat{\mathbf{n}} \cdot \mathbf{r})^{2}\left(-\sin ^{2} \Phi+1+\cos ^{2} \Phi-2 \cos ^{2} \Phi\right) \\
&=r^{2}
\end{aligned}
$$





\end{flushleft}
\end{document}
