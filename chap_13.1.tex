\documentclass{article}
\usepackage[utf8]{inputenc}
\usepackage{latexsym}
\usepackage{amsmath}
\usepackage{amssymb}
\usepackage{bm}
\usepackage{multicol}
\usepackage{graphicx}
\usepackage{booktabs}
\usepackage{wrapfig}
\usepackage{fancybox}
\usepackage{bbm}
\usepackage{enumerate}
\pagestyle{plain}
\usepackage{yfonts}
\usepackage{ragged2e}


\usepackage{titling}
\setlength{\droptitle}{-7em} 


\def\Tiny{\fontsize{4pt}{4pt}\selectfont}
\newcommand*{\eqdef}{\ensuremath{\overset{\mathclap{\text{\Tiny def}}}{=}}}


\usepackage{geometry}
 \geometry{
 a4paper,
 total={160mm,257mm},
 left=24mm,
 top=20mm,
 }
 
 \usepackage{tcolorbox}
\newtcolorbox{mybox}[1]{colback=blue!5!white,colframe=blue!42!black,fonttitle=\bfseries,title=Problem #1}
 

\title{Title}
\author{Carlos Faz}
\date{ \ }

\begin{document}

\maketitle

\begin{flushleft}


\begin{mybox}{13.1.1}
Derive the recurrence relations
$$
\Gamma(z+1)=z \Gamma(z)
$$
from the Euler integral, Eq. (13.5),
$$
\Gamma(z)=\int_{0}^{\infty} e^{-t} t^{z-1} d t
$$
\end{mybox}

$\boxed{\textbf{Solution}}$ Consider the Euler integral 
$$\Gamma z=\int_{0}^{\infty} e^{-t} t^{z-1} d t$$
Put, $z=z+1$
$$
\begin{aligned}
\Gamma(z+1) &=\int_{0}^{\infty} e^{-t} t^{z+1-1} d t \\
&=\int_{0}^{\infty} e^{-t} t^{z} d t \\
&=t^{z} \int_{0}^{\infty} e^{-t} d t-\int_{0}^{\infty} \frac{d t^{z}}{d x} \int e^{-t} d t \\
=&-t^{z} e^{-t}\Big|_{0} ^{\infty}+z \int_{0}^{\infty} e^{-t} t^{z-1} d t \\
=& z \Gamma(z)
\end{aligned}
$$

\begin{mybox}{13.1.2}
In a power-series solution for the Legendre functions of the second kind we encounter the expression
$$
\frac{(n+1)(n+2)(n+3) \cdots(n+2 s-1)(n+2 s)}{2 \cdot 4 \cdot 6 \cdot 8 \cdots(2 s-2)(2 s) \cdot(2 n+3)(2 n+5)(2 n+7) \cdots(2 n+2 s+1)}
$$
in which $s$ is a positive integer.
\begin{enumerate}[$(a)$]
\item Rewrite this expression in terms of factorials.
\item Rewrite this expression using Pochhammer symbols; see Eq. (1.72).
\end{enumerate}
\end{mybox}

$\boxed{\textbf{Solution}}$ For $(a)$ Notice that
$$\frac{(n+1)(n+2)(n+3) \cdots(n+2 s-1)(n+2 s)}{2.4 .6 .8 \cdot \cdots(2 s-2)(2 s) \cdot(2 n+3)(2 n+5)(2 n+7) \cdots(2 n+2 s+1)}$$
$$=\frac{[n !(n+1)(n+2)(n+3) \cdots(n+2 s-1)(n+2 s)]}{n ! s ! 2^{s} \cdot(2 n+3)(2 n+5)(2 n+7) \cdots(2 n+2 s+1)}$$
$$=\frac{(n+2 s) !(2 n+1) !}{n ! s ! 2^{s} \cdot[(2 n+1) !(2 n+3)(2 n+5)(2 n+7) \cdots(2 n+2 s+1)}$$
$$
=\frac{(n+2 s) !(2 n+1) ![(2 n+2)(2 n+4)(2 n+6) \cdots(2 n+2 s)]}{n ! s ! 2^{s} \cdot[(2 n+1) !(2 n+3)(2 n+4)(2 n+5)(2 n+6)(2 n+7) \cdots(2 n+2 s)(2 n+2 s+1)]}
$$
$$
=\frac{(n+2 s) !(2 n+1) ! 2^{s}[(n+1)(n+2)(n+3) \cdots(n+s)]}{n ! s ! 2^{s} \cdot[(2 n+1) !(2 n+3)(2 n+4)(2 n+5)(2 n+6)(2 n+7) \cdots(2 n+2 s)(2 n+2 s+1)]}
$$
$$
=\frac{(n+2 s) !(2 n+1) ![n !(n+1)(n+2)(n+3) \cdots(n+s)]}{n ! s ! n ![(2 n+1) !(2 n+3)(2 n+4)(2 n+5)(2 n+6)(2 n+7) \cdots(2 n+2 s)(2 n+2 s+1)]}
$$
$$
=\frac{(n+2 s) !(2 n+1) !(n+s) !}{n ! n ! s !(2 n+2 s+1) !}
$$
$\boxed{\textbf{Solution}}$ For $(b)$ we notice that
$$
\frac{(n+1)(n+2)(n+3) \cdots(n+2 s-1)(n+2 s)}{2\cdot 4 \cdot 6 \cdot 8 \cdots \cdot(2 s-2)(2 s) \cdot(2 n+3)(2 n+5)(2 n+7) \cdots(2 n+2 s+1)}
$$
$$
=\frac{(n+1)(n+2)(n+3) \cdots[(n+1)+(2 s-2)][(n+1)+(2 s-1)]}{\left(2^{s}[1 \cdot 2 \cdot 3 \cdot \cdots \cdot(s-1) s]\right) \cdot[(2 n+3)(2 n+5)(2 n+7) \cdots(2 n+2 s+1)]}
$$
$$
=\frac{(n+1)_{(2 s-1)+1} \cdot[(2 n+2)(2 n+4)(2 n+6) \cdots(2 n+2 s)]}{\left(2^{s}[1 \cdot 2 \cdot 3 \cdot \cdots \cdot\{1+(s-2)\}\{1+(s-1)\}) \cdot[(2 n+2)(2 n+3)(2 n+4)(2 n+5) \cdots(2 n+2 s)(2 n+2 s+1)]\right.}
$$	
$$
=\frac{(n+1)_{2 s} \cdot[(n+1)(n+2)(n+3) \cdots(n+s)] \cdot 2^{s}}{2^{s}(1)_{(s-1)+1} \cdot[(2 n+2)(2 n+3)(2 n+4) \cdots\{(2 n+2)+(2 s-1)\}]}
$$
$$
=\frac{(n+1)_{2 s} \cdot[(n+1)(n+2)(n+3) \cdots\{(n+1)+(s-1)\}]}{(1)_{s} \cdot(2 n+2)_{(2 s-1)+1}}
$$
$$=\frac{(n+1)_{2 s} \cdot(n+1)_{(s-1)+1}}{(1)_{s} \cdot(2 n+2)_{2 s}}$$
$$=\frac{(n+1)_{2 s} \cdot(n+1)_{s}}{(1)_{s} \cdot(2 n+2)_{2 s}}$$

\begin{mybox}{13.1.3}
Show that $\Gamma(z)$ may be written
$$\Gamma(z)=2 \int_{0}^{\infty} e^{-t^{2}} t^{2 z-1} d t, \quad \operatorname{Re}(z)>0$$
$$\Gamma(z)=\int_{0}^{1}\left[\ln \left(\frac{1}{t}\right)\right]^{z-1} d t, \quad \Re e(z)>0$$
\end{mybox}
$\boxed{\textbf{Solution}}$ Changing variables $t=u^{2}$ and $d t=2 u d u$ we have
$$
\begin{aligned}
\Gamma z &=\int_{0}^{\infty} e^{-u^{2}} u^{2 z-2} u d u \\
&=\int_{0}^{\infty} e^{-u^{2}} u^{2 z-1} d u \\
&=\int_{0}^{\infty} e^{-t^{2}} t^{2 z-1} d t
\end{aligned}
$$
as $t \rightarrow 0$ to $\infty u \rightarrow 0$ to 1 the equation takes the form of 
$$
\begin{aligned}
\Gamma z &=\int_{0}^{1} e^{-\ln \frac{1}{u}}\left(\ln \frac{1}{u}\right)^{z-1} u d u \\
&=\int_{0}^{1} u\left(\ln \frac{1}{u}\right)^{z-1} u d u \\
&=\int_{0}^{1}\left(\ln \frac{1}{u}\right)^{z-1} d u \\
&=\int_{0}^{1}\left(\ln \frac{1}{t}\right)^{z-1} d t
\end{aligned}
$$

\begin{mybox}{13.1.4}
In a Maxwellian distribution the fraction of particles of mass $m$ with speed between $v$ and $v+d v$ is
$$
\frac{d N}{N}=4 \pi\left(\frac{m}{2 \pi k T}\right)^{3 / 2} \exp \left(-\frac{m v^{2}}{2 k T}\right) v^{2} d v
$$
where $N$ is the total number of particles, $k$ is Boltzmann's constant, and $T$ is the absolute temperature. The average or expectation value of $v^{n}$ is defined as $\left\langle v^{n}\right\rangle=$ $N^{-1} \int v^{n} d N .$ Show that
$$
\left\langle v^{n}\right\rangle=\left(\frac{2 k T}{m}\right)^{n / 2} \frac{\Gamma\left(\frac{n+3}{2}\right)}{\Gamma\left(\frac{3}{2}\right)}
$$
This is an extension of Example $13.1 .1,$ in which the distribution was in kinetic energy $E=m v^{2} / 2,$ with $d E=m v d v$
\end{mybox}
$\boxed{\textbf{Solution}}$ 
$$\left\langle v^{n}\right\rangle=N^{-1} \int v^{n} d N$$
$$=\int v^{n} \frac{d N}{N}$$
$$=\int_{0}^{\infty} v^{n} \cdot 4 \pi\left(\frac{m}{2 \pi k T}\right)^{\frac{3}{2}} e^{\frac{m^{2}}{2 k T}} v^{2} d v$$
$$=4 \pi\left(\frac{m}{2 \pi k T}\right)^{\frac{3}{2}} \int_{0}^{\infty} v^{n} e^{\frac{m^{2}}{2 k T}} v^{n+1} v d v$$

Let $\frac{m v^{2}}{2 k T}=u^{2} .$ Then $v=\left(\frac{2 k T}{m}\right)^{\frac{1}{2}} u$ and $v d v=\frac{2 k T}{m} u d u$. As $v \rightarrow 0, u \rightarrow 0$ and as $v \rightarrow \infty, u \rightarrow \infty$. Then the above integral becomes
$$
\left\langle v^{n}\right\rangle=4 \pi\left(\frac{m}{2 \pi k T}\right)^{\frac{3}{2}} \int_{0}^{\infty} e^{-u^{2}} u^{n+1}\left(\frac{2 k T}{m}\right)^{\frac{n+1}{2}} \cdot \frac{2 k T}{m} u d u
$$
$$
=4 \pi\left(\frac{m}{2 \pi k T}\right)^{\frac{3}{2}} \cdot\left(\frac{2 k T}{m}\right)^{\frac{n+3}{2}} \int_{0}^{\infty} e^{-u^{2}} u^{n+2} d u
$$
Let $u^{2}=t \cdot$ Then $2 u d u=d t$ As $u \rightarrow 0, t \rightarrow 0$ and as $u \rightarrow \infty, t \rightarrow \infty$. As $u \rightarrow 0, t \rightarrow 0$ and as $u \rightarrow \infty, t \rightarrow \infty$.
$$
\left\langle v^{n}\right\rangle=4 \pi\left(\frac{m}{2 \pi k T}\right)^{\frac{3}{2}} \cdot\left(\frac{2 k T}{m}\right)^{\frac{n+3}{2}} \int_{0}^{\infty} e^{-t} t^{\frac{n+1}{2}} \frac{d t}{2}
$$
$$
=2 \pi\left(\frac{m}{2 \pi k T}\right)^{\frac{3}{2}} \cdot\left(\frac{2 k T}{m}\right)^{\frac{n+3}{2}} \int_{0}^{\infty} e^{-t} t^{\frac{n+3}{2}} d t
$$
$$
=\frac{2 \pi}{\pi \sqrt{\pi}}\left(\frac{2 k T}{m}\right)^{\frac{n+3}{2}-\frac{3}{2}} \Gamma\left(\frac{n+3}{2}\right)
$$
$$
=\frac{2}{\sqrt{\pi}}\left(\frac{2 k T}{m}\right)^{\frac{n}{2}} \Gamma\left(\frac{n+3}{2}\right)
$$
$$
=\left(\frac{2 k T}{m}\right)^{\frac{n}{2}} \frac{\Gamma\left(\frac{n+3}{2}\right)}{\Gamma\left(\frac{3}{2}\right)}
$$


since $\Gamma\left(\dfrac{3}{2}\right)=\dfrac{\sqrt{\pi}}{2}$. Hence 
$$\left\langle v^{n}\right\rangle=\left(\dfrac{2 k T}{m}\right)^{\frac{n}{2}} \frac{\Gamma\left(\frac{n+3}{2}\right)}{\Gamma\left(\frac{3}{2}\right)}$$



\begin{mybox}{13.1.5}
By transforming the integral into a gamma function, show that
$$
-\int_{0}^{1} x^{k} \ln x d x=\frac{1}{(k+1)^{2}}, \quad k>-1
$$
\end{mybox}

$\boxed{\textbf{Solution}}$ Put $x=e^{t} .$ Then $t=\ln x$ and $d x=e^{\prime} d t$. As $x \rightarrow 0, t \rightarrow \infty$ and as $x \rightarrow 1, t \rightarrow 0$.
$$-\int_{0}^{1} x^{k} \ln x d x$$
$$=-\int_{\infty}^{0} e^{k t} t e^{\prime} d t$$
$$=\int_{0}^{\infty} e^{(k+1) t} t d t$$
Now put $-(k+1) t=z .$ Then 
$$d t=-\frac{d z}{(k+1)} $$
As $t \rightarrow 0, z \rightarrow 0$ and as $t \rightarrow \infty, z \rightarrow 0$. Then
$$-\int_{0}^{1} x^{k} \ln x d x$$
$$=\int_{0}^{\infty} e^{(k+1) t} t d t$$
$$=\int_{0}^{\infty} e^{-z}\left(\frac{z}{-(k+1)}\right)\left(\frac{d z}{-(k+1)}\right)$$
$$=\frac{1}{(k+1)^{2}} \int_{0}^{\infty} z e^{-z} d z$$
$$=\frac{1}{(k+1)^{2}} \int_{0}^{\infty} z^{2-1} e^{-z} d z$$
$$=\frac{1}{(k+1)^{2}} \Gamma(2)$$
$$=\frac{1}{(k+1)^{2}} \cdot 1 !$$
$$=\frac{1}{(k+1)^{2}}$$
Hence
$$
-\int_{0}^{1} x^{k} \ln x d x=\frac{1}{(k+1)^{2}}, \quad k>-1
$$


\begin{mybox}{13.1.6}
Show that
$$
\int_{0}^{\infty} e^{-x^{4}} d x=\Gamma\left(\frac{5}{4}\right)
$$
\end{mybox}

$\textbf{\textbf{Solution}}$ Consider $x^{4}=t$ and put $4x^3 dx = dt$ as $t \rightarrow 0$ to $\infty x \rightarrow 0$ to $\infty$ and using
$$
\int_{0}^{\infty} e^{-t} t^{z-1} d t=\Gamma z
$$
and
$$
z \Gamma z=\Gamma(z+1)
$$
the integral takes the form of

$$\begin{aligned} \frac{1}{4} \int_{0}^{\infty} e^{-t} t^{-3 / 4} d t &=\frac{1}{4} \int_{0}^{\infty} e^{-t} t^{1 / 4-1} d t \\ &=\frac{1}{4} \Gamma\left(\frac{1}{4}\right) \\ &=\Gamma\left(\frac{5}{4}\right) \end{aligned}$$



\begin{mybox}{13.1.7}
Show that
$$
\lim _{x \rightarrow 0} \frac{\Gamma(a x)}{\Gamma(x)}=\frac{1}{a}
$$
\end{mybox}

$\boxed{\textbf{Solution}}$ 

$$=\lim _{x \rightarrow 0} \frac{\left(\frac{a x \Gamma(a x)}{a x}\right)}{\left(\frac{x \Gamma(x)}{x}\right)}$$
$$=\lim _{x \rightarrow 0}\left(\frac{\Gamma(a x+1)}{\Gamma(x+1)} \cdot \frac{x}{a x}\right)$$
$$=\frac{1}{a} \lim _{x \rightarrow 0} \frac{\Gamma(a x+1)}{\Gamma(x+1)}$$
$$=\frac{1}{a} \frac{\Gamma(1)}{\Gamma(1)}$$
$$=\frac{1}{a}$$


\begin{mybox}{13.1.8}
Locate the poles of $\Gamma(z)$. Show that they are simple poles and determine the residues.
\end{mybox}
$\boxed{\textbf{Solution}}$
Recall that 
$$\Gamma(z)=\lim _{n \rightarrow \infty} \frac{1 \cdot 2 \cdot 3 \cdot \cdots n}{z(z+1)(z+2) \cdots(z+n)} \cdot n^{2},$$ where $z \neq 0,-1,-2,-3, \cdots$. The denominator shows that $\Gamma(z)$ has simple poles at $z=0,-1,-2,-3, \cdots$
$$\Gamma(z)=\int_{0}^{\infty} e^{-t} t^{z-1} d t$$
$$=\int_{0}^{1} e^{-t} t^{z-1} d t+\int_{1}^{\infty} e^{-t} t^{z-1} d t$$
$$=\int_{0}^{1} t^{z-1} \sum_{n=0}^{\infty} \frac{(-t)^{n}}{n !} d t+\int_{1}^{\infty} e^{-t} t^{z-1} d t$$
$$=\sum_{n=0}^{\infty} \frac{(-1)^{n}}{n !} \int_{0}^{1} t^{n+z-1} d t+\int_{1}^{\infty} e^{-t} t^{z-1} d t$$
$$=\sum_{n=0}^{\infty} \frac{(-1)^{n}}{n !} \cdot\left[\frac{t^{n+z}}{n+z}\right]_{0}^{1}+\int_{1}^{\infty} e^{-t} t^{z-1} d t$$
$$=\sum_{n=0}^{\infty} \frac{(-1)^{n}}{n !} \cdot\left[\frac{1}{n+z}-0\right]+\int_{1}^{\infty} e^{-t} t^{z-1} d t$$
$$=\sum_{n=0}^{\infty} \frac{(-1)^{n}}{n !(n+z)}+\int_{1}^{\infty} e^{-t} t^{z-1} d t$$
The series 
$$\sum_{n=0}^{\infty} \frac{(-1)^{n}}{n !(n+z)}$$ 
shows that the first order poles at all negative integers $z=-n$ has respective residues 
$$\frac{(-1)^{n}}{n !}$$


\begin{mybox}{13.1.10}
Show that, for integer $s$
\begin{enumerate}[$(a)$]
\item $$\int_{0}^{\infty} x^{2 s+1} \exp \left(-a x^{2}\right) d x=\frac{s !}{2 a^{s+1}}$$
\item $$\int_{0}^{\infty} x^{2 s} \exp \left(-a x^{2}\right) d x=\frac{\Gamma\left(s+\frac{1}{2}\right)}{2 a^{s+1 / 2}}=\frac{(2 s-1) ! !}{2^{s+1} a^{s}} \sqrt{\frac{\pi}{a}}$$
\end{enumerate}


\end{mybox}
$\boxed{\textbf{Solution}}$ For $(a)$ Put $a x^{2}=z \cdot$ Then $2 a x d x=d z$. This implies
$$
d x=\frac{d z}{2 \sqrt{a z}}
$$
As $x \rightarrow 0, z \rightarrow 0$ and as $x \rightarrow \infty, z \rightarrow \infty$.
The given integral is
$$
\int_{0}^{\infty} x^{2 s+1} \exp \left(-a x^{2}\right) d x
$$
$$
=\int_{0}^{\infty}\left(\sqrt{\frac{z}{a}}\right)^{2 s+1} e^{-z} \frac{d z}{2 \sqrt{a z}}
$$
$$
=\frac{1}{2 \sqrt{a}} \int_{0}^{\infty}\left(\frac{z}{a}\right)^{\frac{2 s+1}{2}} e^{-z} z^{-\frac{1}{2}} d z
$$
$$
=\frac{1}{2 a^{\frac{1}{2}}} \cdot \frac{1}{a^{\frac{2 s+1}{2}}} \int_{0}^{\infty} e^{-z} z^{\frac{2 s+1}{2}-\frac{1}{2}} d z
$$
$$
=\frac{1}{2 a^{s+1}} \int_{0}^{\infty} e^{-z} z^{s} d z
$$
$$
=\frac{1}{2 a^{s+1}} \int_{0}^{\infty} e^{-z} z^{(s+1)-1} d z
$$
$$
=\frac{1}{2 a^{s+1}} \Gamma(s+1)
$$
since $s$ is an integer, therefore $\Gamma(s+1)=s !$. Hence 
$$\int_{0}^{\infty} x^{2 s+1} \exp \left(-a x^{2}\right) d x=\frac{s !}{2 a^{s+1}}$$ 

$\boxed{\textbf{Solution}}$ For $(b)$ Put $a x^{2}=z \cdot$ Then $2 a x d x=d z$. This implies
$$
d x=\frac{d z}{2 \sqrt{a z}}
$$
As $x \rightarrow 0, z \rightarrow 0$ and as $x \rightarrow \infty, z \rightarrow \infty$.
The given integral is
$$
\int_{0}^{\infty} x^{2 s} \exp \left(-a x^{2}\right) d x
$$
$$
=\int_{0}^{\infty}\left(\sqrt{\frac{z}{a}}\right)^{2 s} e^{-z} \frac{d z}{2 \sqrt{a z}}
$$
$$
=\frac{1}{2 \sqrt{a}} \int_{0}^{\infty}\left(\frac{z}{a}\right)^{s} e^{-z} z^{-\frac{1}{2}} d z
$$
$$
=\frac{1}{2 a^{\frac{1}{2}}} \cdot \frac{1}{a^{s}} \int_{0}^{\infty} e^{-z} z^{s-\frac{1}{2}} d z
$$
$$
=\frac{1}{2 a^{s+\frac{1}{2}}} \int_{0}^{\infty} e^{-z} z^{\left(s+\frac{3}{2}\right)-1} d z
$$
$$
=\frac{1}{2 a^{s+\frac{1}{2}}} \Gamma\left(s+\frac{3}{2}\right)
$$
since 
$$\Gamma\left(s+\frac{1}{2}\right)=\frac{\sqrt{\pi}}{2^{s}} \cdot(2 s-1) ! !$$
$$
=\frac{(2 s-1) ! !}{2^{s+1} a^{s}} \sqrt{\frac{\pi}{a}}
$$
Thus
$$
\int_{0}^{\infty} x^{2 s} \exp \left(-a x^{2}\right) d x=\frac{\Gamma\left(s+\frac{1}{2}\right)}{2 a^{s+\frac{1}{2}}}=\frac{(2 s-1) ! !}{2 a^{s+1} a^{s}} \sqrt{\frac{\pi}{a}}
$$

\begin{mybox}{13.1.11}
Express the coefficient of the $n$ th term of the expansion of $(1+x)^{1 / 2}$ in powers of $x$
\begin{enumerate}[$(a)$]
\item in terms of factorials of integers,
\item in terms of the double factorial (!!) functions.
\end{enumerate} 
$$
A N S . \ a_{n}=(-1)^{n+1} \frac{(2 n-3) !}{2^{2 n-2} n !(n-2) !}=(-1)^{n+1} \frac{(2 n-3) ! !}{(2 n) ! !},\quad n=2,3, \ldots
$$
\end{mybox}

$\boxed{\textbf{Solution}}$ For $(a)$ the $n$ th term of the expansion of $(1+x)^{1 / 2}$ in powers of $x$ is:
$$a_{n}=\left(\begin{array}{c}\frac{1}{2} \\ n-1\end{array}\right)$$
$$=\frac{\frac{1}{2}\left(\frac{1}{2}-1\right)\left(\frac{1}{2}-2\right)\left(\frac{1}{2}-3\right) \cdots\left(\frac{1}{2}-(n-1)\right)}{n !}$$
$$=\frac{\left(\frac{1}{2}\right)\left(-\frac{1}{2}\right)\left(-\frac{3}{2}\right)\left(-\frac{5}{2}\right) \cdots\left(-\frac{2 n-3}{2}\right)}{n !}$$
$$=\frac{(-1)^{n-1}}{n ! 2^{n}}[1.3 .5 \ldots \cdot(2 n-3)]$$
$$=\frac{(-1)^{n+1}}{n ! 2^{n}}\left[\frac{1.2 .3 .4 .5 .6 \cdots \cdot(2 n-4) \cdot(2 n-3)}{2.4 .6 . \cdots .(2 n-4)}\right]$$
$$=\frac{(-1)^{n}}{n ! 2^{n}} \cdot \frac{(2 n-3) !}{(n-2) ! 2^{n-2}}$$
$$=(-1)^{n+1} \cdot \frac{(2 n-3) !}{2^{2 n-2} \cdot n !(n-2) !}$$
Therefore, 
$$
a_{n}=(-1)^{n+1} \cdot \frac{(2 n-3) !}{2^{2 n-2} n !(n-2) !}, \quad  n=1,2,3, \cdots
$$

$\boxed{\textbf{Solution}}$ For $(b)$ the $n$ th term expansion of $(1+x)^{1 / 2}$ 
$$a_{n}=\left(\begin{array}{c}-\frac{1}{2} \\ n-1\end{array}\right)$$
$$=\frac{\frac{1}{2}\left(\frac{1}{2}-1\right)\left(\frac{1}{2}-2\right)\left(\frac{1}{2}-3\right) \cdots\left(\frac{1}{2}-(n-1)\right)}{n !}$$
$$=\frac{\left(\frac{1}{2}\right)\left(-\frac{1}{2}\right)\left(-\frac{3}{2}\right)\left(-\frac{5}{2}\right) \cdots\left(-\frac{2 n-3}{2}\right)}{n !}$$
$$=\frac{(-1)^{n-1}}{n ! 2^{n}}[1.3 .5 \cdot \cdots \cdot(2 n-3)]$$
$$=(-1)^{n+1} \cdot\left[\frac{1.3 .5 \cdots \cdot(2 n-3)}{2.4 .6 \cdot \cdots .2 n}\right]$$
$$=(-1)^{n+1} \cdot \frac{(2 n-3) ! !}{(2 n) ! !}$$
Therefore
$$
a_{n}=(-1)^{n+1} \cdot \frac{(2 n-3) ! !}{(2 n) ! !}, \quad \text { for } n=1,2,3, \cdots
$$

\begin{mybox}{13.1.12}
Express the coefficient of the $n$ th term of the expansion of $(1+x)^{-1 / 2}$ in powers of $x$
\begin{enumerate}[$(a)$]
\item in terms of the factorials of integers,
\item in terms of the double factorial
(!!) functions.
\end{enumerate}
$$
A N S . \quad a_{n}=(-1)^{n} \frac{(2 n) !}{2^{2 n}(n !)^{2}}=(-1)^{n} \frac{(2 n-1) ! !}{(2 n) ! !}, \quad n=1,2,3 \ldots
$$
\end{mybox}

$\boxed{\textbf{Solution}}$ For $(a)$ the $n$ th term of the expansion of $(1+x)^{-1 / 2}$ in powers of $x$ is:
$$
a_{n}=\left(\begin{array}{c}
-\frac{1}{2} \\
n-1
\end{array}\right)
$$
$$
=\frac{-\frac{1}{2}\left(-\frac{1}{2}-1\right)\left(-\frac{1}{2}-2\right)\left(-\frac{1}{2}-3\right) \cdots\left(-\frac{1}{2}-(n-1)\right)}{n !}
$$
$$
=\frac{\left(-\frac{1}{2}\right)\left(-\frac{3}{2}\right)\left(-\frac{5}{2}\right) \cdots\left(-\frac{2 n-1}{2}\right)}{n !}
$$
$$
=\frac{(-1)^{n}}{n ! 2^{n}}[1.3 .5 . \cdots .(2 n-1)]
$$
$$
=\frac{(-1)^{n}}{n ! 2^{n}}\left[\frac{1.2 .3 .4 .5 .6 . \cdots .(2 n-1) \cdot 2 n}{2.4 .6 . \cdots .2 n}\right]
$$
$$=\frac{(-1)^{n}}{n ! 2^{n}} \cdot \frac{(2 n) !}{n ! 2^{n}}$$
$$=(-1)^{n} \cdot \frac{(2 n) !}{2^{2 n} \cdot(n !)^{2}}$$
Therefore, 
$$
a_{n}=(-1)^{n} \cdot \frac{(2 n) !}{2^{2 n} \cdot(n !)^{2}}, \quad \text { for } n=1,2,3, \cdots
$$

$\boxed{\textbf{Solution}}$ For $(b)$ the $n$ th term expansion of $(1+x)^{-1 / 2}$ in powers of $x$ in terms of the double factorial $(!!)$ functions.
$$
a_{n}=\left(\begin{array}{c}
-\frac{1}{2} \\
n-1
\end{array}\right)
$$
$$
=\frac{-\frac{1}{2}\left(-\frac{1}{2}-1\right)\left(-\frac{1}{2}-2\right)\left(-\frac{1}{2}-3\right) \cdots\left(-\frac{1}{2}-(n-1)\right)}{n !}
$$
$$
=\frac{\left(-\frac{1}{2}\right)\left(-\frac{3}{2}\right)\left(-\frac{5}{2}\right) \cdots\left(-\frac{2 n-1}{2}\right)}{n !}
$$
$$=\frac{(-1)^{n}}{n ! 2^{n}}[1.3 .5 \ldots .(2 n-1)]$$
$$=(-1)^{n} \cdot\left[\frac{1.3 .5 \ldots .(2 n-1)}{2.4 .6 . \cdots .2 n}\right]$$
$$=(-1)^{n} \cdot \frac{(2 n-1) ! !}{(2 n) ! !}$$
Therefore
$$
a_{n}=(-1)^{n} \cdot \frac{(2 n-1) ! !}{(2 n) ! !}, \quad \text { for } n=1,2,3, \cdots
$$

\begin{mybox}{13.1.14}
\begin{enumerate}[$(a)$]
\item Show that $\Gamma\left(\frac{1}{2}-n\right) \Gamma\left(\frac{1}{2}+n\right)=(-1)^{n} \pi,$ where $n$ is an integer. 
\item Express $\Gamma\left(\frac{1}{2}+n\right)$ and $\Gamma\left(\frac{1}{2}-n\right)$ separately in terms of $\pi^{1 / 2}$ and a double factorial function.
\end{enumerate}
$$
A N S . \quad \Gamma\left(\frac{1}{2}+n\right)=\frac{(2 n-1) ! !}{2^{n}} \pi^{1 / 2}
$$
\end{mybox}

$\boxed{\textbf{Solution}}$ For $(a)$ recall that 
$$\Gamma(z) \Gamma(1-z)=\frac{\pi}{\sin \pi z}$$
Putting $z=\frac{1}{2}+n$ in the above relation, it becomes
$$
\Gamma\left(\frac{1}{2}+n\right) \Gamma\left(1-\frac{1}{2}-n\right)=\frac{\pi}{\sin \left[\pi\left(\frac{1}{2}+n\right)\right]}
$$
$$
=\frac{\pi}{\cos (n \pi)}
$$
$$
=\frac{\pi}{(-1)^{n}}
$$
since $\cos (n \pi)=(-1)^{n}$ and
$$
=(-1)^{n} \pi
$$
Therefore 
$$\Gamma\left(\frac{1}{2}-n\right) \Gamma\left(\frac{1}{2}+n\right)=(-1)^{n} \pi$$ where $n$ is an integer.

$\boxed{\textbf{Solution}}$ For $(b)$ recall the Legendre's duplication formula,
$$\Gamma(1+z) \Gamma\left(z+\frac{1}{2}\right)=2^{-2 z} \sqrt{\pi} \Gamma(2 z+1)$$
Putting $z=n$ in the above relation, it becomes
$$\Gamma(1+n) \Gamma\left(n+\frac{1}{2}\right)=2^{-2 n} \sqrt{\pi} \Gamma(2 n+1)$$
$$\Gamma\left(n+\frac{1}{2}\right)=\frac{2^{-2 n} \sqrt{\pi} \Gamma(2 n+1)}{\Gamma(1+n)}$$
$$\Gamma\left(n+\frac{1}{2}\right)=\frac{\sqrt{\pi}}{2^{2 n}} \cdot \frac{(2 n) !}{n !}$$
$$\Gamma\left(n+\frac{1}{2}\right)=\frac{\sqrt{\pi}}{2^{2 n}} \cdot \frac{(1.2 .3 .4 .5 \ldots . .2 n)}{(1.2 .3 \ldots n)}$$
$$\Gamma\left(n+\frac{1}{2}\right)=\frac{\sqrt{\pi}}{2^{n}} \cdot \frac{(1.2 .3 .4 .5 \ldots . .2 n)}{(2.4 .6 \ldots . .2 n)}$$
$$\Gamma\left(n+\frac{1}{2}\right)=\frac{\sqrt{\pi}}{2^{n}} \cdot[1.3 .5 \ldots \ldots(2 n-1)]$$
$$
\Gamma\left(\frac{1}{2}+n\right)=\frac{\sqrt{\pi}}{2^{n}} \cdot(2 n-1) ! ! \cdots
$$
From part $(a)$ 
$$
\Gamma\left(\frac{1}{2}-n\right) \Gamma\left(\frac{1}{2}+n\right)=(-1)^{n} \pi
$$
$$\Gamma\left(\frac{1}{2}-n\right)=\frac{(-1)^{n} \pi}{\Gamma\left(\frac{1}{2}+n\right)}$$
$$\Gamma\left(\frac{1}{2}-n\right)=\frac{(-1)^{n} \pi}{\left(\frac{\sqrt{\pi}}{2^{n}} \cdot(2 n-1) ! !\right)}$$
$$
\Gamma\left(\frac{1}{2}-n\right)=\frac{(-1)^{n} \cdot 2^{n} \sqrt{\pi}}{(2 n-1) ! !}
$$
$$
\Gamma\left(\frac{1}{2}+n\right)=\frac{\sqrt{\pi}}{2^{n}} \cdot(2 n-1) ! ! \text { and } \Gamma\left(\frac{1}{2}-n\right)=\frac{(-1)^{n} \cdot 2^{n} \sqrt{\pi}}{(2 n-1) ! !}
$$

\begin{mybox}{13.1.6}
Prove that 
$$|\Gamma(\alpha+i \beta)|=|\Gamma(\alpha)| \prod_{n=0}^{\infty}\left[1+\frac{\beta^{2}}{(\alpha+n)^{2}}\right]^{-1 / 2}$$
\end{mybox}

$\boxed{\textbf{Solution}}$ Recall 
$$
\frac{1}{\Gamma(z)}=z e^{\gamma z} \prod_{n=1}^{\infty}\left(1+\frac{z}{n}\right) e^{-\frac{z}{n}}
$$
Putting $z=\alpha+i \beta$ and $z=\alpha-i \beta$ successively in the above relation, it becomes
$$
\frac{1}{\Gamma(\alpha+i \beta)}=(\alpha+i \beta) e^{\gamma(\alpha+i \beta)} \prod_{n=1}^{\infty}\left(1+\frac{\alpha+i \beta}{n}\right) e^{-\frac{a+i \beta}{n}}
$$
and 
$$
\frac{1}{\Gamma(\alpha-i \beta)}=(\alpha-i \beta) e^{\gamma(\alpha-i \beta)} \prod_{n=1}^{\infty}\left(1+\frac{\alpha-i \beta}{n}\right) e^{\frac{a-i \beta}{n}}
$$
Multiplying these equations it becomes
$$
\frac{1}{\Gamma(\alpha+i \beta)} \cdot \frac{1}{\Gamma(\alpha-i \beta)}=(\alpha+i \beta) e^{\gamma(a+i \beta)} \cdot(\alpha-i \beta) e^{\gamma(a-i \beta)}$$
$$\times \prod_{n=1}^{\infty}\left[\left(1+\frac{\alpha+i \beta}{n}\right) e^{\frac{a+i \beta}{n}} \cdot\left(1+\frac{\alpha-i \beta}{n}\right) e^{\frac{\alpha-i \beta}{n}}\right]
$$
$$
\frac{1}{|\Gamma(\alpha+i \beta)|^{2}}=\left(\alpha^{2}+\beta^{2}\right) e^{2\gamma \alpha} \prod_{n=1}^{\infty} e^{-\frac{2 a}{n}}\left[\left(1+\frac{\alpha+i \beta}{n}\right) \cdot\left(1+\frac{\alpha-i \beta}{n}\right)\right]
$$
$$
=\left(\alpha^{2}+\beta^{2}\right) e^{2 \gamma a} \prod_{n=1}^{\infty}\left[e^{\frac{2 a}{n} \cdot \frac{\left(1+\frac{\alpha+i \beta}{n}\right) \cdot\left(1+\frac{\alpha-i \beta}{n}\right)}{\left(1+\frac{\alpha}{n}\right)^{2}}} \cdot\left(1+\frac{\alpha}{n}\right)^{2}\right]
$$
$$
=\left(\alpha^{2}+\beta^{2}\right) e^{2 \gamma a} \prod_{n=1}^{\infty}\left[e^{-\frac{2 a}{n}} \cdot \frac{\left(1+\frac{\alpha+i \beta}{n}\right) \cdot\left(1+\frac{\alpha-i \beta}{n}\right)}{\left(1+\frac{\alpha}{n}\right)^{2}} \cdot\left(1+\frac{\alpha}{n}\right)^{2}\right]
$$

$$
=\left(\alpha^{2}+\beta^{2}\right) e^{2 \gamma a} \prod_{n=1}^{\infty}\left[e^{-\frac{2 a}{n}} \cdot \frac{\left(1+\frac{\alpha+i \beta}{n}\right) \cdot\left(1+\frac{\alpha-i \beta}{n}\right)}{\left(1+\frac{\alpha}{n}\right)^{2}} \cdot\left(1+\frac{\alpha}{n}\right)^{2}\right]
$$

$$
=\left(\frac{\alpha^{2}+\beta^{2}}{\alpha^{2}}\right)\left(\alpha e^{\gamma \alpha} \prod_{n=1}^{\infty}\left[e^{-\frac{a}{n}} \cdot\left(1+\frac{\alpha}{n}\right)\right]\right)^{2} \prod_{n=1}^{\infty}\left[\frac{\left(1+\frac{2 \alpha}{n}+\frac{\alpha^{2}+\beta^{2}}{n^{2}}\right)}{\frac{(n+\alpha)^{2}}{n^{2}}}\right]
$$


$$
=\left(1+\frac{\beta^{2}}{\alpha^{2}}\right) \frac{1}{\Gamma(\alpha)^{2}} \prod_{n=1}^{\infty}\left[\frac{\left(1+2 \alpha n+\alpha^{2}+\beta^{2}\right)}{(n+\alpha)^{2}}\right]
$$

$$
=\frac{1}{\Gamma(\alpha)^{2}} \cdot\left(1+\frac{\beta^{2}}{\alpha^{2}}\right) \prod_{n=1}^{\infty}\left[\frac{(n+\alpha)^{2}+\beta^{2}}{(n+\alpha)^{2}}\right]
$$

$$
=\frac{1}{\Gamma(\alpha)^{2}} \cdot\left(1+\frac{\beta^{2}}{\alpha^{2}}\right) \prod_{n=1}^{\infty}\left[1+\frac{\beta^{2}}{(n+\alpha)^{2}}\right]
$$

$$
=\frac{1}{\Gamma(\alpha)^{2}} \prod_{n=0}^{\infty}\left[1+\frac{\beta^{2}}{(n+\alpha)^{2}}\right]
$$
Hence
$$
\frac{1}{|\Gamma(\alpha+i \beta)|^{2}}=\frac{1}{\Gamma(\alpha)^{2}} \prod_{n=0}^{\infty}\left[1+\frac{\beta^{2}}{(n+\alpha)^{2}}\right]
$$

$$
\frac{1}{|\Gamma(\alpha+i \beta)|}=\frac{1}{|\Gamma(\alpha)|} \prod_{n=0}^{\infty}\left[1+\frac{\beta^{2}}{(n+\alpha)^{2}}\right]^{\frac{1}{2}}
$$

$$
|\Gamma(\alpha+i \beta)|=|\Gamma(\alpha)| \prod_{n=0}^{\infty}\left[1+\frac{\beta^{2}}{(\alpha+n)^{2}}\right]^{-\frac{1}{2}}
$$




\begin{mybox}{13.1.17}

Show that for $n$, a positive integer,
$$
|\Gamma(n+i b+1)|=\left(\frac{\pi b}{\sinh \pi b}\right)^{1 / 2} \prod_{s=1}^{n}\left(s^{2}+b^{2}\right)^{1 / 2}
$$
\end{mybox}

$\boxed{\textbf{Solution}}$ Recall 
$$
\frac{1}{\Gamma(z)}=z e^{\gamma z} \prod_{n=1}^{\infty}\left(1+\frac{z}{n}\right) e^{-\frac{z}{n}}
$$
Putting $z=\alpha+i \beta$ and $z=\alpha-i \beta$ successively in the above relation, it becomes
$$
\frac{1}{\Gamma(\alpha+i \beta)}=(\alpha+i \beta) e^{\gamma(\alpha+i \beta)} \prod_{n=1}^{\infty}\left(1+\frac{\alpha+i \beta}{n}\right) e^{-\frac{a+i \beta}{n}}
$$
and 
$$
\frac{1}{\Gamma(\alpha-i \beta)}=(\alpha-i \beta) e^{\gamma(\alpha-i \beta)} \prod_{n=1}^{\infty}\left(1+\frac{\alpha-i \beta}{n}\right) e^{\frac{a-i \beta}{n}}
$$
Multiplying these equations it becomes
$$
\frac{1}{\Gamma(\alpha+i \beta)} \cdot \frac{1}{\Gamma(\alpha-i \beta)}=(\alpha+i \beta) e^{\gamma(a+i \beta)} \cdot(\alpha-i \beta) e^{\gamma(a-i \beta)}$$
$$\times \prod_{n=1}^{\infty}\left[\left(1+\frac{\alpha+i \beta}{n}\right) e^{\frac{a+i \beta}{n}} \cdot\left(1+\frac{\alpha-i \beta}{n}\right) e^{\frac{\alpha-i \beta}{n}}\right]
$$
$$
\frac{1}{|\Gamma(\alpha+i \beta)|^{2}}=\left(\alpha^{2}+\beta^{2}\right) e^{2\gamma \alpha} \prod_{n=1}^{\infty} e^{-\frac{2 a}{n}}\left[\left(1+\frac{\alpha+i \beta}{n}\right) \cdot\left(1+\frac{\alpha-i \beta}{n}\right)\right]
$$
$$
=\left(\alpha^{2}+\beta^{2}\right) e^{2 \gamma a} \prod_{n=1}^{\infty}\left[e^{\frac{2 a}{n} \cdot \frac{\left(1+\frac{\alpha+i \beta}{n}\right) \cdot\left(1+\frac{\alpha-i \beta}{n}\right)}{\left(1+\frac{\alpha}{n}\right)^{2}}} \cdot\left(1+\frac{\alpha}{n}\right)^{2}\right]
$$
$$
=\left(\alpha^{2}+\beta^{2}\right) e^{2 \gamma a} \prod_{n=1}^{\infty}\left[e^{-\frac{2 a}{n}} \cdot \frac{\left(1+\frac{\alpha+i \beta}{n}\right) \cdot\left(1+\frac{\alpha-i \beta}{n}\right)}{\left(1+\frac{\alpha}{n}\right)^{2}} \cdot\left(1+\frac{\alpha}{n}\right)^{2}\right]
$$

$$
=\left(\alpha^{2}+\beta^{2}\right) e^{2 \gamma a} \prod_{n=1}^{\infty}\left[e^{-\frac{2 a}{n}} \cdot \frac{\left(1+\frac{\alpha+i \beta}{n}\right) \cdot\left(1+\frac{\alpha-i \beta}{n}\right)}{\left(1+\frac{\alpha}{n}\right)^{2}} \cdot\left(1+\frac{\alpha}{n}\right)^{2}\right]
$$

$$
=\left(\frac{\alpha^{2}+\beta^{2}}{\alpha^{2}}\right)\left(\alpha e^{\gamma \alpha} \prod_{n=1}^{\infty}\left[e^{-\frac{a}{n}} \cdot\left(1+\frac{\alpha}{n}\right)\right]\right)^{2} \prod_{n=1}^{\infty}\left[\frac{\left(1+\frac{2 \alpha}{n}+\frac{\alpha^{2}+\beta^{2}}{n^{2}}\right)}{\frac{(n+\alpha)^{2}}{n^{2}}}\right]
$$


$$
=\left(1+\frac{\beta^{2}}{\alpha^{2}}\right) \frac{1}{\Gamma(\alpha)^{2}} \prod_{n=1}^{\infty}\left[\frac{\left(1+2 \alpha n+\alpha^{2}+\beta^{2}\right)}{(n+\alpha)^{2}}\right]
$$

$$
=\frac{1}{\Gamma(\alpha)^{2}} \cdot\left(1+\frac{\beta^{2}}{\alpha^{2}}\right) \prod_{n=1}^{\infty}\left[\frac{(n+\alpha)^{2}+\beta^{2}}{(n+\alpha)^{2}}\right]
$$

$$
=\frac{1}{\Gamma(\alpha)^{2}} \cdot\left(1+\frac{\beta^{2}}{\alpha^{2}}\right) \prod_{n=1}^{\infty}\left[1+\frac{\beta^{2}}{(n+\alpha)^{2}}\right]
$$

$$
=\frac{1}{\Gamma(\alpha)^{2}} \prod_{n=0}^{\infty}\left[1+\frac{\beta^{2}}{(n+\alpha)^{2}}\right]
$$
Hence
$$
\frac{1}{|\Gamma(\alpha+i \beta)|^{2}}=\frac{1}{\Gamma(\alpha)^{2}} \prod_{n=0}^{\infty}\left[1+\frac{\beta^{2}}{(n+\alpha)^{2}}\right]
$$
Now put $\alpha=1$ and $\beta=b$ in the above identity. Then it becomes
$$
\frac{1}{|\Gamma(1+i b)|^{2}}=\frac{1}{\Gamma(1)^{2}} \prod_{n=0}^{\infty}\left[1+\frac{b^{2}}{(n+1)^{2}}\right]
$$
$$
=\prod_{n=0}^{\infty}\left[1+\frac{b^{2}}{(n+1)^{2}}\right], \quad \text { as } \quad \Gamma(1)=1
$$

$$
=\prod_{n=0}^{\infty}\left[1-\frac{(i b \pi)^{2}}{(n+1)^{2} \pi^{2}}\right]
$$

$$
=\prod_{n=1}^{\infty}\left[1-\frac{(i b \pi)^{2}}{n^{2} \pi^{2}}\right]
$$
$$
=\frac{1}{(i b \pi)}\left\{(i b \pi) \prod_{n=1}^{\infty}\left[1-\frac{(i b \pi)^{2}}{n^{2} \pi^{2}}\right]\right\}
$$
$$
=\frac{1}{i b \pi} \cdot \sin (i b \pi)
$$
Using the identy
$$
\sin z=z \prod_{n=1}^{\infty}\left[1-\frac{z^{2}}{n^{2} \pi^{2}}\right] \quad \text{for} z=ib\pi
$$
$$=\frac{1}{i b \pi} \cdot i \sinh (b \pi)$$
$$=\frac{\sinh (b \pi)}{b \pi}$$
$$\frac{1}{|\Gamma(1+i b)|^{2}}=\frac{\sinh (b \pi)}{b \pi}$$
$$|\Gamma(1+i b)|^{2}=\frac{b \pi}{\sinh (b \pi)} .$$
since $n$ is an integer, therefore
$$\Gamma(n+i b+1)=\Gamma(\{1+i b+(n-1)\}+1)$$
$$=\{1+i b+(n-1)\} \Gamma(\{1+i b+(n-1)\})$$
$$
(1+i b)(2+i b)(3+i b) \cdots(n+i b) \Gamma(1+i b)
$$
$$
\Gamma(n+i b+1)=(1+i b)(2+i b)(3+i b) \cdots(n+i b) \Gamma(1+i b)
$$
$$
\Gamma(n-i b+1)=(1-i b)(2-i b)(3-i b) \cdots(n-i b) \Gamma(1-i b)
$$
$$|\Gamma(n+i b+1)|^{2}$$
$$=\Gamma(n+i b+1) \Gamma(n-i b+1)$$
$$=(1+i b)(2+i b)(3+i b) \cdots(n+i b) \Gamma(1+i b) \times(1-i b)(2-i b)(3-i b) \cdots(n-i b) \Gamma(1-i b)$$
$$=\{(1+i b)(1-i b)\}\{(2+i b)(2-i b)\}\{(3+i b)(3-i b)\} \cdots\{(n+i b)(n-i b)\} \Gamma(1+i b) \Gamma(1-i b)$$
$$=\left(1^{2}+b^{2}\right)\left(2^{2}+b^{2}\right)\left(3^{2}+b^{2}\right) \cdots\left(n^{2}+b^{2}\right)|\Gamma(1+i b)|^{2}$$
$$
=\prod_{s=1}^{n}\left(s^{2}+b^{2}\right) \times \frac{b \pi}{\sinh (b \pi)}
$$
Hence 
$$|\Gamma(n+i b+1)|^{2}=\prod_{s=1}^{n}\left(s^{2}+b^{2}\right) \times \frac{b \pi}{\sinh (b \pi)}$$
This gives 
$$|\Gamma(n+i b+1)|=\left(\frac{b \pi}{\sinh (b \pi)}\right)^{\frac{1}{2}} \prod_{s=1}^{n}\left(s^{2}+b^{2}\right)^{\frac{1}{2}} $$



\begin{mybox}{13.1.18}
Show that for all real values of $x$ and $y,|\Gamma(x)| \geq|\Gamma(x+i y)|$
\end{mybox}


$\boxed{\textbf{Solution}}$ Recall 
$$
\frac{1}{\Gamma(z)}=z e^{\gamma z} \prod_{n=1}^{\infty}\left(1+\frac{z}{n}\right) e^{-\frac{z}{n}}
$$
Putting $z=\alpha+i \beta$ and $z=\alpha-i \beta$ successively in the above relation, it becomes
$$
\frac{1}{\Gamma(\alpha+i \beta)}=(\alpha+i \beta) e^{\gamma(\alpha+i \beta)} \prod_{n=1}^{\infty}\left(1+\frac{\alpha+i \beta}{n}\right) e^{-\frac{a+i \beta}{n}}
$$
and 
$$
\frac{1}{\Gamma(\alpha-i \beta)}=(\alpha-i \beta) e^{\gamma(\alpha-i \beta)} \prod_{n=1}^{\infty}\left(1+\frac{\alpha-i \beta}{n}\right) e^{\frac{a-i \beta}{n}}
$$
Multiplying these equations it becomes
$$
\frac{1}{\Gamma(\alpha+i \beta)} \cdot \frac{1}{\Gamma(\alpha-i \beta)}=(\alpha+i \beta) e^{\gamma(a+i \beta)} \cdot(\alpha-i \beta) e^{\gamma(a-i \beta)}$$
$$\times \prod_{n=1}^{\infty}\left[\left(1+\frac{\alpha+i \beta}{n}\right) e^{\frac{a+i \beta}{n}} \cdot\left(1+\frac{\alpha-i \beta}{n}\right) e^{\frac{\alpha-i \beta}{n}}\right]
$$
$$
\frac{1}{|\Gamma(\alpha+i \beta)|^{2}}=\left(\alpha^{2}+\beta^{2}\right) e^{2\gamma \alpha} \prod_{n=1}^{\infty} e^{-\frac{2 a}{n}}\left[\left(1+\frac{\alpha+i \beta}{n}\right) \cdot\left(1+\frac{\alpha-i \beta}{n}\right)\right]
$$
$$
=\left(\alpha^{2}+\beta^{2}\right) e^{2 \gamma a} \prod_{n=1}^{\infty}\left[e^{\frac{2 a}{n} \cdot \frac{\left(1+\frac{\alpha+i \beta}{n}\right) \cdot\left(1+\frac{\alpha-i \beta}{n}\right)}{\left(1+\frac{\alpha}{n}\right)^{2}}} \cdot\left(1+\frac{\alpha}{n}\right)^{2}\right]
$$
$$
=\left(\alpha^{2}+\beta^{2}\right) e^{2 \gamma a} \prod_{n=1}^{\infty}\left[e^{-\frac{2 a}{n}} \cdot \frac{\left(1+\frac{\alpha+i \beta}{n}\right) \cdot\left(1+\frac{\alpha-i \beta}{n}\right)}{\left(1+\frac{\alpha}{n}\right)^{2}} \cdot\left(1+\frac{\alpha}{n}\right)^{2}\right]
$$

$$
=\left(\alpha^{2}+\beta^{2}\right) e^{2 \gamma a} \prod_{n=1}^{\infty}\left[e^{-\frac{2 a}{n}} \cdot \frac{\left(1+\frac{\alpha+i \beta}{n}\right) \cdot\left(1+\frac{\alpha-i \beta}{n}\right)}{\left(1+\frac{\alpha}{n}\right)^{2}} \cdot\left(1+\frac{\alpha}{n}\right)^{2}\right]
$$

$$
=\left(\frac{\alpha^{2}+\beta^{2}}{\alpha^{2}}\right)\left(\alpha e^{\gamma \alpha} \prod_{n=1}^{\infty}\left[e^{-\frac{a}{n}} \cdot\left(1+\frac{\alpha}{n}\right)\right]\right)^{2} \prod_{n=1}^{\infty}\left[\frac{\left(1+\frac{2 \alpha}{n}+\frac{\alpha^{2}+\beta^{2}}{n^{2}}\right)}{\frac{(n+\alpha)^{2}}{n^{2}}}\right]
$$


$$
=\left(1+\frac{\beta^{2}}{\alpha^{2}}\right) \frac{1}{\Gamma(\alpha)^{2}} \prod_{n=1}^{\infty}\left[\frac{\left(1+2 \alpha n+\alpha^{2}+\beta^{2}\right)}{(n+\alpha)^{2}}\right]
$$

$$
=\frac{1}{\Gamma(\alpha)^{2}} \cdot\left(1+\frac{\beta^{2}}{\alpha^{2}}\right) \prod_{n=1}^{\infty}\left[\frac{(n+\alpha)^{2}+\beta^{2}}{(n+\alpha)^{2}}\right]
$$

$$
=\frac{1}{\Gamma(\alpha)^{2}} \cdot\left(1+\frac{\beta^{2}}{\alpha^{2}}\right) \prod_{n=1}^{\infty}\left[1+\frac{\beta^{2}}{(n+\alpha)^{2}}\right]
$$

$$
=\frac{1}{\Gamma(\alpha)^{2}} \prod_{n=0}^{\infty}\left[1+\frac{\beta^{2}}{(n+\alpha)^{2}}\right]
$$
Hence
$$
\frac{1}{|\Gamma(\alpha+i \beta)|^{2}}=\frac{1}{\Gamma(\alpha)^{2}} \prod_{n=0}^{\infty}\left[1+\frac{\beta^{2}}{(n+\alpha)^{2}}\right]
$$
Now put $\alpha=x$ and $\beta=y$ in the above identity. Then it becomes
$$
\frac{1}{|\Gamma(x+i y)|^{2}}=\frac{1}{\Gamma(x)^{2}} \prod_{n=0}^{\infty}\left[1+\frac{\beta^{2}}{(n+x)^{2}}\right]
$$
$$
\left|\frac{\Gamma(x)}{\Gamma(x+i y)}\right|^{2}=\prod_{n=0}^{\infty}\left[1+\frac{\beta^{2}}{(n+x)^{2}}\right]
$$
$$
\left|\frac{\Gamma(x)}{\Gamma(x+i y)}\right|^{2} \geq 1, \quad \text { since }\quad 1+\frac{\beta^{2}}{(n+x)^{2}} \geq 1
$$
$$\left|\frac{\Gamma(x)}{\Gamma(x+i y)}\right| \geq 1$$
$$|\Gamma(x)| \geq|\Gamma(x+i y)|$$
Hence is proved








\begin{mybox}{13.1.19}
Show that 
$$\left|\Gamma(\frac{1}{2}+i y)\right|^{2}=\frac{\pi}{\cosh \pi y}$$
\end{mybox}


$\boxed{\textbf{Solution}}$ Recall 
$$
\frac{1}{\Gamma(z)}=z e^{\gamma z} \prod_{n=1}^{\infty}\left(1+\frac{z}{n}\right) e^{-\frac{z}{n}}
$$
Putting $z=\alpha+i \beta$ and $z=\alpha-i \beta$ successively in the above relation, it becomes
$$
\frac{1}{\Gamma(\alpha+i \beta)}=(\alpha+i \beta) e^{\gamma(\alpha+i \beta)} \prod_{n=1}^{\infty}\left(1+\frac{\alpha+i \beta}{n}\right) e^{-\frac{a+i \beta}{n}}
$$
and 
$$
\frac{1}{\Gamma(\alpha-i \beta)}=(\alpha-i \beta) e^{\gamma(\alpha-i \beta)} \prod_{n=1}^{\infty}\left(1+\frac{\alpha-i \beta}{n}\right) e^{\frac{a-i \beta}{n}}
$$
Multiplying these equations it becomes
$$
\frac{1}{\Gamma(\alpha+i \beta)} \cdot \frac{1}{\Gamma(\alpha-i \beta)}=(\alpha+i \beta) e^{\gamma(a+i \beta)} \cdot(\alpha-i \beta) e^{\gamma(a-i \beta)}$$
$$\times \prod_{n=1}^{\infty}\left[\left(1+\frac{\alpha+i \beta}{n}\right) e^{\frac{a+i \beta}{n}} \cdot\left(1+\frac{\alpha-i \beta}{n}\right) e^{\frac{\alpha-i \beta}{n}}\right]
$$
$$
\frac{1}{|\Gamma(\alpha+i \beta)|^{2}}=\left(\alpha^{2}+\beta^{2}\right) e^{2\gamma \alpha} \prod_{n=1}^{\infty} e^{-\frac{2 a}{n}}\left[\left(1+\frac{\alpha+i \beta}{n}\right) \cdot\left(1+\frac{\alpha-i \beta}{n}\right)\right]
$$
$$
=\left(\alpha^{2}+\beta^{2}\right) e^{2 \gamma a} \prod_{n=1}^{\infty}\left[e^{\frac{2 a}{n} \cdot \frac{\left(1+\frac{\alpha+i \beta}{n}\right) \cdot\left(1+\frac{\alpha-i \beta}{n}\right)}{\left(1+\frac{\alpha}{n}\right)^{2}}} \cdot\left(1+\frac{\alpha}{n}\right)^{2}\right]
$$
$$
=\left(\alpha^{2}+\beta^{2}\right) e^{2 \gamma a} \prod_{n=1}^{\infty}\left[e^{-\frac{2 a}{n}} \cdot \frac{\left(1+\frac{\alpha+i \beta}{n}\right) \cdot\left(1+\frac{\alpha-i \beta}{n}\right)}{\left(1+\frac{\alpha}{n}\right)^{2}} \cdot\left(1+\frac{\alpha}{n}\right)^{2}\right]
$$

$$
=\left(\alpha^{2}+\beta^{2}\right) e^{2 \gamma a} \prod_{n=1}^{\infty}\left[e^{-\frac{2 a}{n}} \cdot \frac{\left(1+\frac{\alpha+i \beta}{n}\right) \cdot\left(1+\frac{\alpha-i \beta}{n}\right)}{\left(1+\frac{\alpha}{n}\right)^{2}} \cdot\left(1+\frac{\alpha}{n}\right)^{2}\right]
$$

$$
=\left(\frac{\alpha^{2}+\beta^{2}}{\alpha^{2}}\right)\left(\alpha e^{\gamma \alpha} \prod_{n=1}^{\infty}\left[e^{-\frac{a}{n}} \cdot\left(1+\frac{\alpha}{n}\right)\right]\right)^{2} \prod_{n=1}^{\infty}\left[\frac{\left(1+\frac{2 \alpha}{n}+\frac{\alpha^{2}+\beta^{2}}{n^{2}}\right)}{\frac{(n+\alpha)^{2}}{n^{2}}}\right]
$$


$$
=\left(1+\frac{\beta^{2}}{\alpha^{2}}\right) \frac{1}{\Gamma(\alpha)^{2}} \prod_{n=1}^{\infty}\left[\frac{\left(1+2 \alpha n+\alpha^{2}+\beta^{2}\right)}{(n+\alpha)^{2}}\right]
$$

$$
=\frac{1}{\Gamma(\alpha)^{2}} \cdot\left(1+\frac{\beta^{2}}{\alpha^{2}}\right) \prod_{n=1}^{\infty}\left[\frac{(n+\alpha)^{2}+\beta^{2}}{(n+\alpha)^{2}}\right]
$$

$$
=\frac{1}{\Gamma(\alpha)^{2}} \cdot\left(1+\frac{\beta^{2}}{\alpha^{2}}\right) \prod_{n=1}^{\infty}\left[1+\frac{\beta^{2}}{(n+\alpha)^{2}}\right]
$$

$$
=\frac{1}{\Gamma(\alpha)^{2}} \prod_{n=0}^{\infty}\left[1+\frac{\beta^{2}}{(n+\alpha)^{2}}\right]
$$
Hence
$$
\frac{1}{|\Gamma(\alpha+i \beta)|^{2}}=\frac{1}{\Gamma(\alpha)^{2}} \prod_{n=0}^{\infty}\left[1+\frac{\beta^{2}}{(n+\alpha)^{2}}\right]
$$

Now put $\alpha=\frac{1}{2}$ and $\beta=y$ in the above identity. Then it becomes
$$
\frac{1}{\left|\Gamma\left(\frac{1}{2}+i y\right)\right|^{2}}=\frac{1}{\Gamma\left(\frac{1}{2}\right)^{2}} \prod_{n=0}^{\infty}\left[1+\frac{y^{2}}{\left(n+\frac{1}{2}\right)^{2}}\right]
$$
$$
\frac{1}{\left|\Gamma\left(\frac{1}{2}+i y\right)\right|^{2}}=\frac{1}{\pi} \prod_{n=0}^{\infty}\left[1+\frac{y^{2}}{\left(n+\frac{1}{2}\right)^{2}}\right]
$$
since $\Gamma\left(\frac{1}{2}\right)=\sqrt{\pi}$
$$
\frac{1}{\left|\Gamma\left(\frac{1}{2}+i y\right)\right|^{2}}=\frac{1}{\pi} \prod_{n=0}^{\infty}\left[1+\frac{y^{2}}{\left(n+\frac{1}{2}\right)^{2}}\right]
$$
Recall
$$
\cos z=\prod_{n=1}^{\infty}\left[1-\frac{z^{2}}{\left(n-\frac{1}{2}\right)^{2} \pi^{2}}\right]
$$
and putting $z=i \pi y$ it becomes
$$
\cos (i \pi y)=\prod_{n=1}^{\infty}\left[1-\frac{i^{2} \pi^{2} y^{2}}{\left(n-\frac{1}{2}\right)^{2} \pi^{2}}\right]
$$
$$
\cosh (\pi y)=\prod_{n=1}^{\infty}\left[1+\frac{y^{2}}{\left(n-\frac{1}{2}\right)^{2}}\right]
$$
$$
\cosh (\pi y)=\prod_{n=0}^{\infty}\left[1+\frac{y^{2}}{\left(n+1-\frac{1}{2}\right)^{2}}\right]
$$
$$
\cosh (\pi y)=\prod_{n=0}^{\infty}\left[1+\frac{y^{2}}{\left(n+\frac{1}{2}\right)^{2}}\right]
$$
$$
\frac{1}{\left|\Gamma\left(\frac{1}{2}+i y\right)\right|^{2}}=\frac{1}{\pi} \cosh (\pi y)
$$

\begin{mybox}{13.1.20}
The probability density associated with the normal distribution of statistics is given by
$$
f(x)=\frac{1}{\sigma(2 \pi)^{1 / 2}} \exp \left[-\frac{(x-\mu)^{2}}{2 \sigma^{2}}\right]
$$
with $(-\infty, \infty)$ for the range of $x$. Show that
(a) 
\begin{enumerate}[$(a)$]
\item $\langle x\rangle,$ the mean value of $x,$ is equal to $\mu$
\item the standard deviation $\left(\left\langle x^{2}\right\rangle-\langle x\rangle^{2}\right)^{1 / 2}$ is given by $\sigma$.
\end{enumerate}

\end{mybox}

$\boxed{\textbf{Solution}}$ For $(a)$ For the mean
$$
\langle x\rangle=\int_{-\infty}^{\infty} x f(x) d x
$$
$$
=\int_{-\infty}^{\infty} x \cdot \frac{1}{\sigma(2 \pi)^{\frac{1}{2}}} \exp \left[-\frac{(x-\mu)^{2}}{2 \sigma^{2}}\right] d x
$$
$$
=\frac{1}{\sigma(2 \pi)^{\frac{1}{2}}} \int_{-\infty}^{\infty} x e^{\frac{(x-\mu)^{2}}{2 \sigma^{2}}} d x
$$
Put $x-\mu=y .$ Then $d x=d y .$ As $x \rightarrow 0, y \rightarrow 0$ and $x \rightarrow \infty, y \rightarrow \infty$.
$$
\langle x\rangle=\frac{1}{\sigma(2 \pi)^{\frac{1}{2}}} \int_{-\infty}^{\infty} x e^{\frac{(x-\mu)^{2}}{2 \sigma^{2}}} d x
$$
$$
=\frac{1}{\sigma(2 \pi)^{\frac{1}{2}}} \int_{-\infty}^{\infty}(\mu+y) e^{-\frac{y^{2}}{2 \sigma^{2}}} d y
$$
$$
=\frac{1}{\sigma(2 \pi)^{\frac{1}{2}}} \int_{-\infty}^{\infty}(\mu+y) e^{\frac{y^{2}}{2 \sigma^{2}}} d y
$$
$$
=\frac{\mu}{\sigma(2 \pi)^{\frac{1}{2}}} \int_{-\infty}^{\infty} e^{-\frac{y^{2}}{2 \sigma^{2}}} d y+\frac{1}{\sigma(2 \pi)^{\frac{1}{2}}} \int_{-\infty}^{\infty} y e^{-\frac{y^{2}}{2 \sigma^{2}}} d y
$$
since $e^{-\frac{y^{2}}{2 \sigma^{2}}}$ is an even function, therefore 
$$\int_{-\infty}^{\infty} e^{\frac{y^{2}}{2 \sigma^{2}}} d y=2 \int_{0}^{\infty} e^{-\frac{y^{2}}{2 \sigma^{2}}} d y$$
and since $y e^{-\frac{y^{2}}{2 \sigma^{2}}}$ is an odd
function, therefore 
$$\int_{-\infty}^{\infty} y e^{\frac{y^{2}}{2 \sigma^{2}}} d y=0$$
Therefore, the integral becomes
$$
\langle x\rangle=\frac{2 \mu}{\sigma(2 \pi)^{\frac{1}{2}}} \int_{0}^{\infty} e^{-\frac{y^{2}}{2 \sigma^{2}}} d y
$$
Put $\frac{y^{2}}{2 \sigma^{2}}=z,$ then $2 y d y=2 \sigma^{2} d z \cdot$ This implies $d y=\frac{\sigma^{2}}{y} d z,$ that is, $d y=\frac{\sigma}{\sqrt{2}} z^{-\frac{1}{2}} d z$
As $y \rightarrow 0, z \rightarrow 0$ and $y \rightarrow \infty, z \rightarrow \infty$. Therefore
$$\langle x\rangle=\frac{2 \mu}{\sigma(2 \pi)^{\frac{1}{2}}} \int_{0}^{\infty} e^{\frac{y^{2}}{2 \sigma^{2}}} d y$$
$$=\frac{2 \mu}{\sigma(2 \pi)^{\frac{1}{2}}} \int_{0}^{\infty} e^{-z} \frac{\sigma}{\sqrt{2}} z^{-\frac{1}{2}} d z$$
$$=\frac{2 \mu}{\sigma(2 \pi)^{\frac{1}{2}}} \cdot \frac{\sigma}{\sqrt{2}} \int_{0}^{\infty} e^{-z} z^{\frac{1}{2}-1} d z$$
$$=\frac{2 \mu}{\sigma(2 \pi)^{\frac{1}{2}}} \cdot \frac{\sigma}{\sqrt{2}} \Gamma\left(\frac{1}{2}\right)$$
$$=\frac{2 \mu}{\sigma(2 \pi)^{\frac{1}{2}}} \cdot \frac{\sigma}{\sqrt{2}} \sqrt{\pi}$$
$$=\frac{2 \mu}{\sigma(2 \pi)^{\frac{1}{2}}} \cdot \frac{\sigma}{\sqrt{2}} \sqrt{\pi}$$
$$=\mu$$
$\boxed{\textbf{Solution}}$ For $(b)$ we start saying
$$
\left\langle x^{2}\right\rangle=\int_{0}^{\infty} x^{2} f(x) d x
$$
$$
=\int_{-\infty}^{\infty} x^{2} \cdot \frac{1}{\sigma(2 \pi)^{\frac{1}{2}}} \exp \left[-\frac{(x-\mu)^{2}}{2 \sigma^{2}}\right] d x
$$
$$
=\frac{1}{\sigma(2 \pi)^{\frac{1}{2}}} \int_{-\infty}^{\infty} x^{2} e^{\frac{(x-\mu)^{2}}{2 \sigma^{2}}} d x
$$
Put $x-\mu=y .$ Then $d x=d y .$ As $x \rightarrow 0, y \rightarrow 0$ and $x \rightarrow \infty, y \rightarrow \infty$.
$$
\left\langle x^{2}\right\rangle=\frac{1}{\sigma(2 \pi)^{\frac{1}{2}}} \int_{-\infty}^{\infty} x^{2} e^{-\frac{(x-\mu)^{2}}{2 \sigma^{2}}} d x
$$
$$
=\frac{1}{\sigma(2 \pi)^{\frac{1}{2}}} \int_{-\infty}^{\infty}(\mu+y)^{2} e^{-\frac{y^{2}}{2 \sigma^{2}}} d y
$$
$$
=\frac{1}{\sigma(2 \pi)^{\frac{1}{2}}} \int_{-\infty}^{\infty}\left(\mu^{2}+2 \mu y+y^{2}\right) e^{\frac{y^{2}}{2 \sigma^{2}}} d y
$$
$$
=\frac{\mu^{2}}{\sigma(2 \pi)^{\frac{1}{2}}} \int_{-\infty}^{\infty} e^{-\frac{y^{2}}{2 \sigma^{2}}} d y+\frac{2 \mu}{\sigma(2 \pi)^{\frac{1}{2}}} \int_{-\infty}^{\infty} y e^{-\frac{y^{2}}{2 \sigma^{2}}} d y+\frac{1}{\sigma(2 \pi)^{\frac{1}{2}}} \int_{-\infty}^{\infty} y^{2} e^{-\frac{y^{2}}{2 \sigma^{2}}} d y
$$
since $e^{-\frac{y^{2}}{2 \sigma^{2}}}$ is an even function, therefore
$$
\int_{-\infty}^{\infty} e^{-\frac{y^{2}}{2 \sigma^{2}}} d y=2 \int_{0}^{\infty} e^{-\frac{y^{2}}{2 \sigma^{2}}} d y
$$
since $y e^{-\frac{y^{2}}{2 \sigma^{2}}}$ is an odd function, therefore
$$
\int_{-\infty}^{\infty} y e^{-\frac{y^{2}}{2 \sigma^{2}}} d y=0
$$
since $y e^{-\frac{y^{2}}{2 \sigma^{2}}}$ is an odd function, therefore
$$
\int_{-\infty}^{\infty} y^{2} e^{-\frac{y^{2}}{2 \sigma^{2}}} d y=2 \int_{0}^{\infty} y^{2} e^{-\frac{y^{2}}{2 \sigma^{2}}} d y
$$
Therefore the above integral becomes
$$
\left\langle x^{2}\right\rangle=\frac{2 \mu^{2}}{\sigma(2 \pi)^{\frac{1}{2}}} \int_{0}^{\infty} e^{\frac{y^{2}}{2 \sigma^{2}}} d y+\frac{2}{\sigma(2 \pi)^{\frac{1}{2}}} \int_{0}^{\infty} y^{2} e^{\frac{y^{2}}{2 \sigma^{2}}} d y
$$
Put $\frac{y^{2}}{2 \sigma^{2}}=z,$ then $2 y d y=2 \sigma^{2} d z$. This implies $d y=\frac{\sigma^{2}}{y} d z,$ that is, $d y=\frac{\sigma}{\sqrt{2}} z^{-\frac{1}{2}} d z$
As $y \rightarrow 0, z \rightarrow 0$ and $y \rightarrow \infty, z \rightarrow \infty$. Therefore
$$\left\langle x^{2}\right\rangle=\frac{2 \mu^{2}}{\sigma(2 \pi)^{\frac{1}{2}}} \int_{0}^{\infty} e^{\frac{y^{2}}{2 \sigma^{2}}} d y+\frac{2}{\sigma(2 \pi)^{\frac{1}{2}}} \int_{0}^{\infty} y^{2} e^{\frac{y^{2}}{2 \sigma^{2}}} d y$$
$$=\frac{2 \mu^{2}}{\sigma(2 \pi)^{\frac{1}{2}}} \int_{0}^{\infty} e^{-z} \cdot \frac{\sigma}{\sqrt{2}} z^{\frac{1}{2}} d z+\frac{2}{\sigma(2 \pi)^{\frac{1}{2}}} \int_{0}^{\infty} 2 \sigma^{2} z e^{-z} \cdot \frac{\sigma}{\sqrt{2}} z^{-\frac{1}{2}} d z$$
$$=\frac{2 \mu^{2}}{\sigma(2 \pi)^{\frac{1}{2}}} \cdot \frac{\sigma}{\sqrt{2}} \int_{0}^{\infty} e^{-z} z^{\frac{1}{2}} d z+\frac{2 \sqrt{2} \sigma^{3}}{\sigma(2 \pi)^{\frac{1}{2}}} \int_{0}^{\infty} e^{-z} z^{\frac{1}{2}} d z$$
$$=\frac{\mu^{2}}{\sqrt{\pi}} \int_{0}^{\infty} e^{-z} z^{\frac{1}{2}-1} d z+\frac{2 \sigma^{2}}{\sqrt{\pi}} \int_{0}^{\infty} e^{-z} z^{\frac{3}{2}-1} d z$$
$$=\frac{\mu^{2}}{\sqrt{\pi}} \Gamma\left(\frac{1}{2}\right)+\frac{2 \sigma^{2}}{\sqrt{\pi}} \Gamma\left(\frac{3}{2}\right)$$
$$=\frac{\mu^{2}}{\sqrt{\pi}} \cdot \sqrt{\pi}+\frac{2 \sigma^{2}}{\sqrt{\pi}} \cdot \frac{1}{2} \sqrt{\pi}$$
$$=\mu^{2}+\sigma^{2}$$
So the standard deviation 
$$\left(\left\langle x^{2}\right\rangle-\langle x\rangle^{2}\right)^{\frac{1}{2}}=\left(\mu^{2}+\sigma^{2}-\mu^{2}\right)^{\frac{1}{2}}$$
$$\left(\left\langle x^{2}\right\rangle-\langle x\rangle^{2}\right)^{\frac{1}{2}}=\sqrt{\sigma^{2}}$$
$$\left(\left\langle x^{2}\right\rangle-\langle x\rangle^{2}\right)^{\frac{1}{2}}=\sigma$$



\begin{mybox}{13.1.21}
For the gamma distribution
$$
f(x)=\left\{\begin{array}{ll}
\frac{1}{\beta^{\alpha} \Gamma(\alpha)} x^{\alpha-1} e^{-x / \beta}, & x>0 \\
0, & x \leq 0
\end{array}\right.
$$
\begin{enumerate}[$(a)$]
\item $\langle x\rangle,$ the mean value of $x,$ is equal to $\alpha \beta$
\item $\sigma^{2},$ its variance, defined as $\left\langle x^{2}\right\rangle-\langle x\rangle^{2},$ has the value $\alpha \beta^{2}$
\end{enumerate}
\end{mybox}

$\boxed{\textbf{Solution}}$ For $(a)$ the mean
$$
\langle x\rangle=\int_{0}^{\infty} x f(x) d x
$$
$$=\int_{0}^{\infty} x \cdot \frac{1}{\beta^{a} \Gamma(\alpha)} x^{a-1} e^{-\frac{x}{\beta}} d x$$
$$=\frac{1}{\Gamma(\alpha)} \int_{0}^{\infty}\left(\frac{x}{\beta}\right)^{a} e^{-\frac{x}{\beta}} d x$$
Put $\frac{x}{\beta}=z .$ Then $d x=\beta d z .$ As $x \rightarrow 0, z \rightarrow 0$ and $x \rightarrow \infty, z \rightarrow \infty$.
$$
\langle x\rangle=\frac{1}{\Gamma(\alpha)} \int_{0}^{\infty} z^{a} e^{-z} \beta d z
$$
$$=\frac{\beta}{\Gamma(\alpha)} \int_{0}^{\infty} z^{(a+1)-1} e^{-z} d z$$
$$=\frac{\beta}{\Gamma(\alpha)} \Gamma(\alpha+1)$$
$$=\frac{\beta}{\Gamma(\alpha)} \cdot \alpha \Gamma(\alpha)$$
$$=\alpha \beta$$

$\boxed{\textbf{Solution}}$ For $(b)$ 
$$
\left\langle x^{2}\right\rangle=\int_{0}^{\infty} x^{2} f(x) d x
$$
$$=\int_{0}^{\infty} x^{2} \cdot \frac{1}{\beta^{a} \Gamma(\alpha)} x^{\alpha-1} e^{-\frac{x}{\beta}} d x$$
$$=\frac{\beta}{\Gamma(\alpha)} \int_{0}^{\infty}\left(\frac{x}{\beta}\right)^{\alpha+1} e^{-\frac{x}{\beta}} d x$$
Put $\frac{x}{\beta}=z .$ Then $d x=\beta d z .$ As $x \rightarrow 0, z \rightarrow 0$ and $x \rightarrow \infty, z \rightarrow \infty$
$$
\left\langle x^{2}\right\rangle=\frac{\beta}{\Gamma(\alpha)} \int_{0}^{\infty} z^{a+1} e^{-z} \beta d z
$$
$=\frac{\beta^{2}}{\Gamma(\alpha)} \int_{0}^{\infty} z^{(\alpha+2)-1} e^{-z} d z$
$$=\frac{\beta^{2}}{\Gamma(\alpha)} \Gamma(\alpha+2)$$
$$=\frac{\beta^{2}}{\Gamma(\alpha)} \cdot(\alpha+1) \alpha \Gamma(\alpha)$$
$$=\alpha(\alpha+1) \beta^{2}$$
$$=\alpha^{2} \beta^{2}+\alpha \beta^{2}$$
Hence variance, $\sigma^{2}=\left\langle x^{2}\right\rangle-\langle x\rangle^{2}$
$$=\alpha^{2} \beta^{2}+\alpha \beta^{2}-\alpha^{2} \beta^{2}$$
$$=\alpha \beta^{2}$$



\end{flushleft}
\end{document}
