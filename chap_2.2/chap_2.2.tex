 
$\vspace{7cm}$

 
 
 % Suppresses headers and footers on the title page

	{\centering % Centre everything on the title page
	
	 % Use small caps for all text on the title page
	
	\vspace*{\baselineskip} % White space at the top of the page
	
	%------------------------------------------------
	%	Title
	%------------------------------------------------
	
	\rule{\textwidth}{1.6pt}\vspace*{-\baselineskip}\vspace*{2pt} % Thick horizontal rule
	\rule{\textwidth}{0.4pt} % Thin horizontal rule
	
	\vspace{0.75\baselineskip} % Whitespace above the title
	
	{\LARGE Chapter 2 \\ Determinants and \\ Matrices \\} % Title
	
	\vspace{0.75\baselineskip} % Whitespace below the title
	
	\rule{\textwidth}{0.4pt}\vspace*{-\baselineskip}\vspace{3.2pt} % Thin horizontal rule
	\rule{\textwidth}{1.6pt} % Thick horizontal rule
	
	\vspace{2\baselineskip} % Whitespace after the title block
	
	%------------------------------------------------
	%	Subtitle
	%------------------------------------------------
	

	
	\vspace*{3\baselineskip} % Whitespace under the subtitle
	

}


\newpage

\begin{center}
{\Large\text{Chapter 2.2: Matrices}}
\end{center}

\begin{mybox}{2.2.1}
Show that matrix multiplication is associative, $(\mathbf{A}\mathbf{B}) \mathbf{C}=\mathbf{A}(\mathbf{B}\mathbf{C})$
\end{mybox}



$\boxed{\textbf{Solution}}$ The product $\mathbf{BC}$ is defined because the column of $\mathbf{B}$ and rows of $\mathbf{C}$ are same.
Suppose $\mathbf{D}=\mathbf{BC}$
Then element of $\mathbf{D}$ is of the form
$$d_{ik}=\sum_{j} b_{ij} c_{j k}$$
Now the product $\mathbf{AD}$ is defined because the column of $\mathbf{A}$ and rows of $\mathbf{D}$ are same. Then element of $\mathbf{E}$ is of the form
$$e_{l k}=\sum_{k} a_{l i}\left(\sum_{j} b_{i j} c_{j k}\right)$$
Therefore, the matrix $\mathbf{E}=\mathbf{A}(\mathbf{B} \mathbf{C})$ have the elements $e_{lk}$. The product $\mathbf{AB}$ is defined because the column of $\mathbf{A}$ and rows of $\mathbf{B}$ are same. Let, $\mathbf{D}=\mathbf{AB}$. Then element of $\mathbf{D}$ is of the form
$$d_{l j}=\sum_{i} a_{li} b_{i j}$$
Now the product $\mathbf{D C}$ is defined because the column of $\mathbf{D}$ and rows of $\mathbf{C}$ are same. Let, $\mathbf{E}=\mathbf{D C}$. Then element of $\mathbf{D}$ is of the form
$$e_{lk}=\sum_{j}\left(\sum_{i} a_{li} b_{i j}\right) c_{j k}$$
Therefore, the matrix $\mathbf{E}=(\mathbf{A} \mathbf{B}) \mathbf{C}$ have the elements $e_{lk}$
Therefore,
$$\mathbf{A}(\mathbf{B} \mathbf{C})=(\mathbf{A} \mathbf{B}) \mathbf{C}$$
Hence, matrix multiplication is associative.

$$$$

\begin{mybox}{2.2.2}
Show that
$$
(\mathbf{A}+\mathbf{B})(\mathbf{A}-\mathbf{B})=\mathbf{A}^{2}-\mathbf{B}^{2}
$$
if and only if $\mathbf{A}$ and $\mathbf{B}$ commute
$$[\mathbf{A}, \mathbf{B}]=0$$
\end{mybox}




$\boxed{\textbf{Solution}}$ 
$$(\mathbf{A}+\mathbf{B})(\mathbf{A}-\mathbf{B})=(\mathbf{A}-\mathbf{B})(\mathbf{A}+\mathbf{B})=\mathbf{A}(\mathbf{A}+\mathbf{B})-\mathbf{B}(\mathbf{A}+\mathbf{B})$$
$$(\mathbf{A}+\mathbf{B})(\mathbf{A}-\mathbf{B})=\mathbf{A}^2+\mathbf{A}\mathbf{B}-\mathbf{B}\mathbf{A}-\mathbf{B}^2$$
$$(\mathbf{A}+\mathbf{B})(\mathbf{A}-\mathbf{B})=\mathbf{A}^2-\mathbf{B}^2+(\mathbf{A}\mathbf{B}-\mathbf{B}\mathbf{A})$$
Because $\mathbf{A}$ and $\mathbf{B}$ conmute, the term $(\mathbf{A}\mathbf{B}-\mathbf{B}\mathbf{A})$ equals to zero, hence is proved.
$$(\mathbf{A}+\mathbf{B})(\mathbf{A}-\mathbf{B})=\mathbf{A}^2 -\mathbf{B}^2$$


\newpage


\begin{mybox}{2.2.3}
\begin{enumerate}[$(a)$]
\item Complex numbers, $a+i b,$ with $a$ and $b$ real, may be represented by (or are isomorphic with) $2 \times 2$ matrices:
$$
a+i b \longleftrightarrow\begin{bmatrix}{a} & {b} \\ {-b} & {a}\end{bmatrix}
$$
Show that this matrix representation is valid for 
\begin{enumerate}[(i)]
\item addition
\item multiplication
\end{enumerate}
\item Find the matrix corresponding to $(a+i b)^{-1}$.
\end{enumerate}
\end{mybox}




$\boxed{\textbf{Solution}}$  $(a)$ Let us start with addition. For complex numbers, we have (straightforwardly)
$$
(a+i b)+(c+i d)=(a+c)+i(b+d)
$$
whereas, if we used matrices we would get
$$\begin{bmatrix}{a} & {b} \\ {-b} & {a}\end{bmatrix}+\begin{bmatrix}{c} & {d} \\ {-d} & {c}\end{bmatrix}=\begin{bmatrix}{(a+c)} & {(b+d)} \\ {-(b+d)} & {(a+c)}\end{bmatrix}$$
which shows that the sum of matrices yields the proper representation of the complex number $(a+c)+i(b+d)$. We now handle multiplication in the same manner. First, we have
$$(a+i b)(c+i d)=(a c-b d)+i(a d+b c)$$
while matrix multiplication gives
$$
\begin{bmatrix}{a} & {b} \\ {-b} & {a}\end{bmatrix}\begin{bmatrix}{c} & {d} \\ {-d} & {c}\end{bmatrix}=\begin{bmatrix}{(a c-b d)} & {(a d+b c)} \\ {-(a d+b c)} & {(a c-b d)}\end{bmatrix}
$$
which is again the correct result.

$$$$

$\boxed{\textbf{Solution}}$  For $(b)$ Find the matrix oorresponding to $(a+i b)^{-1}$ We can find the matrix in two ways. We first do standand complex arithmetic
$$
(a+i b)^{-1}=\frac{1}{a+i b}=\frac{a-i b}{(a+i b)(a-i b)}=\frac{1}{a^{2}+b^{2}}(a-i b)
$$
This corresponds to the $2 \times 2$ matrix
$$
(a+i b)^{-1} \longleftrightarrow \frac{1}{a^{2}+b^{2}}\begin{bmatrix}{a} & {-b} \\ {b} & {a}\end{bmatrix}
$$
Alternatively, we first convert to a matrix representation, and then find the inverse
matrix
$$
(a+i b)^{-1} \leftrightarrow\begin{bmatrix}{a} & {b} \\ {-b} & {a}\end{bmatrix}^{-1}=\frac{1}{a^{2}+b^{2}}\begin{bmatrix}{a} & {-b} \\ {b} & {a}\end{bmatrix}
$$
Either way, we obtain the same result.

\newpage

\begin{mybox}{2.2.4}
If $\mathbf{A}$ is an $n \times n$ matrix, show that
$$
\operatorname{det}(-\mathbf{A})=(-1)^{n} \text { det } \mathbf{A}.
$$
\end{mybox}




$\boxed{\textbf{Solution}}$ 
So from the above identity we can write
$$-A=(-I) A$$
$$\operatorname{det}(-A)=\operatorname{det}(-I A)$$
We know $\operatorname{det}(A B)=\operatorname{det}(A) \operatorname{det}(B)$
From this
$$\operatorname{det}(-I A)=\operatorname{det}(-I) \operatorname{det}(\mathrm{A})$$
$$\operatorname{det}(-I) \operatorname{det}(\mathrm{A})=(-1)^{n} \operatorname{det}(A)$$

$$$$

\begin{mybox}{2.2.5}
\begin{enumerate}[$(a)$]
\item The matrix equation $\mathbf{A}^{2}=0$ does not imply $\mathbf{A}=0 .$ Show that the most general
$2 \times 2$ matrix whose square is zero may be written as
$$
\begin{bmatrix}{a b} & {b^{2}} \\ {-a^{2}} & {-a b}\end{bmatrix}
$$
where $a$ and $b$ are real or complex numbers.
\item If $\mathbf{C}=\mathbf{A}+\mathbf{B}$, in general
$$
\text { det } \mathbf{C}\neq \operatorname{det} \mathbf{A}+\text { det } \mathbf{B} \text.
$$
Construct a specific numerical example to illustrate this inequality.
\end{enumerate}
\end{mybox}


$\boxed{\textbf{Solution}}$ For $(a)$ first we check the condition
$$
\left(\begin{array}{cc}
a b & b^{2} \\
-a^{2} & -a b
\end{array}\right) \times\left(\begin{array}{cc}
a b & b^{2} \\
-a^{2} & -a b
\end{array}\right)=\left(\begin{array}{cc}
a^{2} b^{2}-a^{2} b^{2} & a b^{3}-a b^{3} \\
-a^{3} b+a^{3} b & -a^{2} b^{2}+a^{2} b^{2}
\end{array}\right)=0
$$
Therefore, the $2\times 2$ matrix square is zero

$$$$

$\boxed{\textbf{Solution}}$ For $(b)$ we know $\mathrm{C}=\mathrm{A}+\mathrm{B},$ let us consider following matrices to show that
$$
\operatorname{det} C \neq \operatorname{det} A+\operatorname{det} B
$$
Now, let
$$
A=\left(\begin{array}{ll}
1 & 0 \\
0 & 1
\end{array}\right), B=\left(\begin{array}{ll}
1 & 0 \\
0 & 1
\end{array}\right)
$$
then 
$$
C=\left(\begin{array}{ll}
2 & 0 \\
0 & 2
\end{array}\right)
$$
$$
\begin{array}{l}
\operatorname{det} A=1-0=1 \\
\operatorname{det} B=1-0=1 \\
\operatorname{det} C=4-0=4
\end{array}
$$
From this
$$
\begin{array}{l}
\operatorname{det} C \neq \operatorname{det} A+\operatorname{det} B \\
4 \neq 2
\end{array}
$$
Therefore, the following matrix satisfies the condition

$$$$

\begin{mybox}{2.2.6}
Given
$$
\mathbf{K}=\begin{bmatrix}{0} & {0} & {i} \\ {-i} & {0} & {0} \\ {0} & {-1} & {0}\end{bmatrix}
$$
show that
$$
\mathbf{K}^{n}=\mathbf{KKK} \ \cdots \ (n \text { factors})=1
$$
(with the proper choice of $n, n \neq 0$ ).
\end{mybox}



$\boxed{\textbf{Solution}}$ We calculate for different $n$
$$\mathbf{K}^2 = \begin{bmatrix}{0} & {-i} & {0} \\ {0} & {0} & {1} \\ {i} & {0} & {0}\end{bmatrix} \quad \mathbf{K}^3=\begin{bmatrix}{-1} & {0} & {0} \\ {0} & {-1} & {0} \\ {0} & {0} & {-1}\end{bmatrix} \quad \mathbf{K}^4 = \begin{bmatrix}{0} & {0} & {-i} \\ {i} & {0} & {0} \\ {0} & {1} & {0}\end{bmatrix}$$
$$\mathbf{K}^5 = \begin{bmatrix}{0} & {i} & {0} \\ {0} & {0} & {-1} \\ {-i} & {0} & {0}\end{bmatrix} \quad \mathbf{K}^6 = \begin{bmatrix}{1} & {0} & {0} \\ {0} & {1} & {0} \\ {0} & {0} & {1}\end{bmatrix}$$
With this, the answer is $n=6$


$$$$

\begin{mybox}{2.2.7}
Verify the Jacobi identity,
$$[\mathbf{A},[\mathbf{B}, \mathbf{C}]]=[\mathbf{B},[\mathbf{A}, \mathbf{C}]]-[\mathbf{C},[\mathbf{A}, \mathbf{B}]]$$
\end{mybox}

$\boxed{\textbf{Solution}}$ 

$$
[\mathbf{A}[\mathbf{B}, \mathbf{C}]]=[\mathbf{A}, \mathbf{B} \mathbf{C}-\mathbf{C} \mathbf{B}]=\mathbf{A}(\mathbf{B} \mathbf{C}-\mathbf{C} \mathbf{B})-(\mathbf{B} \mathbf{C}-\mathbf{C} \mathbf{B}) \mathbf{A} $$
$$[\mathbf{A}[\mathbf{B}, \mathbf{C}]]=\mathbf{A}(\mathbf{B} \mathbf{C})-\mathbf{A}(\mathbf{C} \mathbf{B})-(\mathbf{B} \mathbf{C}) \mathbf{A}+(\mathbf{C} \mathbf{B}) \mathbf{A} $$
$$[\mathbf{B}[\mathbf{C}, \mathbf{A}]]=[\mathbf{B}, \mathbf{C} \mathbf{A}-\mathbf{A} \mathbf{C}]=\mathbf{B}(\mathbf{C A}-\mathbf{A} \mathbf{C})-(\mathbf{C A}-\mathbf{A} \mathbf{C}) \mathbf{B} $$
$$[\mathbf{B}[\mathbf{C}, \mathbf{A}]]=\mathbf{B}(\mathbf{C} \mathbf{A})-\mathbf{B}(\mathbf{A} \mathbf{C})-(\mathbf{C} \mathbf{A}) \mathbf{B}+(\mathbf{A} \mathbf{C}) \mathbf{B} $$
$$[\mathbf{C}[\mathbf{A}, \mathbf{B}]]=[\mathbf{C}, \mathbf{A} \mathbf{B}-\mathbf{B} \mathbf{A}]=\mathbf{C}(\mathbf{A} \mathbf{B}-\mathbf{B} \mathbf{A})-(\mathbf{A} \mathbf{B}-\mathbf{B} \mathbf{A}) \mathbf{C} $$
$$[\mathbf{C}[\mathbf{A}, \mathbf{B}]]=\mathbf{C}(\mathbf{A} \mathbf{B})-\mathbf{C}(\mathbf{B} \mathbf{A})-(\mathbf{A} \mathbf{B}) \mathbf{C}+(\mathbf{B} \mathbf{A}) \mathbf{C} $$
$\mathbf{A}, \mathbf{B}, \mathbf{C}$ are obey associative law $ \mathbf{C}(\mathbf{A} \mathbf{B})=(\mathbf{C} \mathbf{A}) \mathbf{B},$ $\mathbf{C}(\mathbf{B} \mathbf{A})=(\mathbf{C} \mathbf{B}) \mathbf{A},$ $(\mathbf{A} \mathbf{B}) \mathbf{C}=\mathbf{A}(\mathbf{B} \mathbf{C})$ and $(\mathbf{B} \mathbf{A}) \mathbf{C}=\mathbf{B}(\mathbf{A} \mathbf{C})$


$$[\mathbf{C}[\mathbf{A}, \mathbf{B}]]=(\mathbf{C} \mathbf{A}) \mathbf{B}-(\mathbf{C} \mathbf{B}) \mathbf{A}-\mathbf{A}(\mathbf{B} \mathbf{C})+\mathbf{B}(\mathbf{A} \mathbf{C})$$
$$[\mathbf{C}[\mathbf{A}, \mathbf{B}]]=[\mathbf{A},[\mathbf{B}, \mathbf{C}]]+[\mathbf{B}[\mathbf{C}, \mathbf{A}]]+[\mathbf{C}[\mathbf{A}, \mathbf{B}]] $$
$$=(\mathbf{A}(\mathbf{B} \mathbf{C})-\mathbf{A}(\mathbf{C} \mathbf{B})-(\mathbf{B} \mathbf{C}) \mathbf{A}+(\mathbf{C} \mathbf{B}) \mathbf{A})+(\mathbf{B}(\mathbf{C} \mathbf{A})-\mathbf{B}(\mathbf{A} \mathbf{C})-(\mathbf{C} \mathbf{A}) \mathbf{B} $$
$$+(\mathbf{A} \mathbf{C}) \mathbf{B})+(\mathbf{C} \mathbf{A}) \mathbf{B}-(\mathbf{C} \mathbf{B}) \mathbf{A}-\mathbf{A}(\mathbf{B} \mathbf{C})+\mathbf{B}(\mathbf{A} \mathbf{C})=0$$

\newpage

\begin{mybox}{2.2.8}
Show that the matrices
$$\mathbf{A}=\begin{bmatrix}{0} & {1} & {0} \\ {0} & {0} & {0} \\ {0} & {0} & {0}\end{bmatrix}, \quad \mathbf{B}=\begin{bmatrix}{0} & {0} & {0} \\ {0} & {0} & {1} \\ {0} & {0} & {0}\end{bmatrix}, \quad \mathbf{C}=\begin{bmatrix}{0} & {0} & {1} \\ {0} & {0} & {0} \\ {0} & {0} & {0}\end{bmatrix}$$
satisfy the commutation relations
$$
[\mathbf{A}, \mathbf{B}]=\mathbf{C}, \quad[\mathbf{A}, \mathbf{C}]=0, \quad \text { and } \quad[\mathbf{B}, \mathbf{C}]=0
$$
\end{mybox}


$\boxed{\textbf{Solution}}$ We simply multiply the matrices
$$\mathbf{C}=[\mathbf{A}, \mathbf{B}] = \mathbf{\mathbf{AB}}- \mathbf{BA}$$
$$\begin{bmatrix}{0} & {1} & {0} \\ {0} & {0} & {0} \\ {0} & {0} & {0}\end{bmatrix} \begin{bmatrix}{0} & {0} & {0} \\ {0} & {0} & {1} \\ {0} & {0} & {0}\end{bmatrix} - \begin{bmatrix}{0} & {0} & {0} \\ {0} & {0} & {1} \\ {0} & {0} & {0}\end{bmatrix}\begin{bmatrix}{0} & {1} & {0} \\ {0} & {0} & {0} \\ {0} & {0} & {0}\end{bmatrix} = C$$

$$[\mathbf{A}, \mathbf{C}] = A\mathbf{C}-\mathbf{C}\mathbf{A} = 0$$

$$\begin{bmatrix}{0} & {1} & {0} \\ {0} & {0} & {0} \\ {0} & {0} & {0}\end{bmatrix} \begin{bmatrix}{0} & {0} & {1} \\ {0} & {0} & {0} \\ {0} & {0} & {0}\end{bmatrix}- \begin{bmatrix}{0} & {0} & {1} \\ {0} & {0} & {0} \\ {0} & {0} & {0}\end{bmatrix}\begin{bmatrix}{0} & {1} & {0} \\ {0} & {0} & {0} \\ {0} & {0} & {0}\end{bmatrix} = 0$$

$$[\mathbf{B}, \mathbf{C}] = \mathbf{BC}-\mathbf{C}\mathbf{B} = 0$$

$$\begin{bmatrix}{0} & {0} & {0} \\ {0} & {0} & {1} \\ {0} & {0} & {0}\end{bmatrix}\begin{bmatrix}{0} & {0} & {1} \\ {0} & {0} & {0} \\ {0} & {0} & {0}\end{bmatrix}- \begin{bmatrix}{0} & {0} & {1} \\ {0} & {0} & {0} \\ {0} & {0} & {0}\end{bmatrix}\begin{bmatrix}{0} & {0} & {0} \\ {0} & {0} & {1} \\ {0} & {0} & {0}\end{bmatrix} = 0 $$


\newpage

\begin{mybox}{2.2.9}
Let
$$
i=\begin{bmatrix}{0} & {1} & {0} & {0} \\ {-1} & {0} & {0} & {0} \\ {0} & {0} & {0} & {1} \\ {0} & {0} & {-1} & {0}\end{bmatrix}, \quad j=\begin{bmatrix}{0} & {0} & {0} & {-1} \\ {0} & {0} & {-1} & {0} \\ {0} & {1} & {0} & {0} \\ {1} & {0} & {0} & {0}\end{bmatrix} \quad k=\begin{bmatrix}{0} & {0} & {-1} & {0} \\ {0} & {0} & {0} & {1} \\ {1} & {0} & {0} & {0} \\ {0} & {-1} & {0} & {0}\end{bmatrix}
$$

\begin{enumerate}[$(a)$]
\item $i^{2}=j^{2}=k^{2}=-I,$ where $\mathbf{I}$ is the unit matrix. 
\item $ij=-ji=k$, $jk=-k j=i$, $ki=-ik=j$
\end{enumerate}

These three matrices $(i, j,$ and $k)$ plus the unit matrix 1 form a basis for quaternions. An alternate basis is provided by the four 2 $\times 2$ matrices, $i \sigma_{1}, i \sigma_{2},-i \sigma_{3},$ and $1,$ where the $\sigma_{i}$ are the Pauli spin matrices of Example 2.2.1.
\end{mybox}


$\boxed{\textbf{Solution}}$ 
$$i^2=j^2 = k^2 =\begin{bmatrix}{-1} & {0} & {0} & {0} \\ {0} & {-1} & {0} & {0} \\ {0} & {0} & {-1} & {0} \\ {0} & {0} & {0} & {-1}\end{bmatrix}$$
$$ij = -ij = k =\begin{bmatrix}{0} & {0} & {-1} & {0} \\ {0} & {0} & {0} & {1} \\ {1} & {0} & {0} & {0} \\ {0} & {-1} & {0} & {0}\end{bmatrix} $$
$$jk=-kj=i=\begin{bmatrix}{0} & {1} & {0} & {0} \\ {-1} & {0} & {0} & {0} \\ {0} & {0} & {0} & {1} \\ {0} & {0} & {-1} & {0}\end{bmatrix}$$
$$ki=-ik=j = \begin{bmatrix}{0} & {0} & {0} & {-1} \\ {0} & {0} & {-1} & {0} \\ {0} & {1} & {0} & {0} \\ {1} & {0} & {0} & {0}\end{bmatrix}$$

$$$$

\begin{mybox}{2.2.10}
A matrix with elements $a_{i j}=0$ for $j<i$ may be called upper right triangular. The
elements in the lower left (below and to the left of the main diagonal) vanish. Show that
the product of two uper right triangular matrices is an upper right triangular matrix. 
\end{mybox}



$\boxed{\textbf{Solution}}$ We build 2 matrix with terms $a,b,c,x,y,z$ that can take any number
$$\begin{bmatrix}{a} & {b} & {c} \\ {0} & {d} & {e} \\ {0} & {0} & {0}\end{bmatrix} \times\begin{bmatrix}{x} & {y} & {z} \\ {0} & {u} & {w} \\ {0} & {0} & {0}\end{bmatrix}=\begin{bmatrix}{a\times x} & {b \times u+a\times y} & {b \times w+a\times z} \\ {0} & {d \times u} & {d \times w} \\ {0} & {0} & {0}\end{bmatrix}$$
Hence is demostrated that the product of two upper right triangular matrices is an upper right triagular matrix.


\newpage

\begin{mybox}{2.2.11}
The three Pauli spin matrices are
$$
\sigma_{1}=\begin{bmatrix}{0} & {1} \\ {1} & {0}\end{bmatrix}, \quad \sigma_{2}=\begin{bmatrix}{0} & {-i} \\ {i} & {0}\end{bmatrix}, \quad \text { and } \quad \sigma_{3}=\begin{bmatrix}{1} & {0} \\ {0} & {-1}\end{bmatrix}
$$
Show that

\begin{enumerate}[$(a)$]
\item $\left(\sigma_{i}\right)^{2}=\hat{1}_{2}$
\item $\sigma_{i} \sigma_{j}=i \sigma_{k},(i, j, k)=(1,2,3)$ or a cyclic permutation thereof,
\item $\sigma_{i} \sigma_{j}+\sigma_{j} \sigma_{i}=2 \delta_{i j} \hat{1}_{2} ; \quad \hat{1}_{2}$ is the $2 \times 2$ unit matrix.
\end{enumerate}
\end{mybox}

$\boxed{\textbf{Solution}}$ For $i=1, j=2, k=3$
$$\sigma_{i} \sigma_{j} = \sigma_{1} \sigma_{2} = \begin{bmatrix}{0} & {1} \\ {1} & {0}\end{bmatrix}\begin{bmatrix}{0} & {-1} \\ {1} & {0}\end{bmatrix}=\begin{bmatrix}{i} & {0} \\ {0} & {-i}\end{bmatrix}=i\begin{bmatrix}{1} & {0} \\ {0} & {-1}\end{bmatrix}= i\sigma_3 = i\sigma_k$$
For $i=2, j=3, k=1$
$$\sigma_{i} \sigma_{j}=\sigma_{2} \sigma_{3}=\begin{bmatrix}{0} & {i} \\ {i} & {0}\end{bmatrix}\begin{bmatrix}{1} & {0} \\ {0} & {-1}\end{bmatrix}=\begin{bmatrix}{0} & {i} \\ {i} & {0}\end{bmatrix}=i\begin{bmatrix}{0} & {1} \\ {1} & {0}\end{bmatrix} = i\sigma_3 = i\sigma_k$$
For $i=2, j=3, k=1$
$$\sigma_{i}\sigma_j = \sigma_3\sigma_1=\begin{bmatrix}{1} & {0} \\ {0} & {-1}\end{bmatrix}\begin{bmatrix}{0} & {1} \\ {1} & {0}\end{bmatrix}=\begin{bmatrix}{0} & {1} \\ {-1} & {0}\end{bmatrix}=i\begin{bmatrix}{0} & {-i} \\ {i} & {0}\end{bmatrix}=i\sigma_2 = i\sigma_k$$
So, we conclude that $\sigma_1\sigma_j = i\sigma_k$




$\boxed{\textbf{Solution}}$  $(c)$ For this proof we need only to work out the commutation relation and use the proofs done in part $(a)$ and $(b)$
$$\sigma_{2} \sigma_{1}=\begin{bmatrix}{0} & {-i} \\ {i} & {0}\end{bmatrix}\begin{bmatrix}{0} & {1} \\ {1} & {0}\end{bmatrix}=\begin{bmatrix}{-1} & {0} \\ {0} & {i}\end{bmatrix}=-i\begin{bmatrix}{1} & {0} \\ {0} & {1}\end{bmatrix} = -i\sigma_3$$
$$\sigma_{1} \sigma_{3}=\begin{bmatrix}{0} & {1} \\ {1} & {0}\end{bmatrix}\begin{bmatrix}{1} & {0} \\ {0} & {-1}\end{bmatrix}=\begin{bmatrix}{0} & {-1} \\ {1} & {0}\end{bmatrix}=-i\begin{bmatrix}{0} & {i} \\ {-i} & {0}\end{bmatrix}=-i \sigma_{2}$$
$$\sigma_{3} \cdot \sigma_{2}=\begin{bmatrix}{1} & {0} \\ {0} & {-1}\end{bmatrix}\begin{bmatrix}{0} & {-i} \\ {i} & {0}\end{bmatrix}=\begin{bmatrix}{0} & {-i} \\ {-i} & {0}\end{bmatrix}=-i\begin{bmatrix}{0} & {i} \\ {1} & {0}\end{bmatrix} = -i\sigma_1$$

$$\left.\begin{array}{l}{\sigma_{i} \sigma_{j}+\sigma_{j} \sigma_{i}=\sigma_{i} \sigma_{j}-\sigma_{i} \sigma_{j}=0} \vspace{3mm} \\ {\sigma_{i} \sigma_{j}+\sigma_{j} \sigma_{i}=2 \sigma_{i}^{2}}\end{array}\right\} \begin{array}{l}{\text { for } j \neq i} \\ {\text { for } j=i}\end{array}$$
since $\sigma_{i}^{2}=1$ and using the kronecker delta we have
$$\sigma_{i} \sigma_{j}+\sigma_{j} \sigma_{i}=2 \delta_{i j} 1$$


\newpage

\begin{mybox}{2.2.12}
One description of spin$-$1 particles uses the matrices
$$
\mathbf{M}_{x}=\frac{1}{\sqrt{2}}\begin{bmatrix}{0} & {1} & {0} \\ {1} & {0} & {1} \\ {0} & {1} & {0}\end{bmatrix}, \quad \mathbf{M}_{y}=\frac{1}{\sqrt{2}}\begin{bmatrix}{0} & {-i} & {0} \\ {i} & {0} & {-i} \\ {0} & {i} & {0}\end{bmatrix}
$$
and
$$\mathbf{M}_{z}=\begin{bmatrix}{1} & {0} & {0} \\ {0} & {0} & {0} \\ {0} & {0} & {-1}\end{bmatrix}$$

\begin{enumerate}[$(a)$]
\item $\left[\mathbf{M}_{x}, \mathbf{M}_{y}\right]=i \mathbf{M}_{z},$ and so on (cyclic permutation of indices). Using the Levi-Civita
symbol, we may write
$$
\left[\mathbf{M}_{i}, \mathbf{M}_{j}\right]=i \sum_{k} \varepsilon_{i j k} \mathbf{M}_{k}
$$
\item $\mathbf{M}^{2} \equiv \mathbf{M}_{x}^{2}+\mathbf{M}_{y}^{2}+\mathbf{M}_{z}^{2}=2 \mathbf{I}_{3},$ where $\mathbf{I}_{3}$ is the $3 \times 3$ unit matrix.
\item $\left[\mathbf{M}^{2}, \mathbf{M}_{i}\right]=0$
$\left[\mathbf{M}_{z}, \mathbf{L}^{+}\right]=\mathbf{L}^{+}$
$\left[\mathbf{L}^{+}, \mathbf{L}^{-}\right]=2 \mathbf{M}_{z}$
where $\mathbf{L}^{+} \equiv \mathbf{M}_{x}+i \mathbf{M}_{y}$ and $\mathbf{L}^{-} \equiv \mathbf{M}_{x}-i \mathbf{M}_{y}$
\end{enumerate}
\end{mybox}


$\boxed{\textbf{Solution}}$ For $(a)$
$$\mathbf{M}_{x} \mathbf{M}_{y}=\frac{1}{2}\left[\begin{array}{lll}{0} & {1} & {0} \\ {1} & {0} & {1} \\ {0} & {1} & {0}\end{array}\right]\left[\begin{array}{lll}{0} & {-1} & {0} \\ {i} & {0} & {-i} \\ {0} & {i} & {0}\end{array}\right]= \frac{1}{2}\left[\begin{array}{ccc}{i} & {0} & {-i} \\ {0} & {0} & {0} \\ {i} & {0} & {-i}\end{array}\right]$$

$$\mathbf{M}_{y} \mathbf{M}_{x}=\frac{1}{2}\left[\begin{array}{ccc}{0} & {-i} & {0} \\ {i} & {0} & {-i} \\ {0} & {i} & {0}\end{array}\right]\left[\begin{array}{ccc}{0} & {1} & {0} \\ {1} & {0} & {1} \\ {0} & {1} & {0}\end{array}\right]=\frac{1}{2}\left[\begin{array}{ccc}{-i} & {0} & {-i} \\ {0} & {0} & {0} \\ {i} & {0} & {i}\end{array}\right]$$


$$\mathbf{M}_{x} \mathbf{M}_{y}-\mathbf{M}_y \mathbf{M}_{x}=\frac{1}{2}\left[\begin{array}{ccc}{i} & {0} & {i} \\ {0} & {0} & {0} \\ {-i} & {0} & {-i}\end{array}\right]+\frac{1}{2}\left[\begin{array}{ccc}{i} & {0} & {i} \\ {0} & {0} & {0} \\ {-i} & {0} & {-i}\end{array}\right]$$
$$\mathbf{M}_{x} \mathbf{M}_{y}-\mathbf{M}_y \mathbf{M}_{x}=\left[\begin{array}{rrr}{1} & {0} & {0} \\ {0} & {0} & {0} \\ {0} & {0} & {-1}\end{array}\right]$$

$$\mathbf{M}_{x} \mathbf{M}_{y}-\mathbf{M}_y \mathbf{M}_{x}=\mathbf{M}_{z}$$


$\boxed{\textbf{Solution}}$  For $(b)$
$$\mathbf{M}_{x}^{2}=\frac{1}{2}\left[\begin{array}{ccc}{0} & {1} & {0} \\ {1} & {0} & {1} \\ {0} & {1} & {0}\end{array}\right]\left[\begin{array}{ccc}{0} & {1} & {0} \\ {1} & {0} & {1} \\ {0} & {1} & {0}\end{array}\right]=\frac{1}{2}\left[\begin{array}{ccc}{1} & {0} & {1} \\ {0} & {2} & {0} \\ {1} & {0} & {1}\end{array}\right]$$
$$\mathbf{M}_z^2 = \left[\begin{array}{ccc}{1} & {0} & {0} \\ {0} & {0} & {0} \\ {0} & {0} & {1}\end{array}\right] \quad \mathbf{M}_y^2 = \dfrac{1}{2}\left[\begin{array}{ccc}{1} & {0} & {-1} \\ {0} & {2} & {0} \\ {-1} & {0} & {1}\end{array}\right]$$
Now
$$\mathbf{M}_x^2 + \mathbf{M}_y^2 + \mathbf{M}_z^2 = 2\mathbf{I}$$
$\boxed{\textbf{Solution}}$  $(c)$ we substitute
$$
\mathbf{M}^{2}=\left(\begin{array}{lll}
2 & 0 & 0 \\
0 & 2 & 0 \\
0 & 0 & 2
\end{array}\right), \quad \mathbf{M}_{x}=\frac{1}{\sqrt{2}}\left(\begin{array}{lll}
0 & 1 & 0 \\
1 & 0 & 1 \\
0 & 1 & 0
\end{array}\right)
$$
and
$$
\left[\mathbf{M}^{2}, \mathbf{M}_{x}\right]=\mathbf{M}^{2} \mathbf{M}_{x}-\mathbf{M}_{x} \mathbf{M}^{2}
$$
$$
\mathbf{M}^{2} \mathbf{M}_{x}=\left(\begin{array}{lll}
2 & 0 & 0 \\
0 & 2 & 0 \\
0 & 0 & 2
\end{array}\right) \times \frac{1}{\sqrt{2}}\left(\begin{array}{lll}
0 & 1 & 0 \\
1 & 0 & 1 \\
0 & 1 & 0
\end{array}\right)=\frac{1}{\sqrt{2}}\left(\begin{array}{lll}
0 & 2 & 0 \\
2 & 0 & 2 \\
0 & 2 & 0
\end{array}\right)
$$
$$
\mathbf{M}_{x} \mathbf{M}^{2}=\frac{1}{\sqrt{2}}\left(\begin{array}{lll}
0 & 1 & 0 \\
1 & 0 & 1 \\
0 & 1 & 0
\end{array}\right) \times\left(\begin{array}{lll}
2 & 0 & 0 \\
0 & 2 & 0 \\
0 & 0 & 2
\end{array}\right)=\frac{1}{\sqrt{2}}\left(\begin{array}{lll}
0 & 2 & 0 \\
2 & 0 & 2 \\
0 & 2 & 0
\end{array}\right)
$$
$$
\left[\mathbf{M}^{2}, \mathbf{M}_{x}\right]=\mathbf{M}^{2} \mathbf{M}_{x}-\mathbf{M}_{x} \mathbf{M}^{2}=0
$$
Therefore, $\left[\mathbf{M}^{2}, \mathbf{M}_{i}\right]=0$
Now, we substitute 
$$
\mathbf{M}_{z}=\left(\begin{array}{ccc}
1 & 0 & 0 \\
0 & 0 & 0 \\
0 & 0 & -1
\end{array}\right)
$$
and
$$
\mathbf{L}^{+}=\mathbf{M}_{x}+i \mathbf{M}_{y}=\frac{1}{\sqrt{2}}\left(\begin{array}{lll}
0 & 2 & 0 \\
0 & 0 & 2 \\
0 & 0 & 0
\end{array}\right)
$$
$$
\left[\mathbf{M}_{z}, \mathbf{L}^{+}\right]=\mathbf{M}_{z} \mathbf{L}^{+}-\mathbf{L}^{+} \mathbf{M}_{z}
$$
$$
\mathbf{M}_{z} \mathbf{L}^{+}=\left(\begin{array}{ccc}
1 & 0 & 0 \\
0 & 0 & 0 \\
0 & 0 & -1
\end{array}\right) \times \frac{1}{\sqrt{2}}\left(\begin{array}{ccc}
0 & 2 & 0 \\
0 & 0 & 2 \\
0 & 0 & 0
\end{array}\right)=\frac{1}{\sqrt{2}}\left(\begin{array}{lll}
0 & 2 & 0 \\
0 & 0 & 0 \\
0 & 0 & 0
\end{array}\right)
$$
$$
\mathbf{L}^{+} \mathbf{M}_{z}=\frac{1}{\sqrt{2}}\left(\begin{array}{ccc}
0 & 2 & 0 \\
0 & 0 & 2 \\
0 & 0 & 0
\end{array}\right) \times\left(\begin{array}{ccc}
1 & 0 & 0 \\
0 & 0 & 0 \\
0 & 0 & -1
\end{array}\right)=0
$$
$$
\left[\mathbf{M}_{z}, \mathbf{L}^{+}\right]=\mathbf{M}_{z} \mathbf{L}^{+}-\mathbf{L}^{+} \mathbf{M}_{z}=\frac{1}{\sqrt{2}}\left(\begin{array}{ccc}
0 & 2 & 0 \\
0 & 0 & 0 \\
0 & 0 & 0
\end{array}\right)=\mathbf{L}^{+}
$$
$$
\left[\mathbf{M}_{z}, \mathbf{L}^{+}\right]=\mathbf{M}_{z} \mathbf{L}^{+}-\mathbf{L}^{+} \mathbf{M}_{z}=\mathbf{L}^{+}
$$
Now, substitute 
$$
\mathbf{L}^{+}=\mathbf{M}_{x}+i \mathbf{M}_{y}=\frac{1}{\sqrt{2}}\left(\begin{array}{lll}
0 & 2 & 0 \\
0 & 0 & 2 \\
0 & 0 & 0
\end{array}\right)
$$
and
$$
\mathbf{L}^{-}=\mathbf{M}_{x}-i \mathbf{M}_{y}=\frac{1}{\sqrt{2}}\left(\begin{array}{ccc}
0 & 0 & 0 \\
2 & 0 & 0 \\
0 & 2 & 0
\end{array}\right)
$$
$$
\left[\mathbf{L}^{+}, \mathbf{L}^{-}\right]=\mathbf{L}^{+} \mathbf{L}^{-}-\mathbf{L}^{-} \mathbf{L}^{+}
$$
$$
\mathbf{L}^{+} \mathbf{L}^{-}=\frac{1}{\sqrt{2}}\left(\begin{array}{lll}
0 & 2 & 0 \\
0 & 0 & 2 \\
0 & 0 & 0
\end{array}\right) \times \frac{1}{\sqrt{2}}\left(\begin{array}{lll}
0 & 0 & 0 \\
2 & 0 & 0 \\
0 & 2 & 0
\end{array}\right)=\left(\begin{array}{lll}
2 & 0 & 0 \\
0 & 2 & 0 \\
0 & 0 & 0
\end{array}\right)
$$

$$
\mathbf{L}^{-} \mathbf{L}^{+}=\frac{1}{\sqrt{2}}\left(\begin{array}{lll}
0 & 0 & 0 \\
2 & 0 & 0 \\
0 & 2 & 0
\end{array}\right) \times \frac{1}{\sqrt{2}}\left(\begin{array}{lll}
0 & 2 & 0 \\
0 & 0 & 2 \\
0 & 0 & 0
\end{array}\right)=\left(\begin{array}{lll}
0 & 0 & 0 \\
0 & 2 & 0 \\
0 & 0 & 2
\end{array}\right)
$$

$$
\left[\mathbf{L}^{+}, \mathbf{L}^{-}\right]=\mathbf{L}^{+} \mathbf{L}^{-}-\mathbf{L}^{-} \mathbf{L}^{+}=\left(\begin{array}{ccc}
2 & 0 & 0 \\
0 & 0 & 0 \\
0 & 0 & -2
\end{array}\right)=2 \mathbf{M}_{z}
$$
Therefore, $\left[\mathbf{L}^{+}, \mathbf{L}^{-}\right]=\mathbf{L}^{+} \mathbf{L}^{-}-\mathbf{L}^{-} \mathbf{L}^{+}=2 \mathbf{M}_{z}$








\newpage



\begin{mybox}{2.2.13}
Repeat Exercise $2.2 .12,$ using the matrices for a spin of $3 / 2,$

$$
\mathbf{M}_{x}=\frac{1}{2}\left(\begin{array}{cccc}
0 & \sqrt{3} & 0 & 0 \\
\sqrt{3} & 0 & 2 & 0 \\
0 & 2 & 0 & \sqrt{3} \\
0 & 0 & \sqrt{3} & 0
\end{array}\right), \ \ \mathbf{M}_{y}=\frac{i}{2}\left(\begin{array}{cccc}
0 & -\sqrt{3} & 0 & 0 \\
\sqrt{3} & 0 & -2 & 0 \\
0 & 2 & 0 & -\sqrt{3} \\
0 & 0 & \sqrt{3} & 0
\end{array}\right)
$$
and
$$
\mathbf{M}_{z}=\frac{1}{2}\left(\begin{array}{rrrr}
3 & 0 & 0 & 0 \\
0 & 1 & 0 & 0 \\
0 & 0 & -1 & 0 \\
0 & 0 & 0 & -3
\end{array}\right)
$$
\end{mybox}


$\boxed{\textbf{Solution}}$ For $(a)$ we consider that $\left[\mathbf{M}_{x}, \mathbf{M}_{y}\right]$
$$
=\frac{i}{4}\left[\begin{array}{cccc}
0 & \sqrt{3} & 0 & 0 \\
\sqrt{3} & 0 & 2 & 0 \\
0 & 2 & 0 & \sqrt{3} \\
0 & 0 & \sqrt{3} & 0
\end{array}\right]\left[\begin{array}{cccc}
0 & -\sqrt{3} & 0 & 0 \\
\sqrt{3} & 0 & -2 & 0 \\
0 & 2 & 0 & -\sqrt{3} \\
0 & 0 & \sqrt{3} & 0
\end{array}\right]-$$
$$\frac{i}{4}\left[\begin{array}{cccc}
0 & -\sqrt{3} & 0 & 0 \\
\sqrt{3} & 0 & -2 & 0 \\
0 & 2 & 0 & -\sqrt{3} \\
0 & 0 & \sqrt{3} & 0
\end{array}\right]\left[\begin{array}{cccc}
0 & \sqrt{3} & 0 & 0 \\
\sqrt{3} & 0 & 2 & 0 \\
0 & 2 & 0 & \sqrt{3} \\
0 & 0 & \sqrt{3} & 0
\end{array}\right]
$$
$$
=\frac{i}{2}\left[\begin{array}{cccc}
3 & 0 & 0 & 0 \\
0 & 1 & 0 & 0 \\
0 & 0 & -1 & 0 \\
0 & 0 & 0 & -3
\end{array}\right]
$$
$$
=i \mathbf{M}_{z}
$$
Similarly we can show that $\left[\mathbf{M}_{y}, \mathbf{M}_{z}\right]=i \mathbf{M}_{x}$ and $\left[\mathbf{M}_{z}, \mathbf{M}_{x}\right]=i \mathbf{M}_{y}$
Thus, $\left[\mathbf{M}_{i}, \mathbf{M}_{j}\right]=i \sum_{k} \varepsilon_{i j k} \mathbf{M}_{k}$ where $i, j, k$ can take values 1,2,3 or $x, y, z$.



$\boxed{\textbf{Solution}}$ For $(b)$ we consider that 
$$
\mathbf{M}^{2} \equiv \mathbf{M}_{x}^{2}+\mathbf{M}_{y}^{2}+\mathbf{M}_{z}^{2}
$$
$$
=\frac{1}{4}\left[\begin{array}{cccc}
0 & \sqrt{3} & 0 & 0 \\
\sqrt{3} & 0 & 2 & 0 \\
0 & 2 & 0 & \sqrt{3} \\
0 & 0 & \sqrt{3} & 0
\end{array}\right]^{2}-\frac{1}{4}\left[\begin{array}{cccc}
0 & -\sqrt{3} & 0 & 0 \\
\sqrt{3} & 0 & -2 & 0 \\
0 & 2 & 0 & -\sqrt{3} \\
0 & 0 & \sqrt{3} & 0
\end{array}\right]^{2}+\frac{1}{4}\left[\begin{array}{cccc}
3 & 0 & 0 & 0 \\
0 & 1 & 0 & 0 \\
0 & 0 & -1 & 0 \\
0 & 0 & 0 & -3
\end{array}\right]^{2}
$$
$$
\begin{array}{l}
=2\left[\begin{array}{lll}
1 & 0 & 0 \\
0 & 1 & 0 \\
0 & 0 & 1
\end{array}\right] \\
=21_{3}
\end{array}
$$

$\boxed{\textbf{Solution}}$ For $(c)$ we obtain the result from the previous part 
$$
\begin{aligned}
\left[\mathbf{M}^{2}, \mathbf{M}_{i}\right] &=\left[21_{3}, \mathbf{M}_{i}\right] \\
&=21_{3} \mathbf{M}_{i}-2 \mathbf{M}_{i} 1_{3} \\
&=2 \mathbf{M}_{i}-2 \mathbf{M}_{i} \\
&=0
\end{aligned}
$$
$$
\begin{aligned}
\left[\mathbf{M}_{z}, \mathbf{L}^{+}\right] &=\left[\mathbf{M}_{z}, \mathbf{M}_{x}+i \mathbf{M}_{y}\right] \\
&=\left[\mathbf{M}_{z}, \mathbf{M}_{x}\right]-\left[i \mathbf{M}_{y}, \mathbf{M}_{z}\right] \\
&=\left[\mathbf{M}_{z}, \mathbf{M}_{x}\right]-i\left[\mathbf{M}_{y}, \mathbf{M}_{z}\right] \\
&=i \mathbf{M}_{y}-i\left(i \mathbf{M}_{x}\right)
\end{aligned}
$$
$$
\begin{array}{l}
=i \mathbf{M}_{y}+\mathbf{M}_{x} \\
=\mathbf{M}_{x}+i \mathbf{M}_{y} \\
=\mathbf{L}^{+}
\end{array}
$$
And finally
$$
\begin{aligned}
\left[\mathbf{L}^{+}, \mathbf{L}^{-}\right] &=\left[\mathbf{M}_{x}+i \mathbf{M}_{y}, \mathbf{M}_{x}-i \mathbf{M}_{y}\right] \\
&=\left[\mathbf{M}_{x}, \mathbf{M}_{x}\right]-\left[\mathbf{M}_{x}, i \mathbf{M}_{y}\right]+\left[i \mathbf{M}_{y}, \mathbf{M}_{x}\right]-\left[i \mathbf{M}_{y}, i \mathbf{M}_{y}\right], \\
&=0-i\left[\mathbf{M}_{x}, \mathbf{M}_{y}\right]-\left[\mathbf{M}_{x}, i \mathbf{M}_{y}\right]-0 \\
&=-2 i\left(i \mathbf{M}_{z}\right) \\
&=2 \mathbf{M}_{z}
\end{aligned}
$$


\begin{mybox}{2.2.14}
If $\mathbf{A}$ is a diagonal matrix, with all diagonal elements different, and $\mathbf{A}$ and $\mathbf{B}$ commute,
show that $\mathbf{B}$ is diagonal.
\end{mybox}


$\boxed{\textbf{Solution}}$ Given matrix $\mathbf{A}$ is diagonal matrix
$$\mathbf{A}=\text{diag}(a_1, a_2, a_3, ... a_n)$$
and $B = (b_{ij})$. Here $\mathbf{A}$ and $\mathbf{B}$ matrix conmute $\mathbf{A}\mathbf{B}=\mathbf{BA}$, so
$$(a_i-a_j)b_{kl} = 0 \quad \text{for} \ k\neq l$$
$$b_{kl} = 0\quad \text{for} \ k\neq l$$
Hence from the above statement we can say that is also a diagonal matrix

$$$$

\begin{mybox}{2.2.15}
If $\mathbf{A}$ and $\mathbf{B}$ are diagonal, show that $\mathbf{A}$ and $\mathbf{B}$ commute.
\end{mybox}

$\boxed{\textbf{Solution}}$  consider two $n\times n$ matrices $\mathbf{A}$ and $\mathbf{B}$, which are diagonal.
$$\mathbf{A}=\left[\begin{array}{ll}{a_{1}} & {0} \\ {0} & {a_{2}}\end{array}\right], \quad \mathbf{B}=\left[\begin{array}{ll}{b_{1}} & {0} \\ {0} & {b_{2}}\end{array}\right]$$
$$\mathbf{A} \mathbf{B}=\left[\begin{array}{ll}{a_{1}} & {0} \\ {0} & {a_{2}}\end{array}\right]\left[\begin{array}{ll}{b_{1}} & {0} \\ {0} & {b_{2}}\end{array}\right]=\left[\begin{array}{cc}{a_{1} b_{1}} & {0} \\ {0} & {a_{2} b_{2}}\end{array}\right]$$
$$\mathbf{B} \mathbf{A}=\left[\begin{array}{ll}{b_{1}} & {0} \\ {0} & {b_{2}}\end{array}\right]\left[\begin{array}{ll}{a_{1}} & {0} \\ {0} & {a_{2}}\end{array}\right]=\left[\begin{array}{lll}{b_{1} a_{1}} & {0} \\ {0} & {b_{2} a_{2}}\end{array}\right]$$
Commutative properts of addition:
$$\mathbf{A}+\mathbf{B}=\left[\begin{array}{ll}{a_{1}} & {0} \\ {0} & {a_{2}}\end{array}\right]+\left[\begin{array}{ll}{b_{1}} & {0} \\ {0} & {b_{2}}\end{array}\right]=\left[\begin{array}{ccc}{a_{1}+b_{1}} & {0} \\ {0} & {a_{2}+b_{2}}\end{array}\right]$$
$$\mathbf{B}+\mathbf{A}=\left[\begin{array}{ll}{b_{1}} & {0} \\ {0} & {b_{2}}\end{array}\right]+\left[\begin{array}{ll}{a_{1}} & {0} \\ {0} & {a_{2}}\end{array}\right]=\left[\begin{array}{cc}{b_{1}+a_{1}} & {0} \\ {0} & {b_{2}+a_{2}}\end{array}\right]$$
Hence, diagonal matrices, when added commute.

\newpage


\begin{mybox}{2.2.16}
Show that $\text{Tr}(\mathbf{ABC})=$ $\text{Tr}(\mathbf{CBA})$ if any two of the three matrices commute.
\end{mybox}



$\boxed{\textbf{Solution}}$  The trace of a matrix is the sum of its diagonal elements. Therefore, the trace of the product of three matrices $\mathbf{A}, \mathbf{B},$ and $\mathbf{C}$ is
given by
$$
\operatorname{Tr}(\mathbf{A} \mathbf{B} \mathbf{C})=\sum_{i j k} \mathbf{A}_{i j} \mathbf{B}_{j k} \mathbf{C}_{k i}
$$
By using the fact that $i, j,$ and $k$ are dummy summation indices
with the same range, this sum can be written in the equivalent
forms
$$
\sum_{i j k} \mathbf{A}_{i j} \mathbf{B}_{j k} \mathbf{C}_{k i}=\sum_{i j k} \mathbf{C}_{k i} \mathbf{A}_{i j} \mathbf{B}_{j k}=\sum_{i j k} \mathbf{B}_{j k} \mathbf{C}_{k i} \mathbf{A}_{i j}
$$
But the second and third of these are
$$
\sum_{i j k} \mathbf{C}_{k i} \mathbf{A}_{i j} \mathbf{B}_{j k}=\operatorname{Tr}(\mathbf{C} \mathbf{A} \mathbf{B})
$$
and
$$
\sum_{i j k} \mathbf{B}_{j k} \mathbf{C}_{k i} \mathbf{A}_{i j}=\operatorname{Tr}(\mathbf{B} \mathbf{C} \mathbf{A})
$$
respectively. Thus, we obtain the relation
$$
\operatorname{Tr}(\mathbf{A} \mathbf{B} \mathbf{C})=\operatorname{Tr}(\mathbf{C} \mathbf{A} \mathbf{B})=\operatorname{Tr}(\mathbf{B} \mathbf{C} \mathbf{A})
$$

$$$$

\begin{mybox}{2.2.17}
Angular momentum matrices satisfy a commutation relation
$$[\mathbf{M}_j,\mathbf{M}_k]=i\mathbf{M}_l \quad j,k,l \ \text{cyclic}$$
\end{mybox}



$\boxed{\textbf{Solution}}$  Taking the trace of both sides of the given expression, we have
$$
\operatorname{Tr}\left(i \mathbf{M}_{k}\right)=\operatorname{Tr}\left(\mathbf{M}_{i} \mathbf{M}_{j}-\mathbf{M}_{j} \mathbf{M}_{i}\right)
$$
Hence
$$
i \operatorname{Tr}\left(\mathbf{M}_{k}\right)=\operatorname{Tr}\left(\mathbf{M}_{i} \mathbf{M}_{j}\right)-\operatorname{Tr}\left(\mathbf{M}_{j} \mathbf{M}_{i}\right)
$$
since $\operatorname{Tr}(\mathbf{A}\mathbf{B})=\operatorname{Tr}(\mathbf{B}\mathbf{A}),$ we see that $\operatorname{Tr}\left(\mathbf{M}_{k}\right)=0$ for any $k .$

$$$$

\begin{mybox}{2.2.18}
$\mathbf{A}$ and $\mathbf{B}$ anticommute: $\mathbf{A}\mathbf{B}=-\mathbf{B}\mathbf{A}$. Also, $\mathbf{A}^{2}=1, \mathbf{B}^{2}=1 .$ Show that $\text{Tr}(\mathbf{A})=$
$\text{Tr}(\mathbf{B})=0 .$
Note. The Pauli and Dirac matrices are specific examples.
\end{mybox}



$\boxed{\textbf{Solution}}$ Since $\mathbf{B}^{2}=I, \mathbf{B}$ is non-singular and its inverse exists. Therefore, $\mathbf{A}=-\mathbf{B}^{-1} \mathbf{AB}$. Taking the trace, we get
$$
\operatorname{Tr}(\mathbf{A})=-\operatorname{Tr}\left(\mathbf{B}^{-1} \mathbf{AB}\right)=-\operatorname{Tr}\left(\mathbf{ABB}^{-1}\right)=-\operatorname{Tr}(\mathbf{A})
$$
We see that $\operatorname{Tr}(\mathbf{A})=0 .$ Similarly, we find $\operatorname{Tr}(\mathbf{B})=0$


\newpage




\begin{mybox}{2.2.19}
\begin{enumerate}[$(a)$]
\item If two nonsingular matrices anticommute, show that the trace of each one is zero.
(Nonsingular means that the determinant of the matrix is nonzero.)
\item For the conditions of part $(a)$ to hold, $\mathbf{A}$ and $\mathbf{B}$ must be $n \times n$ matrices with $n$ even.
Show that if $n$ is odd, a contradiction results.
\end{enumerate}
\end{mybox}

$\boxed{\textbf{Solution}}$ For $(a)$ if the matrices are non-singular, then writing 
$$\mathbf{A}=-\mathbf{B} \mathbf{A} \mathbf{B}^{-1}$$ 
and taking the trace, we get 
$$\operatorname{Tr} \mathbf{A}=-\operatorname{Tr} \mathbf{A}$$ 
Hence $\operatorname{Tr} \mathbf{A}=0,$ and the procedure for $\mathbf{B}$ is analogous.


$\boxed{\textbf{Solution}}$ For $(b)$ now, we compute the determinant of both sides of $\mathbf{A} \mathbf{B}=-\mathbf{B} \mathbf{A}$ : this yields det $\mathbf{A}$ det $\mathbf{B}=(-1)^{N}$ det $\mathbf{B}$ det $\mathbf{A},$ where $N$ stands for size of matrices. Now since the $\mathbf{A}, \mathbf{B}$ are non-singular, both sides of the equality are non-zero and the equality is possible only for even $N .$

$$$$

\begin{mybox}{2.2.20}
If $\mathbf{A}^{-1}$ has elements
$$
\left(\mathbf{A}^{-1}\right)_{i j}=a_{i j}^{(-1)}=\frac{\mathbf{C}_{j i}}{|\mathbf{A}|}
$$
where $\mathbf{C}_{j i}$ is the $j i$ th cofactor of $|\mathbf{A}|,$ show that
$$
\mathbf{A}^{-1} \mathbf{A}=\mathbf{I}
$$
Hence $\mathbf{A}^{-1}$ is the inverse of $\mathbf{A}$ (if $|\mathbf{A}| \neq 0$ ).
\end{mybox}

$\boxed{\textbf{Solution}}$ We have to consider that
$$
\left(A^{-1} A\right)_{i j}=\sum_{k} a_{i k}^{-1} a_{k j}
$$

$$=\sum_{k} \frac{C_{k i}}{|A|} a_{k j}$$
$$=\frac{1}{|A|} \sum_{k} C_{k i} a_{k j}$$
$$=\frac{1}{|A|}|A| \delta_{i j}$$
$$=\delta_{i j}$$
Thus, $A^{-1} A=1$. Hence, by definition of inverse of a matrix $A^{-1}$ is the inverse of $A($ if $|A| \neq 0)$. 






\newpage



\begin{mybox}{2.2.21}
Find the matrices $\mathbf{M}_{L}$ such that the product $\mathbf{M}_{L} \mathbf{A}$ will be $\mathbf{A}$ but with:
\begin{enumerate}[$(a)$]
\item The $i$ th row multiplied by a constant $k\left(a_{i j} \rightarrow k a_{i j}, j=1,2,3, \ldots\right)$
\item The $i$ th row replaced by the original $i$ th row minus a multiple of the $m^{\text{th}}$ row
$\left(a_{i j} \rightarrow a_{i j}-K a_{m j}, i=1,2,3, \ldots\right)$
\item The $i$ th and $m$ th rows interchanged $\left(a_{i j} \rightarrow a_{m j}, a_{m j} \rightarrow a_{i j}, j=1,2,3, \ldots\right)$

\end{enumerate}
\end{mybox}

$\boxed{\textbf{Solution}}$  For $(a)$ Let 
$$M_{L}=\left[\begin{array}{ll}k & 0 \\ 0 & 1\end{array}\right]$$ 
and 
$$A=\left[\begin{array}{ll}a & b \\ c & d\end{array}\right]$$
Then 
$$M_{L} A=\left[\begin{array}{cc}k a & k b \\ c & d\end{array}\right]$$

$\boxed{\textbf{Solution}}$ For $(b)$ a unit matrix except that $M_{i m}=-K$.

$$M_{L}=\left[\begin{array}{cc}1 & -K \\ 0 & 1\end{array}\right]$$
and 
$$A=\left[\begin{array}{ll}a & b \\ c & d\end{array}\right]$$
Then
$$
M_{L} A=\left[\begin{array}{cc}
a-K c & b-K d \\
c & d
\end{array}\right]
$$

$\boxed{\textbf{Solution}}$  For $(c)$ A unit matrix except that $M_{i i}=M_{m m}=0$ and $M_{m i}-M_{i m}=1$.
$$M_{L}=\left[\begin{array}{ll}0 & 1 \\ 2 & 0\end{array}\right]$$ 
and 
$$A=\left[\begin{array}{ll}a & b \\ c & d\end{array}\right]$$
$$
M_{L} A=\left[\begin{array}{cc}
c & d \\
2 a & 2 b
\end{array}\right]
$$



\newpage


\begin{mybox}{2.2.22}
Find the matrices $\mathbf{M}_{R}$ such that the product AM $_{R}$ will be A but with:

\begin{enumerate}[$(a)$]
\item The $i$ th column multiplied by a constant $k\left(a_{j i} \rightarrow k a_{j i}, j=1,2,3, \ldots\right)$
\item The $i$ th column replaced by the original $i$ th column minus a multiple of the $m^{\text{th}}$ column $\left(a_{j i} \rightarrow a_{j i}-k a_{j m}, j=1,2,3, \ldots\right)$
\item The $i$ th and $m$ th columns interchanged $\left(a_{j i} \rightarrow a_{j m}, a_{j m} \rightarrow a_{j i}, j=1,2,3, \ldots\right)$
\end{enumerate}
\end{mybox}


$\boxed{\textbf{Solution}}$  For $(a)$, a unit matrix except that $M_{i i}=k$. Let 
$$M_{R}=\left[\begin{array}{ll}k & 0 \\ 0 & 1\end{array}\right]$$ 
and 
$$A=\left[\begin{array}{ll}a & b \\ c & d\end{array}\right]$$
Then 
$$A M_{R}=\left[\begin{array}{ll}k a & b \\ k c & d\end{array}\right]$$

$\boxed{\textbf{Solution}}$  For $(b)$, a unit matrix except that $M_{i m}=-K$.
Let 
$$M_{R}=\left[\begin{array}{cc}1 & -K \\ 0 & 1\end{array}\right]$$ 
and 
$$A=\left[\begin{array}{ll}a & b \\ c & d\end{array}\right]$$
Then 
$$A M_{R}=\left[\begin{array}{ll}a & b-K a \\ c & d-K c\end{array}\right]$$


$\boxed{\textbf{Solution}}$  For $(c)$, a unit matrix except that $M_{i i}=M_{m m}=0$ and $M_{m i}-M_{i m}=1$.
Example: Let 
$$M_{R}=\left[\begin{array}{ll}0 & 1 \\ 2 & 0\end{array}\right]$$ 
and 
$$A=\left[\begin{array}{ll}a & b \\ c & d\end{array}\right]$$
Then 
$$A M_{R}=\left[\begin{array}{ll}2 b & a \\ 2 d & c\end{array}\right]$$


\newpage



\begin{mybox}{2.2.23}
Find the inverse of
$$
\mathbf{A}=\begin{bmatrix}{3} & {2} & {1} \\ {2} & {2} & {1} \\ {1} & {1} & {4}\end{bmatrix}
$$
\end{mybox}


$\boxed{\textbf{Solution}}$ Calculating $\mathbf{A}^{-1}$


$$\mathbf{A}^{-1}=\frac{1}{7}\begin{bmatrix}{7} & {-7} & {0} \\ {-7} & {11} & {-1} \\ {0} & {-1} & {2}\end{bmatrix}$$

$$$$

\begin{mybox}{2.2.24}
Matrices are far too useful to remain the exclusive property of physicists. They may
appear wherever there are linear relations. For instance, in a study of population move-
ment the initial fraction of a fixed population in each of $n$ areas (or industries or
religions, etc.) is represented by an $n$ -component column vector $\mathbf{P} .$ 

The movement of people from one area to another in a given time is described by an $n \times n$ (stochastic) matrix $\mathbf{T}$. Here $\mathbf{T}_{i j}$ is the fraction of the population in the $j$ th area that moves to the $i^{\text{th}}$ area. (Those not moving are covered by $i=j.$) With $\mathbf{P}$ describing the initial population distribution, the final population distribution is given by the matrix equation $\mathbf{TP}=\mathbf{Q}$. From its definition, $\sum_{i=1}^{n} \mathbf{P}_{i}=1$

\begin{enumerate}[$(a)$]
\item Show that conservation of people requires that
$$
\sum_{i=1}^{n} \mathbf{T}_{i j}=1, \quad j=1,2, \ldots, n
$$
\item Prove that
$$
\sum_{i=1}^{n} \mathbf{Q}_{i}=1
$$
continues the conservation of people.
\end{enumerate}
\end{mybox}




$\boxed{\textbf{Solution}}$  For $(a)$ The equation of part states that $\mathbf{T}$ moves people from area $j$ but
does not change their total number. 



$\boxed{\textbf{Solution}}$ For $(b)$ Write the component equation 
$$\sum_{j} \mathbf{T}_{i j} \mathbf{P}_{j}=\mathbf{Q}_{i}$$ 
and sum over $i.$ This summation replaces $\mathbf{T}_{i j}$ by unity, leaving that the sum pver $\mathbf{P}_{j}$ equals the sum over $\mathbf{Q}_{i},$ hence conserving people.


\newpage

\begin{mybox}{2.2.25}
Given a $6 \times 6$ matrix A with elements $a_{i j}=0.5^{|i-j|}, i, j=0,1,2, \ldots, 5, \text { find } \mathbf{A}^{-1}$

$$\mathbf{A}=\begin{bmatrix}
1 & 0.5 & 0.5^2 & 0.5^3 & 0.5^4 & 0.5^5 \\ 
0.5 & 1 & 0.5 & 0.5^2 & 0.5^3 & 0.5^4 \\ 
0.5^2 & 0.5 & 1 & 0.5 & 0.5^2 & 0.5^3\\ 
0.5^3 & 0.5^2 & 0.5 & 1 & 0.5 & 0.5^2\\ 
0.5^4 & 0.5^3 & 0.5^2 & 0.5 & 1 & 0.5\\ 
0.5^5 & 0.5^4 & 0.5^3 & 0.5^2 & 0.5 & 1
\end{bmatrix}$$
\end{mybox}

$\boxed{\textbf{Solution}}$

$$\mathbf{A}^{-1}=\frac{1}{3}\begin{bmatrix}{4} & {-2} & {0} & {0} & {0} & {0} \\ {-2} & {5} & {-2} & {0} & {0} & {0} \\ {0} & {-2} & {5} & {-2} & {0} & {0} \\ {0} & {0} & {-2} & {5} & {-2} & {0} \\ {0} & {0} & {0} & {-2} & {5} & {-2} \\ {0} & {0} & {0} & {0} & {-2} & {4}\end{bmatrix}$$


$$$$

\begin{mybox}{2.2.26}
Show that the product of two orthogonal matrices is orthogonal.
\end{mybox}



$\boxed{\textbf{Solution}}$ Let $Q$ and $P$ be orthogonal matrices. Therefore $\mathbf{Q}^{T} \mathbf{Q}=\mathbf{I}$ and $\mathbf{P}^{T} P=I $
We have that
$$
(\mathbf{P Q})^{T}(\mathbf{P Q})=\mathbf{Q}^{T} \mathbf{P}^{T} \mathbf{P I Q}=\mathbf{Q}^{T} \mathbf{Q}=\mathbf{I}
$$
Therefore, a product of two orthogonal matrix is an orthogonal matrix.

$$$$




\begin{mybox}{2.2.27}
If A is orthogonal, show that its determinant $=\pm 1 .$
\end{mybox}



$\boxed{\textbf{Solution}}$ We know that
$$\ \text{det} \mathbf{A}^T = \ \text{det} \mathbf{A}$$
$$\mathbf{A}^T\mathbf{A}=I$$
$$\ \text{det} \mathbf{A}^T = \text{det}I = 1$$
$$\ \text{det} \mathbf{A}^T\mathbf{A} \ \text{det} \mathbf{A}^T \ \text{det} \mathbf{A}$$
$$(\text{det} \mathbf{A})^2 = \ \text{det} \mathbf{A} \ \text{det} \mathbf{A}^T = \ \text{det} \mathbf{A}^T \ \text{det} \mathbf{A} = \ \text{det} \mathbf{A}^TA = 1$$
So we must have
$$\text{det}\mathbf{A} = \pm 1$$





\begin{mybox}{2.2.28}
Show that the trace of the product of a symmetric and an antisymmetric matrix is zero.
\end{mybox}


$\boxed{\textbf{Solution}}$  
If $\tilde{\mathbf{A}}=-\mathbf{A}, \tilde{\mathbf{S}}=\mathbf{S},$ then

$$\text{Tr}(\widetilde{\mathbf{SA}})=\text{Tr}(\mathbf{SA})=\text{Tr}(\tilde{\mathbf{A}} \tilde{\mathbf{S}})=-\text{Tr}(\mathbf{AS})$$



\newpage


\begin{mybox}{2.2.29}
$\mathbf{A}$ is $2 \times 2$ and orthogonal. Find the most general form of
$$
\mathbf{A}=\begin{bmatrix}{a} & {b} \\ {c} & {d}\end{bmatrix}
$$
\end{mybox}

$\boxed{\textbf{Solution}}$  From $\tilde{\mathbf{A}}=\mathbf{A}^{-1}$ and $\operatorname{det}(\mathbf{A})=1$ we have
$$
\mathbf{A}^{-1}=\begin{bmatrix}{a_{22}} & {-a_{12}} \\ {-a_{21}} & {a_{11}}\end{bmatrix}=\tilde{\mathbf{A}}=\begin{bmatrix}{a_{11}} & {a_{21}} \\ {a_{12}} & {a_{22}}\end{bmatrix}
$$
This gives det $(\mathbf{A})=a_{11}^{2}+a_{12}^{2}=1,$ hence 
$$a_{11}=\cos \theta=a_{22},\quad  a_{12}=\sin \theta=-a_{21},$$
the standard $2 \times 2$ rotation matrix.

$$$$

\begin{mybox}{2.2.30}
Show that
$$
\operatorname{det}\left(\mathbf{A}^{*}\right)=(\operatorname{det} \mathbf{A})^{*}=\operatorname{det}\left(\mathbf{A}^{\dagger}\right)
$$
\end{mybox}


$\boxed{\textbf{Solution}}$  We calculate the determinant of $\mathbf{A}^{*}$

$$\operatorname{det}\left(\mathbf{A}^{*}\right)=\sum_{i_{k}} \varepsilon_{i_{1} i_{2} \ldots i_{n}} a_{1 i_{1}}^{*} a_{2 i_{2}}^{*} \cdots a_{n i_{n}}^{*}=\left(\sum_{i_{k}} \varepsilon_{i_{1} i_{2} \ldots i_{n}} a_{1 i_{1}} a_{2 i_{2}} \cdots a_{n i_{n}}\right)$$
Because, for any $\mathbf{A},$ 
$$\operatorname{det}(\mathbf{A})=\operatorname{det}(\tilde{\mathbf{A}}), \operatorname{det}\left(\mathbf{A}^{*}\right)=\operatorname{det}\left(\mathbf{A}^{\dagger}\right)$$

$$$$

\begin{mybox}{2.2.31}
Three angular momentum matrices satisfy the basic commutation relation
$$[\mathbf{J}_x, \mathbf{J}_y] = i\mathbf{J}_z$$
(and cyclic permutation of indices). If two of the matrices have real elements, show that
the elements of the third must be pure imaginary.
\end{mybox}

$\boxed{\textbf{Solution}}$ We know that basic commutation relation is $\left[\mathbf{J}_{i}, \mathbf{J}_{j}\right]=i \mathbf{J}_{k}$, where $i$ $j$ and $k$ are indices in cyclic permutation.
Here it is clear that $\mathbf{J}_{x}, \mathbf{J}_{y}$ are real, so also must be their commutator. So according to
commutator rule it requires that $\mathbf{J}_{z}$ be pure imaginary.

$$$$

\begin{mybox}{2.2.32}
Show that $(\mathbf{A}\mathbf{B})^{\dagger}=\mathbf{B}^{\dagger} \mathbf{A}^{\dagger}$
\end{mybox}



$\boxed{\textbf{Solution}}$ We know that the transpose of a product is $(A B)^{T}=B^{T} A^{T}$. That is product is transposed by
taking, in reverse order, the transpose of its factor.
From this
$$
(A B)^{\dagger}=\left(A^{*} B^{*}\right)^{T}=\left(B^{*}\right)^{T}\left(A^{*}\right)^{T}=B^{\dagger} A^{\dagger}
$$
The complex conjugate of product is equal to conjugate of its individual factor.

\newpage


\begin{mybox}{2.2.33}
A matrix $\mathbf{C}=\mathbf{S}^{\dagger} \mathbf{S}$. Show that the trace is positive definite unless $\mathbf{S}$ is the null matrix, in which case $\text{Tr}(\mathbf{C})=0 .$
\end{mybox}



$\boxed{\textbf{Solution}}$ As
$$\mathbf{C}_{j k}=\sum_{n} \boldsymbol{S}_{n j}^{*} \boldsymbol{S}_{n k}$$
$$\text{Tr}(\mathbf{C})=\sum_{n j}\left|\boldsymbol{S}_{n j}\right|^{2}$$

$$$$

\begin{mybox}{2.2.34}
If $\mathbf{A}$ and $\mathbf{B}$ are Hermitian matrices, show that $(\mathbf{A}\mathbf{B}+\mathbf{BA})$ and $i(\mathbf{A}\mathbf{B}-\mathbf{B}\mathbf{A})$ are also Hermitian.
\end{mybox}

$\boxed{\textbf{Solution}}$  If $\mathbf{A}^{\dagger}=\mathbf{A}, \mathbf{B}^{\dagger}=B,$ then
$$(\mathbf{A}\mathbf{B}+\mathbf{BA})^{\dagger}=\mathbf{B}^{\dagger} \mathbf{A}^{\dagger}+\mathbf{A}^{\dagger} \mathbf{B}^{\dagger}=\mathbf{AB}+\mathbf{BA}$$
$$-i\left(\mathbf{B}^{\dagger} \mathbf{A}^{\dagger}-\mathbf{A}^{\dagger} \mathbf{B}^{\dagger}\right)=i(\mathbf{A}\mathbf{B}-\mathbf{B}\mathbf{A})$$


$$$$

\begin{mybox}{2.2.35}
The matrix $\mathbf{C}$ is not Hermitian. Show that then $\mathbf{C}+\mathbf{C}^{\dagger}$ and $i\left(\mathbf{C}-\mathbf{C}^{\dagger}\right)$ are Hermitian.
This means that a non-Hermitian matrix may be resolved into two Hermitian parts,
$$
\mathbf{C}=\frac{1}{2}\left(\mathbf{C}+\mathbf{C}^{\dagger}\right)+\frac{1}{2 i} i\left(\mathbf{C}-\mathbf{C}^{\dagger}\right)
$$
This decomposition of a matrix into two Hermitian matrix parts parallels the decompo-
sition of a complex number $z$ into $x+i y,$ where $x=\left(z+z^{*}\right) / 2$ and $y=\left(z-z^{*}\right) / 2 i$
\end{mybox}


$\boxed{\textbf{Solution}}$ If $\mathbf{C}^{\dagger} \neq \mathbf{C},$ then 
$$\left(i \mathbf{C}_{-}\right)^{\dagger} \equiv\left(\mathbf{C}^{\dagger}-\mathbf{C}\right)^{\dagger}=\mathbf{C}-\mathbf{C}^{\dagger}=-i \mathbf{C}_{-}^{\dagger},$$
$$\left(\mathbf{C}_{-}\right)^{\dagger}=\mathbf{C}_{-}$$
$$\mathbf{C}_{+}^{\dagger}=\mathbf{C}_{+}=\mathbf{C}+\mathbf{C}^{\dagger}$$

$$$$

\begin{mybox}{2.2.36}
A and $\mathbf{B}$ are two noncommuting Hermitian matrices:
$$
\mathbf{AB}-\mathbf{B}\mathbf{A}=i \mathbf{C}
$$
Prove that $\mathbf{C}$ is Hermitian.
\end{mybox}


$\boxed{\textbf{Solution}}$ Let's consider

$$-i \mathbf{C}^{\dagger}=(\mathbf{A}\mathbf{B}-\mathbf{B}\mathbf{A})^{\dagger}$$
$$-i \mathbf{C}^{\dagger}=\mathbf{B}^{\dagger} \mathbf{A}^{\dagger}-\mathbf{A}^{\dagger} \mathbf{B}^{\dagger}$$
$$-i \mathbf{C}^{\dagger}=B \mathbf{A}-\mathbf{A}\mathbf{B}$$
$$-i \mathbf{C}^{\dagger}=-i \mathbf{C}$$

\newpage

\begin{mybox}{2.2.37}
Two matrices $\mathbf{A}$ and $\mathbf{B}$ are each Hermitian. Find a necessary and sufficient condition for
their product $\mathbf{AB}$ to be Hermitian.
\end{mybox}


$\boxed{\textbf{Solution}}$ 
$$(\mathbf{A}\mathbf{B})^{\dagger}=\mathbf{B}^{\dagger} \mathbf{A}^{\dagger}=\mathbf{B} \mathbf{A}=\mathbf{AB}$$
With this, we can say that $[\mathbf{A}, \mathbf{B}] = 0$

$$$$

\begin{mybox}{2.2.38}
Show that the reciprocal (that is, inverse) of a unitary matrix is unitary.
\end{mybox}


$\boxed{\textbf{Solution}}$ A matrix is said to be unitary matrix if its adjoint is equal to its inverse. Let $\mathbf{U}$ be unitary matrix. Then $\mathbf{U}^{\dagger}=\mathbf{U}^{-1}$.
$$\left( \mathbf{U}^{\dagger}\right)^{\dagger}=\mathbf{U}$$
$$\left( \mathbf{U}^{\dagger}\right)^{\dagger}=\left( \mathbf{U}^{-1}\right)^{\dagger}$$

$$$$

\begin{mybox}{2.2.39}
Prove that the direct product of two unitary matrices is unitary.
\end{mybox}

$\boxed{\textbf{Solution}}$ 

$$\left( \mathbf{U}_{1}  \mathbf{U}_{2}\right)^{\dagger}= \mathbf{U}_{2}^{\dagger}  \mathbf{U}_{1}^{\dagger}$$
$$\left( \mathbf{U}_{1}  \mathbf{U}_{2}\right)^{\dagger}= \mathbf{U}_{2}^{-1}  \mathbf{U}_{1}^{-1}$$
$$\left( \mathbf{U}_{1}  \mathbf{U}_{2}\right)^{\dagger}=\left( \mathbf{U}_{1}  \mathbf{U}_{2}\right)^{-1}$$

$$$$

\begin{mybox}{2.2.40}
If $\sigma$ is the vector with the $\sigma_{i}$ as components given in Eq. $(2.61),$ and $p$ is an ordinary vector, show that
$$(\sigma \cdot p)^{2}=p^{2} \hat{1}_{2}$$
where $\hat{1}_{2}$ is a $2 \times 2$ unit matrix.
\end{mybox}


$\boxed{\textbf{Solution}}$ 
$$(\mathbf{p} \cdot \boldsymbol{\sigma})^{2}=\left(p_{x} \sigma_{1}+p_{y} \sigma_{2}+p_{z} \sigma_{3}\right)^{2}$$
$$p_{x}^{2} \sigma_{1}^{2}+p_{y}^{2} \sigma_{2}^{2}+p_{z}^{2} \sigma_{3}^{2}+p_{x} p_{y}\left(\sigma_{1} \sigma_{2}+\sigma_{2} \sigma_{1}\right)+p_{x} p_{z}\left(\sigma_{1} \sigma_{3}+\sigma_{3} \sigma_{1}\right)$$
$$+p_{y} p_{z}\left(\sigma_{1} \sigma_{2}+\sigma_{2} \sigma_{1}\right)=p_{x}^{2}+p_{y}^{2}+p_{z}^{2}=\mathbf{p}^{2}$$

\newpage

\begin{mybox}{2.2.41}
Use the equations for the properties of direct products, Eqs. (2.57) and $(2.58),$ to show
that the four matrices $\gamma^{\mu}, \mu=0,1,2,3,$ satisfy the conditions listed in Eqs. (2.74) and $(2.75) .$
\end{mybox}


$\boxed{\textbf{Solution}}$  Writing $\gamma^{0}=\sigma_{3} \otimes \mathbf{1}$ and $\gamma^{i}=\gamma \otimes \sigma_{i}(i=1,2,3),$ where
$$
\gamma=\begin{bmatrix}{0} & {1} \\ {-1} & {0}\end{bmatrix}
$$
and noting fron Eq. (2.57) that if $\mathbf{C}=\mathbf{A} \otimes B$ and $\mathbf{C}'=\mathbf{A}' \otimes B'$ then
$\mathbf{C} \mathbf{C}'=\mathbf{A} \mathbf{A}' \otimes B B'$
$$\left(\gamma^{0}\right)^{2}=\sigma_{3}^{2} \otimes \mathbf{1}_{2}^{2}=\hat{1}_{2} \otimes \mathbf{1}_{2}=\hat{1}_{4}, \quad\left(\gamma^{i}\right)^{2}=\gamma^{2} \otimes \sigma_{i}^{2}=\left(-\hat{1}_{2}\right) \otimes \mathbf{1}_{2}=-\hat{1}_{4}$$
$$\gamma^{0} \gamma^{i}=\sigma_{3} \gamma \otimes \mathbf{1}_{2} \sigma_{i}=\sigma_{1} \otimes \sigma_{i}, \quad \gamma^{i} \gamma^{0}=\gamma \sigma_{3} \otimes \sigma_{i} \mathbf{1}_{2}=\left(-\sigma_{1}\right) \otimes \sigma_{i}$$
$$\gamma^{i} \gamma^{j}=\gamma^{2} \otimes \sigma_{i} \sigma_{j} \quad \gamma^{j} \gamma^{i}=\gamma^{2} \otimes \sigma_{j} \sigma_{i}$$
It is obvious from the second line of the above equation set that $\gamma^{0} \gamma^{i}+$
$\gamma^{i} \gamma^{0}=0 ;$ from the third line of the equation set we find $\gamma^{i} \gamma^{j}+\gamma^{j} \gamma^{i}$ is zero if $j \neq i$ because then $\sigma_{j} \sigma_{i}=-\sigma_{i} \sigma_{j}$

\newpage

\begin{mybox}{2.2.42}
Show that $\gamma^{5}$, Eq. (2.76), anticommutes with all four $\gamma^{\mu}$.
\end{mybox}


$\boxed{\textbf{Solution}}$ 
$$
\gamma^{0}=\begin{bmatrix}
1 & 0 & 0 & 0 \\
0 & 1 & 0 & 0 \\
0 & 0 & -1 & 0 \\
0 & 0 & 0 & -1
\end{bmatrix}, \quad \gamma^{1}=\begin{bmatrix}
0 & 0 & 0 & 1 \\
0 & 0 & 1 & 0 \\
0 & -1 & 0 & 0 \\
-1 & 0 & 0 & 0
\end{bmatrix}, \gamma^{2}=\begin{bmatrix}
0 & 0 & 0 & -i \\
0 & 0 & i & 0 \\
0 & i & 0 & 0 \\
-i & 0 & 0 & 0
\end{bmatrix}
$$
$$
\gamma^{3}=\begin{bmatrix}
0 & 0 & 1 & 0 \\
0 & 0 & 0 & -1 \\
-1 & 0 & 0 & 0 \\
0 & 1 & 0 & 0
\end{bmatrix}\quad \text { and }\quad \gamma^{5}=\begin{bmatrix}
0 & 0 & 1 & 0 \\
0 & 0 & 0 & 1 \\
1 & 0 & 0 & 0 \\
0 & 1 & 0 & 0
\end{bmatrix}
$$
$$
\gamma^{0} \gamma^{5}=\begin{bmatrix}
1 & 0 & 0 & 0 \\
0 & 1 & 0 & 0 \\
0 & 0 & -1 & 0 \\
0 & 0 & 0 & -1
\end{bmatrix}\begin{bmatrix}
0 & 0 & 1 & 0 \\
0 & 0 & 0 & 1 \\
1 & 0 & 0 & 0 \\
0 & 1 & 0 & 0
\end{bmatrix}=\begin{bmatrix}
0 & 0 & 1 & 0 \\
0 & 0 & 0 & 1 \\
-1 & 0 & 0 & 0 \\
0 & -1 & 0 & 0
\end{bmatrix}
$$
$$
\gamma^{5} \gamma^{0}=\begin{bmatrix}
0 & 0 & 1 & 0 \\
0 & 0 & 0 & 1 \\
1 & 0 & 0 & 0 \\
0 & 1 & 0 & 0
\end{bmatrix}\begin{bmatrix}
1 & 0 & 0 & 0 \\
0 & 1 & 0 & 0 \\
0 & 0 & -1 & 0 \\
0 & 0 & 0 & -1
\end{bmatrix}=\begin{bmatrix}
0 & 0 & -1 & 0 \\
0 & 0 & 0 & -1 \\
1 & 0 & 0 & 0 \\
0 & 1 & 0 & 0
\end{bmatrix}
$$
$$
\gamma^{1} \gamma^{5}=\begin{bmatrix}
0 & 0 & 0 & 1 \\
0 & 0 & 1 & 0 \\
0 & -1 & 0 & 0 \\
-1 & 0 & 0 & 0
\end{bmatrix}\begin{bmatrix}
0 & 0 & 1 & 0 \\
0 & 0 & 0 & 1 \\
1 & 0 & 0 & 0 \\
0 & 1 & 0 & 0
\end{bmatrix}=\begin{bmatrix}
0 & 1 & 0 & 0 \\
1 & 0 & 0 & 0 \\
0 & 0 & 0 & -1 \\
0 & 0 & -1 & 0
\end{bmatrix}
$$
$$
\gamma^{5} \gamma^{1}=\begin{bmatrix}
0 & 0 & 1 & 0 \\
0 & 0 & 0 & 1 \\
1 & 0 & 0 & 0 \\
0 & 1 & 0 & 0
\end{bmatrix}\begin{bmatrix}
0 & 0 & 0 & 1 \\
0 & 0 & 1 & 0 \\
0 & -1 & 0 & 0 \\
-1 & 0 & 0 & 0
\end{bmatrix}=\begin{bmatrix}
0 & -1 & 0 & 0 \\
-1 & 0 & 0 & 0 \\
0 & 0 & 0 & 1 \\
0 & 0 & 1 & 0
\end{bmatrix}
$$
$$
\gamma^{2} \gamma^{5}=\begin{bmatrix}
0 & 0 & 0 & -i \\
0 & 0 & i & 0 \\
0 & i & 0 & 0 \\
-i & 0 & 0 & 0
\end{bmatrix}\begin{bmatrix}
0 & 0 & 1 & 0 \\
0 & 0 & 0 & 1 \\
1 & 0 & 0 & 0 \\
0 & 1 & 0 & 0
\end{bmatrix}=\begin{bmatrix}
0 & -i & 0 & 0 \\
i & 0 & 0 & 0 \\
0 & 0 & 0 & i \\
0 & 0 & -i & 0
\end{bmatrix}
$$
$$
\gamma^{5} \gamma^{2}=\begin{bmatrix}
0 & 0 & 1 & 0 \\
0 & 0 & 0 & 1 \\
1 & 0 & 0 & 0 \\
0 & 1 & 0 & 0
\end{bmatrix}\begin{bmatrix}
0 & 0 & 0 & -i \\
0 & 0 & i & 0 \\
0 & i & 0 & 0 \\
-i & 0 & 0 & 0
\end{bmatrix}=\begin{bmatrix}
0 & i & 0 & 0 \\
-i & 0 & 0 & 0 \\
0 & 0 & 0 & -i \\
0 & 0 & i & 0
\end{bmatrix}
$$
$$
\gamma^{3} \gamma^{5}=\begin{bmatrix}
0 & 0 & 1 & 0 \\
0 & 0 & 0 & -1 \\
-1 & 0 & 0 & 0 \\
0 & 1 & 0 & 0
\end{bmatrix}\begin{bmatrix}
0 & 0 & 1 & 0 \\
0 & 0 & 0 & 1 \\
1 & 0 & 0 & 0 \\
0 & 1 & 0 & 0
\end{bmatrix}=\begin{bmatrix}
1 & 0 & 0 & 0 \\
0 & -1 & 0 & 0 \\
0 & 0 & -1 & 0 \\
0 & 0 & 0 & 1
\end{bmatrix}
$$
$$
\gamma^{5} \gamma^{3}=\begin{bmatrix}
0 & 0 & 1 & 0 \\
0 & 0 & 0 & 1 \\
1 & 0 & 0 & 0 \\
0 & 1 & 0 & 0
\end{bmatrix}\begin{bmatrix}
0 & 0 & 1 & 0 \\
0 & 0 & 0 & -1 \\
-1 & 0 & 0 & 0 \\
0 & 1 & 0 & 0
\end{bmatrix}=\begin{bmatrix}
-1 & 0 & 0 & 0 \\
0 & 1 & 0 & 0 \\
0 & 0 & 1 & 0 \\
0 & 0 & 0 & -1
\end{bmatrix}
$$
This shows that $\gamma^{5}$ anticommutes with all four $\gamma^{\mu}(\mu=0,1,2,3)$


\newpage

\begin{mybox}{2.2.43}
In this problem, the summations are over $\mu=0,1,2,3 .$ Define $g_{\mu v}=g^{\mu v}$ by the relations
$$
g_{00}=1 ; \quad g_{k k}=-1, \quad k=1,2,3 ; \quad g_{\mu v}=0, \quad \mu \neq v
$$
and define $\gamma_{\mu}$ as $\sum g_{v \mu} \gamma^{\mu} .$ Using these definitions, show that
\begin{enumerate}[$(a)$]
\item $\sum \gamma_{\mu} \gamma^{\alpha} \gamma^{\mu}=-2 \gamma^{\alpha}$
\item $\sum \gamma_{\mu} \gamma^{\alpha} \gamma^{\beta} \gamma^{\mu}=4 g^{\alpha \beta}$
\item $\sum \gamma_{\mu} \gamma^{\alpha} \gamma^{\beta} \gamma^{v} \gamma^{\mu}=-2 \gamma^{v} \gamma^{\beta} \gamma^{\alpha}$
\end{enumerate}
\end{mybox}

$\boxed{\textbf{Solution}}$ No solution yet.

\newpage

\begin{mybox}{2.2.44}
If $\mathbf{M}=\frac{1}{2}\left(1+\gamma^{5}\right),$ where $\gamma^{5}$ is given in Eq. $(2.76),$ show that
$$
\mathbf{M}^{2}=\mathbf{M}
$$
Note that this equation is still satisfied if $\gamma$ is replaced by any other Dirac matrix listed in Eq. (2.76) 
\end{mybox}


$\boxed{\textbf{Solution}}$ Consider $\mathbf{M}^{2}=\left[\frac{1}{2}\left(1+\gamma^{5}\right)\right]^{2}$
$$
\begin{array}{l}
=\dfrac{1}{4}\left(\hat{1}_{4}+2 \gamma^{5}+\left(\gamma^{5}\right)^{2}\right)\vspace{3mm} \\
=\dfrac{1}{4}\left(\hat{1}_{4}+2 \gamma^{5}+\hat{1}_{4}\right) \vspace{3mm}\\
=\dfrac{1}{4}\left(2\hat{1}_{4}+2 \gamma^{5}\right) \vspace{3mm}\\
=\dfrac{1}{2}\left(\hat{1}_{4}+\gamma^{5}\right) \vspace{3mm}\\
=\mathbf{M}
\end{array}
$$
Thus, $\mathbf{M}^2 = \mathbf{M}$
 
\newpage


\begin{mybox}{2.2.45}
Prove that the 16 Dirac matrices form a linearly independent set.
\end{mybox}


$\boxed{\textbf{Solution}}$ No solution yet.

\newpage

\begin{mybox}{2.2.46}
If we assume that a given $4 \times 4$ matrix $\mathbf{A}$ (with constant elements) can be written as a linear combination of the 16 Dirac matrices (which we denote here as $\Gamma_{i}$ )
$$
\mathbf{A}=\sum_{i=1}^{16} c_{i} \Gamma_{i}
$$
show that
$$
c_{i} \sim \operatorname{trace}\left(\mathbf{A} \Gamma_{i}\right)
$$
The matrix $\mathbf{C}=i \gamma^{2} \gamma^{0}$ is sometimes called the charge conjugation matrix. Show that $\mathbf{C} \boldsymbol{\gamma}^{\mu} \mathbf{C}^{-1}=-\left(\boldsymbol{\gamma}^{\mu}\right)^{T}$
\end{mybox}

$\boxed{\textbf{Solution}}$ No solution yet.

\newpage

\begin{mybox}{2.2.47}
The matrix $\mathbf{C}=i \gamma^{2} \gamma^{0}$ is sometimes called the charge conjugation matrix. Show that $\mathbf{C} \boldsymbol{\gamma}^{\mu} \mathbf{C}^{-1}=-\left(\boldsymbol{\gamma}^{\mu}\right)^{T}$
\end{mybox}
$\boxed{\textbf{Solution}}$ Here
$$
\gamma^{0}=\begin{bmatrix}
1 & 0 & 0 & 0 \\
0 & 1 & 0 & 0 \\
0 & 0 & -1 & 0 \\
0 & 0 & 0 & -1
\end{bmatrix}, \quad \gamma^{1}=\begin{bmatrix}
0 & 0 & 0 & 1 \\
0 & 0 & 1 & 0 \\
0 & -1 & 0 & 0 \\
-1 & 0 & 0 & 0
\end{bmatrix}, \gamma^{2}=\begin{bmatrix}
0 & 0 & 0 & -i \\
0 & 0 & i & 0 \\
0 & i & 0 & 0 \\
-i & 0 & 0 & 0
\end{bmatrix}
$$
$$
\gamma^{3}=\begin{bmatrix}
0 & 0 & 1 & 0 \\
0 & 0 & 0 & -1 \\
-1 & 0 & 0 & 0 \\
0 & 1 & 0 & 0
\end{bmatrix}
$$
Consider $\mathbf{C} \gamma^{0} \mathbf{C}^{-1}=i \gamma^{2} \gamma^{0} \gamma^{0}\left(i \gamma^{2} \gamma^{0}\right)^{-1}$
$$
\begin{array}{l}
=\gamma^{2} \gamma^{0} \gamma^{0}\left(\gamma^{0}\right)^{-1}\left(\gamma^{2}\right)^{-1} \\
=-\gamma^{0} \\
=-\left(\gamma^{0}\right)^{T} \\
\mathbf{C} \gamma^{1} \mathbf{C}^{-1}=i \gamma^{2} \gamma^{0} \gamma^{1}\left(i \gamma^{2} \gamma^{0}\right)^{-1} \\
=\gamma^{2} \gamma^{0} \gamma^{1}\left(\gamma^{0}\right)^{-1}\left(\gamma^{2}\right)^{-1} \\
=\gamma^{1} \\
=-\left(\gamma^{1}\right)^{T}
\end{array}
$$
$$
\begin{aligned}
\mathbf{C} \gamma^{2} \mathbf{C}^{-1} &=i \gamma^{2} \gamma^{0} \gamma^{2}\left(i \gamma^{2} \gamma^{0}\right)^{-1} \\
&=\gamma^{2} \gamma^{0} \gamma^{2}\left(\gamma^{0}\right)^{-1}\left(\gamma^{2}\right)^{-1} \\
&=-\gamma^{2} \\
&=-\left(\gamma^{2}\right)^{T} \\
\mathbf{C} \gamma^{3} \mathbf{C}^{-1} &=i \gamma^{2} \gamma^{0} \gamma^{3}\left(i \gamma^{2} \gamma^{0}\right)^{-1} \\
&=\gamma^{2} \gamma^{0} \gamma^{3}\left(\gamma^{0}\right)^{-1}\left(\gamma^{2}\right)^{-1} \\
&=\gamma^{3} \\
&=-\left(\gamma^{3}\right)^{T}
\end{aligned}
$$
Thus, $\mathbf{C} \gamma^{\mu} \mathbf{C}^{-1}=-\left(\gamma^{\mu}\right)^{T}(\mu=0,1,2,3)$

\newpage


\begin{mybox}{2.2.48}
\begin{enumerate}[$(a)$]
\item Show that, by substitution of the definitions of the $\gamma^{\mu}$ matrices from Eqs. (2.70) and $(2.72),$ that the Dirac equation, Eq. (2.73), takes the following form when written as $2 \times 2$ blocks (with $\psi_{L}$ and $\psi_{S}$ column vectors of dimension 2 ). Here $L$ and $S$ stand, respectively, for "large" and "small" because of their relative size in the nonrelativistic limit):
$$
\begin{bmatrix}
m c^{2}-E & c\left(\sigma_{1} p_{1}+\sigma_{2} p_{2}+\sigma_{3} p_{3}\right) \\
-c\left(\sigma_{1} p_{1}+\sigma_{2} p_{2}+\sigma_{3} p_{3}\right) & -m c^{2}-E
\end{bmatrix}\begin{bmatrix}
\psi_{L} \\
\psi_{S}
\end{bmatrix}=0
$$
\item To reach the nonrelativistic limit, make the substitution $\mathbf{E}=m c^{2}+\varepsilon$ and approximate $-2 m c^{2}-\varepsilon$ by $-2 m c^{2}$. Then write the matrix equation as two simultaneous two-component equations and show that they can be rearranged to yield
$$
\frac{1}{2 m}\left(p_{1}^{2}+p_{2}^{2}+p_{3}^{2}\right) \psi_{L}=\varepsilon \psi_{L}
$$
which is just the Schrödinger equation for a free particle.
\item Explain why is it reasonable to call $\psi_{L}$ and $\psi_{S}$ "large" and "small."
\end{enumerate}
\end{mybox}



$\boxed{\textbf{Solution}}$ No solution yet.

\newpage

\begin{mybox}{2.2.49}
Show that it is consistent with the requirements that they must satisfy to take the Dirac gamma matrices to be (in $2 \times 2$ block form)
$$
\gamma^{0}=\begin{bmatrix}
0 & \mathbf{1}_{2} \\
\mathbf{1}_{2} & 0
\end{bmatrix}, \quad \gamma^{i}=\begin{bmatrix}
0 & \sigma_{i} \\
-\sigma_{i} & 0
\end{bmatrix}, \quad(i=1,2,3)
$$
This choice for the gamma matrices is called the Weyl representation.
\end{mybox}


$\boxed{\textbf{Solution}}$ If $\mathbf{C}=\mathbf{A} \otimes B$ and $\mathbf{C}'=\mathbf{A}' \otimes B'$ then $\mathbf{C} \mathbf{C}'=\mathbf{A} \mathbf{A}' \otimes B B'$ we have
$$
\begin{array}{l}
\left(\gamma^{0}\right)^{2}=\sigma_{2}^{2} \otimes \hat{1}_{2}^{2} \\
=\hat{1}_{2} \otimes \hat{1}_{2} \quad \text { as } \sigma_{i}^{2}=1 \\
=\hat{1}_{4} \\
=1
\end{array}
$$
as $\gamma^{2}=\begin{bmatrix}-1 & 0 \\ 0 & -1\end{bmatrix}=-\hat{1}_{2}$

$$
\begin{array}{l}
\left(\gamma'\right)^{2}=\gamma^{2} \otimes \sigma_{i}^{2} \\
=\left(-\hat{1}_{2}\right) \otimes \hat{1}_{2} \\
=-\hat{1}_{4} \\
=-1
\end{array}
$$
as $\sigma_{1} \gamma=\begin{bmatrix}0 & 1 \\ 1 & 0\end{bmatrix}\begin{bmatrix}0 & 1 \\ -1 & 0\end{bmatrix}=\begin{bmatrix}-1 & 0 \\ 0 & 1\end{bmatrix}=-\sigma_{3}$
$$
\begin{aligned}
\gamma^{0} \gamma^{i} &=\sigma_{1} \gamma \otimes \mathbf{l}_{2} \sigma_{i} \\
&=\left(-\sigma_{3}\right) \otimes \sigma_{i} \\
\gamma^{i} \gamma^{0} &=\gamma \sigma_{1} \otimes \sigma_{i} \mathbf{l}_{2} \\
&=\sigma_{3} \otimes \sigma_{i}
\end{aligned}
$$
as $\gamma \sigma_{1}=\begin{bmatrix}0 & 1 \\ -1 & 0\end{bmatrix}\begin{bmatrix}0 & 1 \\ 1 & 0\end{bmatrix}=\begin{bmatrix}1 & 0 \\ 0 & -1\end{bmatrix}=\sigma_{3}$
Thus $\gamma^{0} \gamma^{i}+\gamma' \gamma^{0}=0$
$$
\begin{array}{l}
\gamma^{i} \gamma^{j}=\gamma^{2} \otimes \sigma_{i} \sigma_{j} \\
\gamma^{j} \gamma^{i}=\gamma^{2} \otimes \sigma_{j} \sigma_{i}=\gamma^{2} \otimes\left(-\sigma_{i} \sigma_{j}\right)
\end{array}
$$
as $\sigma_{i} \sigma_{j}+\sigma_{j} \sigma_{i}=0$. Thus, $\gamma^{i} \gamma^{j}+\gamma^{j} \gamma^{i}=0$ if $j \neq i$

\newpage


\begin{mybox}{2.2.50}
Show that the Dirac equation separates into independent $2 \times 2$ blocks in the Weyl representation (see Exercise 2.2 .49 ) in the limit that the mass $m$ approaches zero. This observation is important in the ultra relativistic regime where the rest mass is inconsequential, or for particles of negligible mass (e.g., neutrinos).
\end{mybox}


$\boxed{\textbf{Solution}}$ In the Weyl representation, the matrices $\gamma^{0}, \alpha_{i}$ and the wave function $\psi$ written as
$2 \times 2$ blocks take the forms
$$
\gamma^{0}=\begin{bmatrix}
0 & \hat{1}_{2} \\
\hat{1}_{2} & 0
\end{bmatrix}, \quad \alpha_{i}=\begin{bmatrix}
-\sigma_{i} & 0 \\
0 & \sigma_{i}
\end{bmatrix} \quad \text { and } \quad \psi=\begin{bmatrix}
\psi_{1} \\
\psi_{2}
\end{bmatrix}
$$
In block form $\left[\gamma^{0} m c^{2}+\alpha \cdot p\right] \psi=E \psi$ becomes
$$
\left[\begin{bmatrix}
0 & m c^{2} \\
m c^{2} & 0
\end{bmatrix}+\begin{bmatrix}
-\sigma \cdot p & 0 \\
0 & \sigma \cdot p
\end{bmatrix}\right]\begin{bmatrix}
\psi_{1} \\
\psi_{2}
\end{bmatrix}=E\begin{bmatrix}
\psi_{1} \\
\psi_{2}
\end{bmatrix}
$$
If $m$ is zero, this matrix equation becomes two independent equations, one for $\psi_{1}$, and one for
$\psi_{2}$ In this limit, one set of solutions will be with $\psi_{2}=0$ and $\psi_{1}$ a solution to $-\sigma \cdot p \psi_{1}=E \psi_{1}$ and a second set of solutions will have $\psi_{1}=0$ and a set of $\psi_{2}$ identical to the previously found set of $\psi_{1}$ but with values of $E$ of the opposite sign.

\newpage

\begin{mybox}{2.2.51}
\begin{enumerate}[$(a)$]
\item Given $\mathbf{r}'=\mathbf{Ur},$ with $\mathbf{U}$ a unitary matrix and $\mathbf{r}$ a (column) vector with complex elements, show that the magnitude of $\mathbf{r}$ is invariant under this operation. 	
\item The matrix $\mathbf{U}$ transforms any column vector $\mathbf{r}$ with complex elements into $\mathbf{r}'$, leaving the magnitude invariant: $\mathbf{r}^{\dagger} \mathbf{r}=\mathbf{r}^{\prime \dagger} \mathbf{r}'$. Show that $\mathbf{U}$ is unitary.
\end{enumerate}
\end{mybox}

$\boxed{\textbf{Solution}}$ For $(a)$ We show that the magnitude of $r$ is invariant i.e. $\mathbf{r}'^{\dagger} \mathbf{r}'=\mathbf{r}^{\dagger} r$. Consider $\mathbf{r}'^{\dagger} \mathbf{r}'=(\mathbf{U} \mathbf{r})^{\dagger} \mathbf{U} \mathbf{r}$
$$
\begin{array}{l}
=\mathbf{r}^{\dagger}  \mathbf{U}^{\dagger} \mathbf{U} \mathbf{r} \\
=\mathbf{r}^{\dagger} 1 \mathbf{r} \\
=\mathbf{r}^{\dagger} \mathbf{r}
\end{array}
$$
This shows that the magnitude of $r$ is invariant under this operation.




$\boxed{\textbf{Solution}}$ For $(b)$ all $r, \quad \mathbf{r}'^{\dagger} \mathbf{r}'=\mathbf{r}^{\dagger} r$
$$
\begin{array}{l}
(\mathbf{U} \mathbf{r})^{\dagger} \mathbf{U} \mathbf{r}=\mathbf{r}^{\dagger} r \\
\mathbf{r}^{\dagger}  \mathbf{U}^{\dagger} \mathbf{U} \mathbf{r}=\mathbf{r}^{\dagger} 1 r \\
 \mathbf{U}^{\dagger} \mathbf{U}=1
\end{array}
$$
This shows that $U$ is unitary.
