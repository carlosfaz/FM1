\documentclass{article}
\usepackage[utf8]{inputenc}
\usepackage{latexsym}
\usepackage{amsmath}
\usepackage{amssymb}
\usepackage{bm}
\usepackage{multicol}
\usepackage{graphicx}
\usepackage{booktabs}
\usepackage{wrapfig}
\usepackage{fancybox}
\usepackage{bbm}
\usepackage{enumerate}
\pagestyle{plain}
\usepackage{yfonts}
\usepackage{ragged2e}


\usepackage{titling}
\setlength{\droptitle}{-7em} 


\def\Tiny{\fontsize{4pt}{4pt}\selectfont}
\newcommand*{\eqdef}{\ensuremath{\overset{\mathclap{\text{\Tiny def}}}{=}}}


\usepackage{geometry}
 \geometry{
 a4paper,
 total={160mm,257mm},
 left=24mm,
 top=20mm,
 }
 
 \usepackage{tcolorbox}
\newtcolorbox{mybox}[1]{colback=blue!5!white,colframe=blue!42!black,fonttitle=\bfseries,title=Problem #1}
 

\title{Title}
\author{Carlos Faz}
\date{ \ }

\begin{document}

\maketitle

\begin{flushleft}

\begin{mybox}{3.3.1}
A rotation $\varphi_{1}+\varphi_{2}$ about the $z$ -axis is carried out as two successive rotations $\varphi_{1}$ and $\varphi_{2},$ each about the $z$-axis. Use the matrix representation of the rotations to derive the trigonometric identities
\end{mybox}
$\boxed{\textbf{Solution}}$
$$
\begin{aligned} \cos \left(\varphi_{1}+\varphi_{2}\right) &=\cos \varphi_{1} \cos \varphi_{2}-\sin \varphi_{1} \sin \varphi_{2} \\ \sin \left(\varphi_{1}+\varphi_{2}\right) &=\sin \varphi_{1} \cos \varphi_{2}+\cos \varphi_{1} \sin \varphi_{2} \end{aligned}
$$
$$\begin{bmatrix}{\cos \left(\varphi_{1}+\varphi_{2}\right) \sin \left(\varphi_{1}+\varphi_{2}\right)} \\ {-\sin \left(\varphi_{1}+\varphi_{2}\right) \cos \left(\varphi_{1}+\varphi_{2}\right)}\end{bmatrix}=\begin{bmatrix}{\cos \varphi_{2} \sin \varphi_{2}} \\ {-\sin \varphi_{2} \cos \varphi_{2}}\end{bmatrix}\begin{bmatrix}{\cos \varphi_{1} \sin \varphi_{1}} \\ {-\sin \varphi_{1} \cos \varphi_{1}}\end{bmatrix}$$
$$=\begin{bmatrix}{\cos \varphi_{1} \cos \varphi_{2}-\sin \varphi_{1} \sin \varphi_{2}} & {\sin \varphi_{1} \cos \varphi_{2}+\cos \varphi_{1} \sin \varphi_{2}} \\ {-\cos \varphi_{1} \sin \varphi_{2}-\sin \varphi_{1} \cos \varphi_{2}} & {-\sin \varphi_{1} \sin \varphi_{2}+\cos \varphi_{1} \cos \varphi_{2}}\end{bmatrix}$$



\begin{mybox}{3.3.2}
A corner reflector is formed by three mutually perpendicular reflecting suffices. Show that a a ay of light incident tupon the cometor (striking all three surfaces) is reflected back along a line parallel to the line of incidence. Hint. Consider the effect of a reflection on the components of a vector describing the direction of the light ray.
\end{mybox}

$\vspace{3mm}$

$\boxed{\textbf{Solution}}$ Here we are asked prove that the ray of light incident upon the corner reflector is reflected of back along line parallel to line of incidence.
So for this align the reflecting surfaces with $xy$, $xz$, and $yz$ planes. If an incoming ray strikes the $xy$ plane, the $z$ component of its direction of propagation is reversed. A strike on the $xz$ plane
reverses its $y$ component, and a strike on $yz$ plane reverses its $x$ component.

$\vspace{3mm}$



\begin{mybox}{3.3.3}
Let $x$ and $y$ be column vectors. Under an orthogonal transformation $S$, they become
$x'=S x$ and $y'=S y .$ Show that $\left(x'\right)^{T} y'=x^{T} y,$ a result equivalent to the invariance of the dot product under a rotational transformation.
\end{mybox}

$\boxed{\textbf{Solution}}$ It is given that $S$ is orthogonal, if so its transpose is also its inverse.
From this
$$
\left(x'\right)^{T}=(S x)^{T}=x^{T} \mathbf{S}^{T}=x^{T} \mathbf{S}^{-1}
$$
Then
$$
\left(x'\right)^{T} y'=x^{T} \mathbf{S}^{-1} S y=x^{T} y
$$
Therefore $\left(x'\right)^{T} y'=x^{T} y$




\begin{mybox}{3.3.4}
Given the orthogonal transformation matrix $S$ and vectors a and $\mathbf{b}$, 
$$S=\begin{bmatrix}{0.80} & {0.60} & {0.00} \\ {-0.48} & {0.64} & {0.60} \\ {0.36} & {-0.48} & {0.80}\end{bmatrix} \quad \mathbf{a}=\begin{bmatrix}{1} \\ {0} \\ {1}\end{bmatrix}, \quad \mathbf{b}=\begin{bmatrix}{0} \\ {2} \\ {-1}\end{bmatrix}$$
\begin{enumerate}[$(a)$]
\item Calculate $\text{det}(S)$.
\item Verify that $\mathbf{a}\cdot \mathbf{b}$ is invariant under application of $\mathbf{S}$ to $\mathbf{a}$ and $\mathbf{b}$.
\item Determine what happens to $\mathbf{a}\times \mathbf{b}$ under application of $\mathbf{S}$ to $\mathbf{a}$ and $\mathbf{b}$. Is this what is expected?
\end{enumerate}
\end{mybox}

$\boxed{\textbf{Solution}}$ For $(a)$ given 
$$
S=\begin{bmatrix}
0.80 & 0.60 & 0.00 \\
-0.48 & 0.64 & 0.60 \\
0.36 & -0.48 & 0.80
\end{bmatrix}
$$
$$
\operatorname{det}(S)=\operatorname{det}\begin{bmatrix}
0.80 & 0.60 & 0.00 \\
-0.48 & 0.64 & 0.60 \\
0.36 & -0.48 & 0.80
\end{bmatrix}=1
$$
$\boxed{\textbf{Solution}}$ For $(b)$ we show that $a \cdot b$ is invariant under application of $\mathbf{S}$ to $a$ and $b$.
$$
\begin{aligned}
\mathbf{a}' &=\mathbf{S} \mathbf{a} \\
&=\begin{bmatrix}
0.80 & 0.60 & 0.00 \\
-0.48 & 0.64 & 0.60 \\
0.36 & -0.48 & 0.80
\end{bmatrix}\begin{bmatrix}
1 \\
0 \\
1
\end{bmatrix} \\
&=\begin{bmatrix}
0.80 \\
0.12 \\
1.16
\end{bmatrix}
\end{aligned}
$$
$$
\begin{aligned}
\mathbf{b}' &=\mathbf{S} \mathbf{b} \\
&=\begin{bmatrix}
0.80 & 0.60 & 0.00 \\
-0.48 & 0.64 & 0.60 \\
0.36 & -0.48 & 0.80
\end{bmatrix}\begin{bmatrix}
0 \\
2 \\
-1
\end{bmatrix} \\
&=\begin{bmatrix}
1.20 \\
0.68 \\
-1.76
\end{bmatrix}
\end{aligned}
$$
$$
a \cdot b=\begin{bmatrix}
1 & 0 & 1
\end{bmatrix} \cdot\begin{bmatrix}
0 \\
2 \\
-1
\end{bmatrix}=-1
$$
$$
a' \cdot b'=\begin{bmatrix}
0.80 & 0.12 & 1.16
\end{bmatrix}\begin{bmatrix}
1.20 \\
0.68 \\
-1.76
\end{bmatrix}=-1
$$
Thus, $a \cdot b$ is invariant under application of $\mathbf{S}$ to $a$ and $b$.

$\boxed{\textbf{Solution}}$ For $(c)$ we find $\mathbf{a} \times \mathbf{b}$
$$
\begin{aligned}
\mathbf{a} \times \mathbf{b}=\left|\begin{array}{ccc}
i & j & k \\
1 & 0 & 1 \\
0 & 2 & -1
\end{array}\right| &=\begin{bmatrix}
-2 \\
1 \\
2
\end{bmatrix} \\
\mathbf{S}(\mathbf{a} \times \mathbf{b})=&\begin{bmatrix}
0.80 & 0.60 & 0.00 \\
-0.48 & 0.64 & 0.60 \\
0.36 & -0.48 & 0.80
\end{bmatrix}\begin{bmatrix}
-2 \\
1 \\
2
\end{bmatrix} \\
=&\begin{bmatrix}
-1 \\
2.8 \\
0.4
\end{bmatrix}
\end{aligned}
$$
$$
\mathbf{a}' \times \mathbf{b}'=\left|\begin{array}{ccc}
i & j & k \\
0.80 & 0.12 & 1.16 \\
1.20 & 0.68 & -1.76
\end{array}\right|=\begin{bmatrix}
-1 \\
2.8 \\
0.4
\end{bmatrix}
$$
Thus, $\mathbf{S}(\mathbf{a} \times \mathbf{b})=\mathbf{a}' \times \mathbf{b}'$ and hence $\mathbf{a} \times \mathbf{b}$ is a vector.






\begin{mybox}{3.3.5}
Using $\mathbf{a}$ and $\mathbf{b}$ as defined in Exercise 3.3.5 but with

$$S=\begin{bmatrix}{0.60} & {0.00} & {0.80} \\ {-0.64} & {-0.60} & {0.48} \\ {-0.48} & {0.80} & {0.36}\end{bmatrix} \quad \text{and} \quad \mathbf{c}=\begin{bmatrix}{2} \\ {1} \\ {3}\end{bmatrix}$$

\begin{enumerate}[$(a)$]
\item Calculate $\text{det}(S)$.
\item $\mathbf{a}\times \mathbf{b}$
\item $(\mathbf{a}\times \mathbf{b}) \cdot \mathbf{c}$
\item $\mathbf{a}\times (\mathbf{b}\times \mathbf{c})$
\end{enumerate}
\end{mybox}

$\boxed{\textbf{Solution}}$ For $(a)$ Given that
$$S=\begin{bmatrix}0.60 & 0.00 & 0.80 \\ -0.64 & -0.60 & 0.48 \\ -0.48 & 0.80 & 0.36\end{bmatrix}$$
Then
$$
\operatorname{det}(S)=\operatorname{det}\begin{bmatrix}
0.60 & 0.00 & 0.80 \\
-0.64 & -0.60 & 0.48 \\
-0.48 & 0.80 & 0.36
\end{bmatrix}=1
$$
Apply $\mathbf{S}$ to $\mathbf{a}, \mathbf{b},$ and $c$.
$$
\begin{aligned}
\mathbf{a}' &=\mathbf{S} \mathbf{a} \\
&=\begin{bmatrix}
0.60 & 0.00 & 0.80 \\
-0.64 & -0.60 & 0.48 \\
-0.48 & 0.80 & 0.36
\end{bmatrix}\begin{bmatrix}
1 \\
0 \\
1
\end{bmatrix} \\
&=\begin{bmatrix}
1.40 \\
-0.16 \\
-0.12
\end{bmatrix}
\end{aligned}
$$
$$
\begin{aligned}
\mathbf{b}' &=\mathbf{S} \mathbf{b} \\
&=\begin{bmatrix}
0.60 & 0.00 & 0.80 \\
-0.64 & -0.60 & 0.48 \\
-0.48 & 0.80 & 0.36
\end{bmatrix}\begin{bmatrix}
0 \\
2 \\
-1
\end{bmatrix} \\
&=\begin{bmatrix}
-0.80 \\
-1.68 \\
1.24
\end{bmatrix}
\end{aligned}
$$
$$
\begin{aligned}
\mathbf{c}' &=\mathbf{S} \mathbf{c} \\
&=\begin{bmatrix}
0.60 & 0.00 & 0.80 \\
-0.64 & -0.60 & 0.48 \\
-0.48 & 0.80 & 0.36
\end{bmatrix}\begin{bmatrix}
2 \\
1 \\
3
\end{bmatrix} \\
&=\begin{bmatrix}
3.60 \\
-0.44 \\
0.92
\end{bmatrix}
\end{aligned}
$$
Now, we determine what happen to $\mathbf{a} \times \mathbf{b}$ under application of $\mathbf{S}$ to $\mathbf{a}, \mathbf{b}, \mathbf{c}$.


$\boxed{\textbf{Solution}}$ For $(b)$
$$(a\times b)=\begin{bmatrix}
-2 \\
1 \\
2
\end{bmatrix}$$

$$
\begin{aligned}
\mathbf{S}(\mathbf{a} \times \mathbf{b}) &=\begin{bmatrix}
0.60 & 0.00 & 0.80 \\
-0.64 & -0.60 & 0.48 \\
-0.48 & 0.80 & 0.36
\end{bmatrix}\begin{bmatrix}
-2 \\
1 \\
2
\end{bmatrix} \\
&=\begin{bmatrix}
0.40 \\
1.64 \\
2.48
\end{bmatrix}
\end{aligned}
$$

$$
\mathbf{a}' \times \mathbf{b}'=\left|\begin{array}{ccc}
i & j & k \\
1.40 & -0.16 & -0.12 \\
-0.80 & -1.68 & 1.24
\end{array}\right|=\begin{bmatrix}
-0.40 \\
-1.64 \\
-2.48
\end{bmatrix}
$$
Thus, $\mathbf{S}(\mathbf{a} \times \mathbf{b})=\mathbf{a}' \times \mathbf{b}'$

$\boxed{\textbf{Solution}}$ For $(c)
$ we determine what happen to $(\mathbf{a} \times \mathbf{b}) \cdot \mathbf{c}$ under application of $\mathbf{S}$ to $\mathbf{a}, \mathbf{b}, \mathbf{c}$
$$
(\mathbf{a} \times \mathbf{b}) \cdot \mathbf{c}=\begin{bmatrix}
-2 & 1 & 2
\end{bmatrix} \cdot\begin{bmatrix}
2 \\
1 \\
3
\end{bmatrix}=-4+1+6=3
$$
$$
\left(\mathbf{a}' \times \mathbf{b}'\right) \cdot \mathbf{c}'=\begin{bmatrix}
-0.40 & -1.64 & -2.48
\end{bmatrix} \cdot\begin{bmatrix}
3.60 \\
-0.44 \\
0.92
\end{bmatrix}=-3
$$
Thus, $(\mathbf{a} \times \mathbf{b}) \cdot \mathbf{c}=-\left(\mathbf{a}' \times \mathbf{b}'\right) \cdot \mathbf{c}'$




$\boxed{\textbf{Solution}}$ For $(d)$ We now determine what happen to $\mathbf{a}\times(\mathbf{b} \times \mathbf{c})$ under application of $\mathbf{S}$ to $\mathbf{a}, \mathbf{b}, \mathbf{c}$.
$$
\mathbf{a}\times(\mathbf{b} \times \mathbf{c})=\left|\begin{array}{ccc}
1 & 0 & 1 \\
0 & 2 & -1 \\
2 & 1 & 3
\end{array}\right|=\begin{bmatrix}
2 \\
11 \\
-2
\end{bmatrix}
$$
$$
\mathbf{S}(\mathbf{a}\times(\mathbf{b} \times \mathbf{c}))=\begin{bmatrix}
0.60 & 0.00 & 0.80 \\
-0.64 & -0.60 & 0.48 \\
-0.48 & 0.80 & 0.36
\end{bmatrix}\begin{bmatrix}
2 \\
11 \\
-2
\end{bmatrix}=\begin{bmatrix}
-0.40 \\
-8.84 \\
7.12
\end{bmatrix}
$$
$$
\mathbf{a}' \times\left(\mathbf{b}' \times \mathbf{c}'\right)=\left|\begin{array}{ccc}
1.40 & -0.16 & -0.12 \\
-0.80 & -1.68 & 1.24 \\
3.60 & -0.44 & 0.92
\end{array}\right|=\begin{bmatrix}
-0.40 \\
-8.84 \\
7.12
\end{bmatrix}
$$
Thus, $\quad \mathbf{S}(\mathbf{a}\times(\mathbf{b} \times \mathbf{c}))=\mathbf{a}' \times\left(\mathbf{b}' \times \mathbf{c}'\right)$






\end{flushleft}
\end{document}
