



\begin{mybox}{3.2.1}
If $\mathbf{P}=\hat{\mathbf{e}}_{x} P_{x}+\hat{\mathbf{e}}_{y} P_{y}$ and $\mathbf{Q}=\hat{\mathbf{e}}_{x} Q_{x}+\hat{\mathbf{e}}_{y} Q_{y}$ are any two nonparallel (Also nonantiparallectors in the $x y$ -plane, show that $\mathbf{P} \times \mathbf{Q}$ is in the $z$ -direction.
\end{mybox}

$\boxed{\textbf{Solution}}$ We write the $P$ and $Q$ vectors as
$$\mathbf{P} = \langle P_x, P_y,0\rangle \quad Q = \langle Q_x, Q_y,0\rangle$$
So
$$\mathbf{P} \times \mathbf{Q} = \begin{vmatrix}
\hat{\mathbf{e}}_{x} & \hat{\mathbf{e}}_{y} & \hat{\mathbf{e}}_{z}\\ 
P_{x} & P_{y} & 0\\ 
Q_{x} & Q_{y} & 0
\end{vmatrix} =  (P_{x} Q_{y} - Q_{x} P_{y})\hat{\mathbf{e}}_{z}$$

$$$$


\begin{mybox}{3.2.2}
Prove that $(\mathbf{A} \times \mathbf{B}) \cdot(\mathbf{A} \times \mathbf{B})=(\mathbf{A} \mathbf{B})^{2}-(\mathbf{A} \cdot \mathbf{B})^{2}$
\end{mybox}


$\boxed{\textbf{Solution}}$
$$(\mathbf{A} \times \mathbf{B})^{2}=(|\mathbf{A}||\mathbf{B}| \sin \theta)^{2}$$
$$(\mathbf{A} \times \mathbf{B})^{2}=\mathbf{A}^{2} \times \mathbf{B}^{2} \times \sin ^{2} \theta$$
$$(\mathbf{A} \times \mathbf{B})^{2}=\mathbf{A}^{2} \times \mathbf{B}^{2} \times\left(1-\cos ^{2} \theta\right)$$
$$(\mathbf{A} \times \mathbf{B})^{2}=\mathbf{A}^{2} \times \mathbf{B}^{2}-\mathbf{A}^{2} \times \mathbf{B}^{2} \times\left(\cos ^{2} \theta\right)$$
$$(\mathbf{A} \times \mathbf{B})^{2}=\mathbf{A}^{2} \mathbf{B}^{2}-(\mathbf{A} \cdot \mathbf{B})^{2}$$

$$$$

\begin{mybox}{3.2.3}
Using the vectors
$$\mathbf{P}=\hat{\mathbf{e}}_{x} \cos \theta+\hat{\mathbf{e}}_{y} \sin \theta$$
$$\mathbf{Q}=\hat{\mathbf{e}}_{x} \cos \varphi-\hat{\mathbf{e}}_{y} \sin \varphi$$
$$\mathbf{R}=\hat{\mathbf{e}}_{x} \cos \varphi+\hat{\mathbf{e}}_{y} \sin \varphi$$
prove the familiar trigonometric identities
$$
\begin{aligned} \sin (\theta+\varphi) &=\sin \theta \cos \varphi+\cos \theta \sin \varphi \\ \cos (\theta+\varphi) &=\cos \theta \cos \varphi-\sin \theta \sin \varphi \end{aligned}
$$
\end{mybox}


$\boxed{\textbf{Solution}}$ Consider $\mathbf{P}\cdot \mathbf{Q}$ as
$$\mathbf{P}\cdot \mathbf{Q} = (\hat{x} \cos \theta+\hat{y} \sin \theta) \cdot\left(\hat{x} \cos \varphi-\hat{y} \sin \varphi\right)+\hat{y} \sin \theta \hat{x} \cos \varphi-\hat{y} \sin \theta \sin \varphi$$
$$\mathbf{P}\cdot \mathbf{Q} = (1 \times \cos \theta \cos \varphi)-(0 \times \cos \theta \sin \varphi)+(0 \times \sin \theta \cos \varphi) - (1\times \sin \theta \sin \varphi)$$
$$\mathbf{P}\cdot \mathbf{Q} = \cos \theta \cos \varphi-\sin \theta \sin \varphi$$
And by the product rule, $\mathbf{P}\cdot \mathbf{Q} = \cos(\theta + \varphi)$
$$\cos (\theta+\varphi)=\cos \theta \cos \varphi-\sin \theta \sin \varphi$$



\newpage
 
\begin{mybox}{3.2.4}
\begin{enumerate}[$(a)$]
\item Find a vector $\mathbf{A}$ that is perpendicular to
$$\mathbf{U}=2 \hat{\mathbf{e}}_{x}+\hat{\mathbf{e}}_{y}-\hat{\mathbf{e}}_{z}$$
$$\mathbf{V}=\hat{\mathbf{e}}_{x}-\hat{\mathbf{e}}_{y}+\hat{\mathbf{e}}_{z}$$
\item What is $\mathbf{A}$ if, in addition to this requirement, we demand that it have unit
magnitude?
\end{enumerate}
\end{mybox}

$\boxed{\textbf{Solution}}$ For $(a)$ we have $\mathbf{U}=2 \hat{\mathbf{e}}_{x}+\hat{\mathbf{e}}_{y}-\hat{\mathbf{e}}_{z}, V=\hat{\mathbf{e}}_{x}-\hat{\mathbf{e}}_{y}+\hat{\mathbf{e}}_{z}$
$$
\mathbf{U} \times \mathbf{V}=\left|\begin{array}{ccc}
\hat{\mathbf{e}}_{x} & \hat{\mathbf{e}}_{y} & \hat{\mathbf{e}}_{z} \\
2 & 1 & -1 \\
1 & -1 & 1
\end{array}\right|=\hat{\mathbf{e}}_{x}(1-1)-\hat{\mathbf{e}}_{y}(2+1)+\hat{\mathbf{e}}_{z}(-2-1)
$$
$$
\mathbf{U} \times \mathbf{V}=-\hat{\mathbf{e}}_{y}(3)+\hat{\mathbf{e}}_{z}(-3)=-3 \hat{\mathbf{e}}_{y}-3 \hat{\mathbf{e}}_{z}
$$

$\boxed{\textbf{Solution}}$ For $(b)$ We know $\mathbf{A}$ is $-3 \hat{\mathbf{e}}_{y}-3 \hat{\mathbf{e}}_{z}$, so the magnitude of $\mathbf{A}$ is
$$|\mathbf{A}|=\sqrt{3^{2}+3^{2}}=\sqrt{18}=3 \sqrt{2}$$
From this 
$$\mathbf{A}=\frac{-3 \hat{\mathbf{e}}_{y}-3 \hat{\mathbf{e}}_{z}}{3 \sqrt{2}}=\frac{-\hat{\mathbf{e}}_{y}-\hat{\mathbf{e}}_{z}}{\sqrt{2}}$$


$$$$



\begin{mybox}{3.2.5}
If four vectors $\mathbf{a}$, $\mathbf{b}$, $\mathbf{c}$, and $\mathbf{d}$ all lie in the same plane, show that
$$
(\mathbf{a} \times \mathbf{b}) \times(\mathbf{c} \times \mathbf{d})=0
$$
Hint. Consider the directions of the cross-product vectors.
\end{mybox}

$\boxed{\textbf{Solution}}$ Since all four vectors lie in the same plane, the cross product of any two of them would be orthogonal to the plane. Thus:
$$\mathbf{v}_1 = (\mathbf{a} \times \mathbf{b})$$
$$\mathbf{v}_2 = (\mathbf{c} \times \mathbf{d})$$
By definition, it $\mathbf{v}_1$ and $\mathbf{v}_2$ are parallel, so $\mathbf{v}_1 \times \mathbf{v}_2 = 0$


\newpage



\begin{mybox}{3.2.6}
Derive the law of sines (see Fig. $3.4):$
$$
\dfrac{\sin \alpha}{|\mathbf{A}|}=\dfrac{\sin \beta}{|\mathbf{B}|}=\dfrac{\sin \gamma}{|\mathbf{C}|}
$$
\end{mybox}


$\boxed{\textbf{Solution}}$ We have $\mathbf{A}-\mathbf{B}-\mathbf{C}=0$ so we cross both sides by $\mathbf{A}$ 



$$\mathbf{A} \times \mathbf{A}-\mathbf{A} \times \mathbf{B}-\mathbf{A} \times \mathbf{C}=\mathbf{A} \times 0$$
$$ 0-\mathbf{A} \times \mathbf{B}-\mathbf{A} \times \mathbf{C}=0$$
$$-\mathbf{A} \times \mathbf{B}-\mathbf{A} \times \mathbf{C}=0$$
$$-\mathbf{A} \times \mathbf{C}=\mathbf{A} \times \mathbf{B}$$
$$ \mathbf{C} \times \mathbf{A}=\mathbf{A} \times \mathbf{B}$$
$$|\mathbf{C}||\mathbf{A}| \sin \beta=|\mathbf{A}| \mathbf{B} \mid \sin \gamma$$

Again, we cross both sides of $\mathbf{A}-\mathbf{B}-\mathbf{C}=0$ by $\mathbf{B}$
$$\mathbf{B} \times \mathbf{A}-\mathbf{B} \times \mathbf{B}-\mathbf{B} \times \mathbf{C}=\mathbf{B} \times 0$$
$$ \mathbf{B} \times \mathbf{A}-\mathbf{B} \times \mathbf{C}=0$$
$$ \mathbf{B} \times \mathbf{A}=\mathbf{B} \times C$$
$$|\mathbf{B} \| A| \sin \gamma=|\mathbf{B}||\mathbf{C}| \sin \alpha$$
$$|\mathbf{A}| \sin \gamma=|\mathbf{C}| \sin \alpha$$
$$ \frac{\sin \gamma}{|\mathbf{C}|}=\frac{\sin \alpha}{|\mathbf{A}|} $$



\newpage



\begin{mybox}{3.2.7}
The magnetic induction $\mathbf{B}$ is defined by the Lorentz force equation,
$$
\mathbf{F}=q(\mathbf{v} \times \mathbf{B})
$$
Carrying out three experiments, we find that if
$$
\begin{array}{l}{\mathbf{v}=\hat{\mathbf{e}}_{x}, \quad \dfrac{\mathbf{F}}{q}=2 \hat{\mathbf{e}}_{z}-4 \hat{\mathbf{e}}_{y}}\vspace{3mm} \\ {\mathbf{v}=\hat{\mathbf{e}}_{y}, \quad \dfrac{\mathbf{F}}{q}=4 \hat{\mathbf{e}}_{x}-\hat{\mathbf{e}}_{z}} \vspace{3mm}\\ {\mathbf{v}=\hat{\mathbf{e}}_{z}, \quad \dfrac{\mathbf{F}}{q}=\hat{\mathbf{e}}_{y}-2 \hat{\mathbf{e}}_{x}}\end{array}
$$
From the results of these three separate experiments calculate the magnetic induction $\mathbf{B}$.
\end{mybox}



$\boxed{\textbf{Solution}}$ From the first condition $\mathbf{v}=\hat{\mathbf{e}}_{x}, \dfrac{F}{q}=2 \hat{\mathbf{e}}_{z}-4 \hat{\mathbf{e}}_{\dot{y}}$

$$
\mathbf{v} \times \mathbf{B}=\left|\begin{array}{ccc}
\hat{\mathbf{e}}_{x} & \hat{\mathbf{e}}_{y} & \hat{\mathbf{e}}_{z} \\
1 & 0 & 0 \\
\mathbf{B}_{x} & \mathbf{B}_{y} & \mathbf{B}_{z}
\end{array}\right|=\hat{\mathbf{e}}_{x}(0)-\hat{\mathbf{e}}_{y}\left(\mathbf{B}_{z}\right)+\hat{\mathbf{e}}_{z}\left(\mathbf{B}_{y}\right)=-\hat{\mathbf{e}}_{y}\left(\mathbf{B}_{z}\right)+\hat{\mathbf{e}}_{z}\left(\mathbf{B}_{y}\right)
$$

$$\frac{\mathbf{F}}{q}=2 \hat{\mathbf{e}}_{z}-4 \hat{\mathbf{e}}_{j},$$
$$ \mathbf{v} \times \mathbf{B}=-\hat{\mathbf{e}}_{y}\left(\mathbf{B}_{z}\right)+\hat{\mathbf{e}}_{z}\left(\mathbf{B}_{y}\right)$$
$$\mathbf{B}_{z}=4, \mathbf{B}_{y}=2$$

Now, from the second condition $\mathbf{v}=\hat{\mathbf{e}}_{y}, \dfrac{F}{q}=4 \hat{\mathbf{e}}_{x}-\hat{\mathbf{e}}_{z}$

$$
\mathbf{v} \times \mathbf{B}=\left|\begin{array}{ccc}
\hat{\mathbf{e}}_{x} & \hat{\mathbf{e}}_{y} & \hat{\mathbf{e}}_{z} \\
0 & 1 & 0 \\
\mathbf{B}_{x} & \mathbf{B}_{y} & \mathbf{B}_{z}
\end{array}\right|=\hat{\mathbf{e}}_{x}\left(\mathbf{B}_{z}\right)-\hat{\mathbf{e}}_{y}(0)-\hat{\mathbf{e}}_{z}\left(\mathbf{B}_{x}\right)=\hat{\mathbf{e}}_{x}\left(\mathbf{B}_{z}\right)-\hat{\mathbf{e}}_{z}\left(\mathbf{B}_{x}\right)
$$
$$\frac{\mathbf{F}}{q}=4 \hat{\mathbf{e}}_{x}-\hat{\mathbf{e}}_{z}$$
$$ \mathbf{v} \times \mathbf{B}=\hat{\mathbf{e}}_{x}\left(\mathbf{B}_{z}\right)-\hat{\mathbf{e}}_{z}\left(\mathbf{B}_{x}\right)$$
$$\mathbf{B}_{z}=4, \mathbf{B}_{x}=1$$
From the third condition 
$$
\mathbf{v}=\hat{\mathbf{e}}_{z}$$
$$ \frac{\mathbf{F}}{q}=\hat{\mathbf{e}}_{y}-2 \hat{\mathbf{e}}_{x}
$$
$$
\mathbf{v} \times \mathbf{B}=\left|\begin{array}{ccc}
\hat{\mathbf{e}}_{x} & \hat{\mathbf{e}}_{y} & \hat{\mathbf{e}}_{z} \\
0 & 0 & 1 \\
\mathbf{B}_{x} & \mathbf{B}_{y} & \mathbf{B}_{z}
\end{array}\right|=\hat{\mathbf{e}}_{x}\left(-\mathbf{B}_{y}\right)-\hat{\mathbf{e}}_{y}\left(-\mathbf{B}_{x}\right)-\hat{\mathbf{e}}_{z}(0)=-\hat{\mathbf{e}}_{x}\left(\mathbf{B}_{y}\right)+\hat{\mathbf{e}}_{y}\left(\mathbf{B}_{x}\right)
$$
$$\mathbf{v} \times \mathbf{B}=-\hat{\mathbf{e}}_{x}\left(\mathbf{B}_{y}\right)+\hat{\mathbf{e}}_{y}\left(\mathbf{B}_{x}\right)$$
$$ \frac{\mathbf{F}}{q}=\hat{\mathbf{e}}_{y}-2 \hat{\mathbf{e}}_{x}$$
$$\mathbf{B}_{y}=2, \mathbf{B}_{x}=1$$

\newpage


\begin{mybox}{3.2.8}
You are given the three vectors $\mathbf{A}, \mathbf{B},$ and $\mathbf{C},$
$$
\begin{array}{l}{\mathbf{A}=\hat{\mathbf{e}}_{x}+\hat{\mathbf{e}}_{y}} \\ {\mathbf{B}=\hat{\mathbf{e}}_{y}+\hat{\mathbf{e}}_{z}} \\ {\mathbf{C}=\hat{\mathbf{e}}_{x}-\hat{\mathbf{e}}_{z}}\end{array}
$$

Therefore, from above three conditions magnetic induction is given by $ \mathbf{B} = \hat { x } + 2 \hat { y } + 4 \hat { z }$
\end{mybox}


$\boxed{\textbf{Solution}}$ For $(a)$, $\mathbf{A} \cdot \mathbf{B} \times \mathbf{C}=0 .$ Because $\mathbf{A}$ is the plane of $\mathbf{B}$ and $\mathbf{C}$. The parallelepiped has zero height above the $BC$ plane. So therefore volume will be zero.
Therefore, the scalar triple product is zero.




$\boxed{\textbf{Solution}}$ For $(b)$ 
$$
(\mathbf{B} \times \mathbf{C})=\left|\begin{array}{ccc}
\hat{\mathbf{e}}_{x} & \hat{\mathbf{e}}_{y} & \hat{\mathbf{e}}_{z} \\
0 & 1 & 1 \\
1 & 0 & -1
\end{array}\right|=\hat{\mathbf{e}}_{x}(-1)-\hat{\mathbf{e}}_{y}(-1)+\hat{\mathbf{e}}_{z}(-1)=-\hat{\mathbf{e}}_{x}+\hat{\mathbf{e}}_{y}-\hat{\mathbf{e}}_{z}
$$
$$
\mathbf{A} \times(\mathbf{B} \times \mathbf{C})=\left|\begin{array}{ccc}
\hat{\mathbf{e}}_{x} & \hat{\mathbf{e}}_{y} & \hat{\mathbf{e}}_{z} \\
1 & 1 & 0 \\
-1 & 1 & -1
\end{array}\right|=\hat{\mathbf{e}}_{x}(-1)-\hat{\mathbf{e}}_{y}(-1)+\hat{\mathbf{e}}_{z}(1+1)=-\hat{\mathbf{e}}_{x}+\hat{\mathbf{e}}_{y}+2 \hat{\mathbf{e}}_{z}
$$


$$$$

\begin{mybox}{3.2.9}
Prove Jacobi's identity for vector products:
$$
\mathbf{a} \times(\mathbf{b} \times \mathbf{c})+\mathbf{b} \times(\mathbf{c} \times \mathbf{a})+\mathbf{c} \times(\mathbf{a} \times \mathbf{b})=0
$$
\end{mybox}


$\boxed{\textbf{Solution}}$ From $\mathbf{BA}\mathbf{C}-\mathbf{C}\mathbf{AB}$ rule $\mathbf{a}\times(\mathbf{b} \times \mathbf{c})=\mathbf{b}(\mathbf{a}\cdot \mathbf{c})-\mathbf{c}(\mathbf{a} \cdot \mathbf{b}).$ The entire equation an written as
$$
\mathbf{a}\times(\mathbf{b} \times \mathbf{c})+\mathbf{b} \times(\mathbf{c} \times \mathbf{a})+\mathbf{c} \times(\mathbf{a} \times \mathbf{b})$$
$$=[\mathbf{b}(\mathbf{a} \cdot \mathbf{c})-\mathbf{c}(\mathbf{a} \cdot \mathbf{b})]+[(\mathbf{b} \cdot \mathbf{a}) \mathbf{c}-(\mathbf{b} \cdot \mathbf{c}) \mathbf{a}]+[(\mathbf{c} \cdot \mathbf{b}) \mathbf{a}-(\mathbf{c} \cdot \mathbf{a}) \mathbf{b}]
$$
since the dot product is commutative so they becomes zero.
Therefore, $$ \mathbf{a}\times(\mathbf{b} \times \mathbf{c})+\mathbf{b} \times(\mathbf{c} \times \mathbf{a})+\mathbf{c} \times(\mathbf{a} \times \mathbf{b})=0$$


\newpage


\begin{mybox}{3.2.10}
A vector $\mathbf{A}$ is decomposed into a radial vector $\mathbf{A}_{r}$ and a tangential vector $\mathbf{A}_{t} .$ If $\hat{\mathbf{r}}$ is a unit vector in the radial direction, show that
$(a)$ $\mathbf{A}_{r}=\hat{\mathbf{r}}(\mathbf{A} \cdot \hat{\mathbf{r}})$ and
$(b)$ $\mathbf{A}_{t}=-\hat{\mathbf{r}} \times(\hat{\mathbf{r}} \times \mathbf{A})$
\end{mybox}


$\boxed{\textbf{Solution}}$ Let $\mathbf{A}=\mathbf{A}_{r} \hat{\mathbf{r}}+\mathbf{A}_{i} \hat{\theta}$
$$
\mathbf{A} \cdot \hat{\mathbf{r}}=\mathbf{A}_{r}, \text { as } \ \hat{\mathbf{r}} \cdot \hat{\mathbf{r}}=1
$$
The left-hand side is:
$$\mathbf{A}_{r}=\mathbf{A}_{r} \hat{\mathbf{r}}$$ since $\hat{\mathbf{r}}$ is the unit vector.
The right-hand side is:
$$
\hat{\mathbf{r}}(\mathbf{A} \cdot \hat{\mathbf{r}})=\hat{\mathbf{r}}\left(\mathbf{A}_{r}\right)=\mathbf{A}_{r} \hat{\mathbf{r}}
$$
For $(b)$, taking dot product of both sides of the equation $\mathbf{A}_{t}=-\hat{\mathbf{r}} \times(\hat{\mathbf{r}} \times \mathbf{A})$ by $\hat{\mathbf{r}}$ we get
$$
\mathbf{A}_{t} \cdot \hat{\mathbf{r}}=[-\hat{\mathbf{r}} \times(\hat{\mathbf{r}} \times \mathbf{A})] \cdot \hat{\mathbf{r}}
$$
The left-hand side is:
$$
\mathbf{A}_{t} \cdot \hat{\mathbf{r}}=\mathbf{A}_{t} \hat{\theta} \cdot \hat{\mathbf{r}}=0
$$

$$
\begin{aligned}
[-\hat{\mathbf{r}} \times(\hat{\mathbf{r}} \times \mathbf{A})] \cdot \hat{\mathbf{r}}&=[\hat{\mathbf{r}} \times(\mathbf{A} \times \hat{\mathbf{r}})] \cdot \hat{\mathbf{r}} \\
&=[\mathbf{A}(\hat{\mathbf{r}} \cdot \hat{\mathbf{r}})-\hat{\mathbf{r}}(\mathbf{A} \cdot \hat{\mathbf{r}})] \cdot \hat{\mathbf{r}} \\
&=\left[\mathbf{A}-\hat{\mathbf{r}} \mathbf{A}_{r}\right] \cdot \hat{\mathbf{r}} \\
&=\mathbf{A} \cdot \hat{\mathbf{r}}-\mathbf{A}_{r} \hat{\mathbf{r}} \cdot \hat{\mathbf{r}} \\
&=\mathbf{A}_{r}-\mathbf{A}_{r}\\
&=0
\end{aligned}
$$

$$$$

\begin{mybox}{3.2.11}
Prove that a necessary and sufficient condition for the three (nonvanishing) vectors $\mathbf{A},$ $\mathbf{B},$ and $\mathbf{C}$ to be coplanar is the vanishing of the scalar triple product
$$
\mathbf{A} \cdot \mathbf{B} \times \mathbf{C}=0
$$
\end{mybox}
$\boxed{\textbf{Solution}}$ It should be keep in mind that scalar triple product can also be represent as the volume of
parallelepiped which is formed by three vectors. So we can say that if scalar triple product is equal to zero then vectors are coplanar as the parallelepipeds have no volume.

\newpage



\begin{mybox}{3.2.12}
Three vectors $\mathbf{A}, \mathbf{B},$ and $\mathbf{C}$ are given by
$$
\begin{array}{l}{\mathbf{A}=3 \hat{\mathbf{e}}_{x}-2 \hat{\mathbf{e}}_{y}+2 \hat{\mathbf{z}}} \\ {\mathbf{B}=6 \hat{\mathbf{e}}_{x}+4 \hat{\mathbf{e}}_{y}-2 \hat{\mathbf{z}}} \\ {\mathbf{C}=-3 \hat{\mathbf{e}}_{x}-2 \hat{\mathbf{e}}_{y}-4 \hat{\mathbf{z}}}\end{array}
$$
Compute the values of $\mathbf{A} \cdot \mathbf{B} \times \mathbf{C}$ and $\mathbf{A} \times(\mathbf{B} \times \mathbf{C}), \mathbf{C} \times(\mathbf{A} \times \mathbf{B})$ and $\mathbf{B} \times(\mathbf{C} \times \mathbf{A})$
\end{mybox}



$\boxed{\textbf{Solution}}$ First we can find $B\times C$ and then can permorm dot product

$$\mathbf{B} \times \mathbf{C}=\left|\begin{array}{ccc}\hat{x} & \hat{y} & \hat{z} \\ 6 & 4 & -2 \\ -3 & -2 & -4\end{array}\right|=\hat{x}(-16-4)-\hat{y}(-24-6)+\hat{z}(-12+12)=-20 \hat{x}+30 \hat{y}$$
$$\mathbf{A} \cdot(\mathbf{B} \times \mathbf{C})=(3 \hat{x}-2 \hat{y}+2 \hat{z}) \cdot(-20 \hat{x}+30 \hat{y})=-60-60=-120$$
With this, $\mathbf{A} \cdot (\mathbf{B} \times \mathbf{C})=-120$



$\boxed{\textbf{Solution}}$ For $(b)$ we have that the vector $\mathbf{A}$

$$\mathbf{A}=(3 \hat{x}-2 \hat{y}+2 \hat{z})$$
$$\mathbf{A} \times(\mathbf{B} \times \mathbf{C})=\left|\begin{array}{ccc}\hat{x} & \hat{y} & \hat{z} \\ 3 & -2 & 2 \\ -20 & 30 & 0\end{array}\right|=\hat{x}(-60)-\hat{y}(40)+\hat{z}(50)$$
With this, $$\mathbf{A} \times(\mathbf{B} \times \mathbf{C})=(-60) \hat{x}-(40) \hat{y}+(50) \hat{z}$$




$\boxed{\textbf{Solution}}$ For $(c)$ 
$$(\mathbf{A} \times \mathbf{B})=\left|\begin{array}{ccc}\hat{x} & \hat{y} & \hat{z} \\ 3 & -2 & 2 \\ 6 & 4 & -2\end{array}\right|=\hat{x}(-4)-\hat{y}(-18)+\hat{z}(24)$$
$$\mathbf{C} \times(\mathbf{A} \times \mathbf{B})=\left|\begin{array}{ccc}\hat{x} & \hat{y} & \hat{z} \\ -3 & -2 & -4 \\ -4 & 18 & 24\end{array}\right|=\hat{x}(26)-\hat{y}(-88)+\hat{z}(-62)$$




$\boxed{\textbf{Solution}}$ For $(d)$ 
$$(\mathbf{C} \times \mathbf{A})=\left|\begin{array}{ccc}\hat{x} & \hat{y} & \hat{z} \\ -3 & -2 & -4 \\ 3 & -2 & 2\end{array}\right|=\hat{x}(-12)-\hat{y}(6)+\hat{z}(12)$$
$$\mathbf{B} \times(\mathbf{C} \times \mathbf{A})=\left|\begin{array}{ccc}\hat{x} & \hat{y} & \hat{z} \\ 6 & 4 & -2 \\ -12 & -6 & 12\end{array}\right|=\hat{x}(36)-\hat{y}(48)+\hat{z}(12)$$

\newpage


\begin{mybox}{3.2.13}
Show that
$$
(\mathbf{A} \times \mathbf{B}) \cdot(\mathbf{C} \times \mathbf{D})=(\mathbf{A} \cdot \mathbf{C})(\mathbf{B} \cdot \mathbf{D})-(\mathbf{A} \cdot \mathbf{D})(\mathbf{B} \cdot \mathbf{C})
$$
\end{mybox}

$\boxed{\textbf{Solution}}$ Let $\mathbf{C} \times \mathbf{D}=m$.
Now, consider the scalar triple product $(\mathbf{A} \times \mathbf{B}) \cdot \mathbf{m}$.
since cross and dot product can be interchanged, we have,
$$
(\mathbf{A} \times \mathbf{B}) \cdot \mathbf{m}=\mathbf{A} \cdot(\mathbf{B} \times \mathbf{m})
$$
Resubstituting $m$ we get
$$\begin{aligned}(\mathbf{A} \times \mathbf{B}) \cdot(\mathbf{C} \times \mathbf{D}) &=\mathbf{A} \cdot[\mathbf{B} \times(\mathbf{C} \times \mathbf{D})] \\ &=\mathbf{A} \cdot[(\mathbf{B} \cdot \mathbf{D}) \mathbf{C}-(\mathbf{B} \cdot \mathbf{C})\mathbf{D}] \\ &=(\mathbf{A} \cdot \mathbf{C})(\mathbf{B} \cdot \mathbf{D})-(\mathbf{A} \cdot \mathbf{D})(\mathbf{B} \cdot \mathbf{C}) \end{aligned}$$
Thus, $(\mathbf{A} \times \mathbf{B}) \cdot(\mathbf{C} \times \mathbf{D})=(\mathbf{A} \cdot \mathbf{C})(\mathbf{B} \cdot \mathbf{D})-(\mathbf{A} \cdot \mathbf{D})(\mathbf{B} \cdot \mathbf{C})$

$$$$

\begin{mybox}{3.2.14}
Show that
$(\mathbf{A} \times \mathbf{B}) \times(\mathbf{C} \times \mathbf{D})=(\mathbf{A} \cdot \mathbf{B} \times \mathbf{D}) \mathbf{C}-(\mathbf{A} \cdot \mathbf{B} \times \mathbf{C}) \mathbf{D}$
\end{mybox}
$\boxed{\textbf{Solution}}$ Let $\mathbf{A} \times \mathbf{B}=\mathbf{m}$
$$(\mathbf{A} \times \mathbf{B}) \times(\mathbf{C} \times \mathbf{D})=\mathbf{m} \times(\mathbf{C} \times \mathbf{D})$$
$$=(\mathbf{m} \cdot \mathbf{D}) \mathbf{C}-(\mathbf{m} \cdot \mathbf{C})\mathbf{D}$$
$$=((\mathbf{A} \times \mathbf{B}) \cdot \mathbf{D}) \mathbf{C}-((\mathbf{A} \times \mathbf{B}) \cdot \mathbf{C})\mathbf{D}$$
$$=(\mathbf{A} \cdot(\mathbf{B} \times \mathbf{D})) \mathbf{C}-(\mathbf{A} \cdot(\mathbf{B} \times \mathbf{C}))\mathbf{D}$$


\newpage


\begin{mybox}{3.2.15}
An electric charge $q_{1}$ moving with velocity $\mathbf{v}_{1}$ produces a magnetic induction $\mathbf{B}$
given by
$$
\mathbf{B}=\frac{\mu_{0}}{4 \pi} q_{1} \frac{\mathbf{v}_{1} \times \hat{\mathbf{r}}}{r^{2}} \quad \text { (mks units), }
$$
where $\hat{\mathbf{r}}$ is a unit vector that points from $q_{1}$ to the point at which $\mathbf{B}$ is measured (Biot and Savart law).

\begin{enumerate}[$(a)$]
\item Show that the magnetic force exerted by $q_{1}$ on a second charge $q_{2},$ velocity $\mathbf{v}_{2},$ is
given by the vector triple product
$$
\mathbf{F}_{2}=\frac{\mu_{0}}{4 \pi} \frac{q_{1} q_{2}}{r^{2}} \mathbf{v}_{2} \times\left(\mathbf{v}_{1} \times \hat{\mathbf{r}}\right)
$$
\item Write out the corresponding magnetic force $\mathbf{F}_{1}$ that $q_{2}$ exerts on $q_{1} .$ Define your
unit radial vector. How do $\mathbf{F}_{1}$ and $\mathbf{F}_{2}$ compare?
\item Calculate $\mathbf{F}_{1}$ and $\mathbf{F}_{2}$ for the case of $q_{1}$ and $q_{2}$ moving along parallel trajectories side by side.
\end{enumerate}
\end{mybox}



$\boxed{\textbf{Solution}}$ For $(a)$ The magnetic force $\mathbf{F}_{2}$ is defined by the Lorentz force equation,
$$
\begin{aligned}
\mathbf{F}_{2} &=q_{2}\left(\mathbf{v}_{2} \times \mathbf{B}_{1}\right) \\
&=\frac{\mu_{0}}{4 \pi} q_{1} q_{2} \frac{\mathbf{v}_{2} \times\left(\mathbf{v}_{1} \times \hat{\mathbf{r}}\right)}{r^{2}}
\end{aligned}
$$
and 
$$
\mathbf{B}_{1}=\frac{\mu_{0}}{4 \pi} q_{1} \frac{\mathbf{v}_{1} \times \hat{\mathbf{r}}}{r^{2}}
$$

$\boxed{\textbf{Solution}}$ For $(b)$ The magnetic force $\mathbf{F}_{1}$ is defined by the Lorentz force equation,
$$
\begin{aligned}
\mathbf{F}_{1} &=q_{1}\left(\mathbf{v}_{1} \times \mathbf{B}_{2}\right) \\
&=-\frac{\mu_{0}}{4 \pi} q_{1} q_{2} \frac{\mathbf{v}_{1} \times\left(\mathbf{v}_{2} \times \hat{\mathbf{r}}\right)}{r^{2}}
\end{aligned}
$$
and
$$
\mathbf{B}_{2}=\frac{\mu_{0}}{4 \pi} q_{2} \frac{\mathbf{v}_{2} \times(-\hat{\mathbf{r}})}{r^{2}}
$$
From part we have, 
$$ \mathbf{F}_{2}=\frac{\mu_{0}}{4 \pi} q_{1} q_{2} \frac{\mathbf{v}_{2} \times\left(\mathbf{v}_{1} \times \hat{\mathbf{r}}\right)}{r^{2}}$$
since $-\mathbf{v}_{1} \times\left(\mathbf{v}_{2} \times \hat{\mathbf{r}}\right) \neq \mathbf{v}_{2} \times\left(\mathbf{v}_{1} \times \hat{\mathbf{r}}\right), \quad \mathbf{F}_{1} \neq \mathbf{F}_{2}$

$\vspace{1mm}$

$\boxed{\textbf{Solution}}$ For $(c)$ we have that
$$
\begin{aligned}
\mathbf{F}_{1} &=-\frac{\mu_{0}}{4 \pi} q_{1} q_{2} \frac{\mathbf{v} \times(\mathbf{v} \times \hat{\mathbf{r}})}{r^{2}} \\
&=-\frac{\mu_{0}}{4 \pi} q_{1} q_{2} \frac{\mathbf{v}(\mathbf{v} \cdot \hat{\mathbf{r}})-\hat{\mathbf{r}}(\mathbf{v} \cdot v)}{r^{2}} \\
&=-\frac{\mu_{0}}{4 \pi} q_{1} q_{2} \frac{0-\hat{\mathbf{r}}(\mathbf{v} \cdot v)}{r^{2}} \\
&=\frac{\mu_{0}}{4 \pi} q_{1} q_{2} \frac{\mathbf{v}^{2} \hat{\mathbf{r}}}{r^{2}}
\end{aligned}
$$
and
$$
\begin{aligned}
\mathbf{F}_{2} &=\frac{\mu_{0}}{4 \pi} q_{1} q_{2} \frac{\mathbf{v} \times(\mathbf{v} \times \hat{\mathbf{r}})}{r^{2}} \\
&=\frac{\mu_{0}}{4 \pi} q_{1} q_{2} \frac{\mathbf{v}(\mathbf{v} \cdot \hat{\mathbf{r}})-\hat{\mathbf{r}}(\mathbf{v} \cdot v)}{r^{2}} \\
&=\frac{\mu_{0}}{4 \pi} q_{1} q_{2} \frac{0-\hat{\mathbf{r}}(\mathbf{v} \cdot v)}{r^{2}} \\
&=-\frac{\mu_{0}}{4 \pi} q_{1} q_{2} \frac{\mathbf{v}^{2} \hat{\mathbf{r}}}{r^{2}}
\end{aligned}
$$
Thus, $\mathbf{F}_1 = -\mathbf{F}_2$


