\documentclass{article}
\usepackage[utf8]{inputenc}
\usepackage{latexsym}
\usepackage{amsmath}
\usepackage{amssymb}
\usepackage{bm}
\usepackage{multicol}
\usepackage{graphicx}
\usepackage{booktabs}
\usepackage{wrapfig}
\usepackage{fancybox}
\usepackage{bbm}
\usepackage{enumerate}
\pagestyle{plain}
\usepackage{yfonts}
\usepackage{ragged2e}


\usepackage{titling}
\setlength{\droptitle}{-7em} 


\def\Tiny{\fontsize{4pt}{4pt}\selectfont}
\newcommand*{\eqdef}{\ensuremath{\overset{\mathclap{\text{\Tiny def}}}{=}}}


\usepackage{geometry}
 \geometry{
 a4paper,
 total={160mm,257mm},
 left=24mm,
 top=20mm,
 }
 
 \usepackage{tcolorbox}
\newtcolorbox{mybox}[1]{colback=blue!5!white,colframe=blue!42!black,fonttitle=\bfseries,title=Problem #1}
 

\title{Title}
\author{Carlos Faz}
\date{ \ }

\begin{document}


\begin{flushleft}



\begin{mybox}{5.1.1}
A function $f(x)$ is expanded in a series of orthonormal functions
$$
f(x)=\sum_{n=0}^{\infty} a_{n} \varphi_{n}(x)
$$
Show that the series expansion is unique for a given set of $\varphi_{n}(x) .$ The functions $\varphi_{n}(x)$ are being taken here as the basis vectors in an infinite-dimensional Hilbert space.
\end{mybox}

$\boxed{\textbf{Solution}}$ Consider the Orthonormal function:
$$
f(x)=\sum_{n=0}^{\infty} a_{n} \phi_{n}(x)
$$
The objective is to show that the series expansion is unique for $\phi_{n}(x)$.
Here, the functions $\phi_{n}(x)$ are as the basis vectors in an infinite-dimensional Hilbert space.
If the functions $\phi_{i}$ are orthogonal and $$f=\sum_{i=1}^{n} a_{i} \phi_{i},$$ then the scalar $$a_{i}=\frac{\left\langle\phi_{i} \mid f\right\rangle}{\left\langle\phi_{i} \mid \phi_{i}\right\rangle}$$
Using the Orthogonality definition, the value of $\left\langle\phi_{n} \mid f\right\rangle$ is,
$$
\begin{aligned}
\left\langle\phi_{n} \mid f\right\rangle &=a_{n} \\
&=\int_{a}^{b} w(x) f(x) \phi_{n}(x) d x
\end{aligned}
$$
This is derived from the function $f$.
Assume that $\left\langle\phi_{n} \mid f\right\rangle=a_{n}'$
$$
\begin{aligned}
\left\langle\phi_{n} \mid f\right\rangle &=a_{n}^{\prime} \\
&=\int_{a}^{b} w^{\prime}(x) f(x) \phi_{n}(x) d x
\end{aligned}
$$
Then, $a_{n}=a_{n}{ }^{\prime}$ since $w(x)=w^{\prime}(x)$ Therefore, $\left\langle\phi_{n} \mid f\right\rangle=a_{n}$ is unique.


\newpage

\begin{mybox}{5.1.2}
A function $f(x)$ is represented by a finite set of basis functions $\varphi_{i}(x)$
$$
f(x)=\sum_{i=1}^{N} c_{i} \varphi_{i}(x)
$$
Show that the components $c_{i}$ are unique, that no different set $c_{i}^{\prime}$ exists.
Note. Your basis functions are automatically linearly independent. They are not necessarily orthogonal.
\end{mybox}
$\boxed{\textbf{Solution}}$ Consider the function:
$$f(x)=\sum_{i=1}^{N} c_{i} \phi_{i}(x)$$
The objective is to show that the components $c_{i}$ are unique.
The function can be written as,
$$
\begin{aligned}
f(x) &=\sum_{i} c_{i} \phi_{i}(x) \\
&=\sum_{j} c_{j}^{\prime} \phi_{j}(x)
\end{aligned}
$$
Then,
$$\sum_{i}\left(c_{i}-c_{i}^{\prime}\right) \phi_{i}=\sum_{i} c_{i} \phi_{i}-\sum_{i} c_{i}^{\prime} \phi_{i}$$
$$=\sum_{i} c_{i} \phi_{i}-\sum_{i} c_{i} \phi_{i}$$
$$=0$$
Assume $c_{m}-c_{m}^{\prime} \neq 0$ Then,
$$
\phi_{m}=\frac{-1}{c_{m}-c_{m}} \sum_{b=m}\left(c_{i}-c_{i}^{\prime}\right) \phi_{i}
$$
It confirms that, $\phi_{m}$ is not linearly independent of the $\phi_{i}$, which is a contradiction to our assumption. So, $c_{m}-c_{m}^{\prime}=0$ Therefore, the scalars $c_{i}$ are unique.

\newpage

\begin{mybox}{5.1.3}
A function $f(x)$ is approximated by a power series $\sum_{i=0}^{n-1} c_{i} x^{i}$ over the interval [0,1] Show that minimizing the mean square error leads to a set of linear equations
$$
\mathrm{Ac}=\mathbf{b}
$$
where
$$
A_{i j}=\int_{0}^{1} x^{i+j} d x=\frac{1}{i+j+1}, \quad i, j=0,1,2, \ldots, n-1
$$
and
$$
b_{i}=\int_{0}^{1} x^{i} f(x) d x, \quad i=0,1,2, \ldots, n-1
$$
Note. The $A_{i j}$ are the elements of the Hilbert matrix of order $n$. The determinant of this Hilbert matrix is a rapidly decreasing function of $n .$ For $n=5,$ det $A=3.7 \times 10^{-12}$ and the set of equations $\mathrm{A} c=\mathbf{b}$ is becoming ill-conditioned and unstable.
\end{mybox}
$\boxed{\textbf{Solution}}$ For 
$$f(x)=\sum_{i=0}^{n-1} c_{i} x^{i}$$
we have
$$
\begin{aligned}
b_{j} &=\int_{0}^{1} x^{j} f(x) d x, \quad j=0,1,2, \ldots, n-1 \\
&=\sum_{l} c_{i} \int_{0}^{1} x^{i+j} d x \\
&=\sum_{i=0}^{n-1} \frac{c_{i}}{i+j+1} \\
&=A_{j i} c_{i}
\end{aligned}
$$
This result also minimizing the mean square error $$\int_{0}^{1}\left[f(x)-\sum_{i=0}^{n-1} c_{i} x^{i}\right]^{2} d x$$ 
upon varying the $c_{i}$

\newpage

\begin{mybox}{5.1.4}
In place of the expansion of a function $F(x)$ given by
$$
F(x)=\sum_{n=0}^{\infty} a_{n} \varphi_{n}(x)
$$
with
$$
a_{n}=\int_{a}^{b} F(x) \varphi_{n}(x) w(x) d x
$$
take the finite series approximation
$$
F(x) \approx \sum_{n=0}^{m} c_{n} \varphi_{n}(x)
$$
Show that the mean square error
$$
\int_{a}^{b}\left[F(x)-\sum_{n=0}^{m} c_{n} \varphi_{n}(x)\right]^{2} w(x) d x
$$
is minimized by taking $c_{n}=a_{n}$

Note. The values of the coefficients are independent of the number of terms in the finite series. This independence is a consequence of orthogonality and would not hold for a least-squares fit using powers of $x$.
\end{mybox}




$\boxed{\textbf{Solution}}$ Consider the function 
$$
F(x)=\sum_{n=0}^{\infty} a_{n} \phi_{n}(x)
$$
Here, $$a_{n}=\int_{a}^{b} F(x) \phi_{n}(x) w(x) d x$$ and $$F(x) \approx \sum_{n=0}^{m} c_{n} \phi_{n}(x)$$
The objective is to show the mean square error is minimized when $c_{n}=a_{n}$.
For $$F(x)=\sum_{n=0}^{m} a_{n} \phi_{n}(x),$$ we have
$$\begin{aligned} c_{j} &=\int_{0}^{1} x^{j} f(x) d x, j=0,1,2, \ldots, m \\ &=\sum_{i} a_{i} \int_{0}^{1} x^{j+j} d x \\ &=\sum_{i=0}^{m} \frac{a_{i}}{i+j+1} \\ &=A_{j i} a_{i} \end{aligned}$$
Note that $A_{i j}$ 's represents the elements of the Hilbert matrix of order $n$.
The determinant of this Hilbert matrix is a decreasing function of $n$.
Write the function as
$$
F(x)=\sum_{n=0}^{m} c_{n} \phi_{n}(x)
$$
$$
F(x)-\sum_{n=0}^{m} c_{n} \phi_{n}(x)=0
$$
$$
\int_{a}^{b}\left[F(x)-\sum_{n=0}^{m} c_{n} \phi_{n}(x)\right]^{2} w(x) d x=0
$$
$$
\frac{\partial}{\partial c_{l}} \int_{a}^{b}\left[F(x)-\sum_{n=0}^{m} c_{n} \phi_{n}(x)\right]^{2} w(x) d x=0
$$
Remember that
$$
c_{n}=\int_{a}^{b} F(x) \phi_{n}(x) w(x) d x
$$
This result is also minimizing the mean square error $$\int_{a}^{b}\left[F(x)-\sum_{n=0}^{m} c_{n} \phi_{n}(x)\right]^{2} w(x) d x$$ is
minimized when $c_{n}=a_{n}$

\newpage

\begin{mybox}{5.1.5}
The functions $\cos n x(n=0,1,2, \ldots)$ and $\sin n x(n=1,2, \ldots)$ have (together) been shown to form a complete set on the interval $-\pi<x<\pi .$ since this determination is obtained subject to convergence in the mean, there is the possibility of deviation at isolated points, thereby permitting the description of functions with isolated discontinuities.

$$
f(x)=\left\{\begin{array}{lr}
\frac{h}{2}, & 0<x<\pi \\
-\frac{h}{2}, & -\pi<x<0
\end{array}\right\}=\frac{2 h}{\pi} \sum_{n=0}^{\infty} \frac{\sin (2 n+1) x}{2 n+1}
$$
\begin{enumerate}[$a)$]
\item Show that
$$
\int_{-\pi}^{\pi}[f(x)]^{2} d x=\frac{\pi}{2} h^{2}=\frac{4 h^{2}}{\pi} \sum_{n=0}^{\infty}(2 n+1)^{-2}
$$
For a finite upper limit this would be Bessel's inequality. For the upper limit $\infty$, this is Parseval's identity.
\item Verify that
$$
\frac{\pi}{2} h^{2}=\frac{4 h^{2}}{\pi} \sum_{n=0}^{\infty}(2 n+1)^{-2}
$$
by evaluating the series. Hint. The series can be expressed in terms of the Riemann zeta function $\zeta(2)=\pi^{2} / 6$
\end{enumerate}
\end{mybox}




$\boxed{\textbf{Solution}}$ The objective is to show that 
$$
\int_{-\pi}^{\pi}[f(x)]^{2} d x=\frac{\pi}{2} h^{2}=\frac{4 h^{2}}{\pi} \sum_{n=0}^{\infty}(2 n+1)^{-2}
$$
First, we start saying that the integral $\int_{-\pi}^{\pi}[f(x)]^{2} d x$ can be evaluated as
$$
\begin{aligned}
\int_{-\pi}^{\pi}[f(x)]^{2} d x &=\int_{-\pi}^{\pi} f(x) \cdot f(x) d x \\
&=\int_{-\pi}^{\pi} f(x) d x \cdot \int_{-\pi}^{\pi} f(x) d x \\
&=\int_{-\pi}^{\pi} \frac{2 h}{\pi} \sum_{n=0}^{\infty} \frac{\sin (2 n+1) x}{2 n+1} d x \int_{-\pi}^{\pi} \frac{2 h}{\pi} \sum_{m=0}^{\infty} \frac{\sin (2 m+1) x}{2 m+1} \\
&=\left(\frac{4 h^{2}}{\pi^{2}}\right) \sum_{m, n=0}^{\infty} \frac{1}{(2 n+1)(2 m+1)} \times \int_{-n}^{\pi} \sin [(2 n+1) x] \sin [(2 m+1) x] d x
\end{aligned}
$$
$$=\frac{4 h^{2}}{\pi^{2}} \sum_{n=0}^{\infty} \frac{1}{(2 n+1)^{2}} \int_{-\pi}^{\pi} \sin ^{2}[(2 n+1) x] d x$$
$$=\frac{4 h^{2}}{\pi^{2}}\left(1+\frac{1}{3^{2}}+\frac{1}{5^{2}}+\frac{1}{7^{2}}+\cdots\right) \int_{-\pi}^{\pi}\left(\frac{1-\cos (2(2 n+1) x)}{2}\right) d x$$
$$
=\frac{4 h^{2}}{\pi^{2}}\left(\frac{\pi^{2}}{8}\right)\left(\frac{x-\frac{\sin (2(2 n+1) x)}{2(2 n+1)}}{2}\right)_{-\pi}^{\pi}
$$

$$
=\frac{4 h^{2}}{\pi^{2}}\left(\frac{\pi^{2}}{8}\right)\left(\frac{\pi-\frac{\sin (2(2 n+1) \pi)}{2(2 n+1)}}{2}-\left(\frac{(-\pi)-\frac{\sin (2(2 n+1)(-\pi))}{2(2 n+1)}}{2}\right)\right)
$$

$$=\frac{4 h^{2}}{\pi^{2}}\left(\frac{\pi^{2}}{8}\right)\left(\frac{\pi}{2}+\frac{\pi}{2}\right)$$
$$=\frac{4 h^{2}}{\pi^{2}}\left(\frac{\pi^{2}}{8}\right)(\pi)$$
$$=\frac{h^{2} \pi}{2}$$


Therefore, $\int_{-\pi}^{\pi}[f(x)]^{2} d x=\frac{\pi}{2} h^{2}$

$$
\int_{-\pi}^{\pi}[f(x)]^{2} d x=\frac{4 h^{2}}{\pi^{2}} \sum_{n=0}^{\infty} \frac{1}{(2 n+1)^{2}} \int_{-\pi}^{\pi} \sin ^{2}[(2 n+1) x] d x
$$

$$=\frac{4 h^{2}}{\pi^{2}} \sum_{n=0}^{\infty}(2 n+1)^{-2}(\pi)$$
$$=\frac{4 h^{2}}{\pi} \sum_{n=0}^{\infty}(2 n+1)^{-2}$$

Hence, $$\int_{-\pi}^{\pi}[f(x)]^{2} d x=\frac{\pi}{2} h^{2}=\frac{4 h^{2}}{\pi} \sum_{n=0}^{\infty}(2 n+1)^{-2}$$


For $(b)$ $$
\begin{aligned}
\mathrm{RHS} &=\frac{4 h^{2}}{\pi}\left(\sum_{n=0}^{\infty} \frac{1}{(2 n+1)^{2}}\right) \\
&=\frac{4 h^{2}}{\pi}\left(1+\frac{1}{3^{2}}+\frac{1}{5^{2}}+\frac{1}{7^{2}}+\cdots\right) \\
&=\frac{4 h^{2}}{\pi}\left(\frac{\pi^{2}}{8}\right) \\
&=\frac{\pi h^{2}}{2}
\end{aligned}
$$

\newpage


\begin{mybox}{5.1.6}
Derive the Schwarz inequality from the identity
$$
\begin{aligned}
\left[\int_{a}^{b} f(x) g(x) d x\right]^{2}=& \int_{a}^{b}[f(x)]^{2} d x \int_{a}^{b}\left[[g(x)]^{2} d x\right.\\
&-\frac{1}{2} \int_{a}^{b} d x \int_{a}^{b} d y[f(x) g(y)-f(y) g(x)]^{2}
\end{aligned}
$$
\end{mybox}

$\boxed{\textbf{Solution}}$ The double integral can be written as,
$$
\begin{aligned}
\left[\int_{a}^{b} f(x) g(x) d x\right]^{2}=& \int_{a}^{b}[f(x)]^{2} d x \int_{a}^{b}[g(x)]^{2} d x \\
&-\frac{1}{2} \int_{a}^{b} d x \int_{a}^{b} d y[f(x) g(y)-f(y) g(x)]^{2}
\end{aligned}
$$

$$
\begin{aligned}
|\langle f \mid g\rangle|^{2} &=\langle f\rangle^{2}\langle g\rangle^{2}-\frac{1}{2} \int_{a}^{b} \int_{a}^{b}[f(x) g(y)-f(y) g(x)]^{2} \\
& \leq\langle f\rangle^{2}\langle g\rangle^{2} \\
|\langle f \mid g\rangle|^{2} & \leq\langle f \mid f\rangle\langle g \mid g\rangle
\end{aligned}
$$
since the double integral is non-negative, so $\left.\langle f \mid g\rangle\right|^{2} \geq 0$.
Hence, the result of Schwarz inequality is derived.

\newpage



\begin{mybox}{5.1.7}
Starting from 
$$I=\left\langle f-\sum_{i} a_{i} \varphi_{i} \mid f-\sum_{j} a_{j} \varphi_{j}\right\rangle \geq 0$$
derive Bessel's inequality, 
$$\langle f \mid f\rangle \geq \sum_{n}\left|a_{n}\right|^{2}$$
\end{mybox}

$\boxed{\textbf{Solution}}$ The functions $\phi_{j}$ are assumed to be orthonormal.
Expand the value of $I,$ we have
$$\begin{aligned} I &=\left\langle f-\sum_{i} a_{i} \phi_{i} \mid f-\sum_{j} a_{j} \phi_{j}\right\rangle \\ &=\langle f \mid f\rangle-\sum_{i} a_{i} *\left\langle\phi_{i} \mid f\right\rangle-\sum_{i} a_{i} *\left\langle f \mid \phi_{i}\right\rangle+\sum_{i} a_{i} * a_{j}\left\langle\phi_{i} \mid \phi_{j}\right\rangle \\ & \geq 0 \end{aligned}$$
Hence, the result of Bessel's inequality is derived.

\newpage

\begin{mybox}{5.1.8}
Expand the function $\sin \pi x$ in a series of functions $\varphi_{i}$ that are orthogonal (but not normalized) on the range $0 \leq x \leq 1$ when the scalar product has definition
$$
\langle f \mid g\rangle=\int_{0}^{1} f^{*}(x) g(x) d x
$$
Keep the first four terms of the expansion. The first four $\varphi_{i}$ are:
$$
\varphi_{0}=1, \quad \varphi_{1}=2 x-1, \quad \varphi_{2}=6 x^{2}-6 x+1, \quad \varphi_{3}=20 x^{3}-30 x^{2}+12 x-1
$$
Note. The integrals that are needed are the subject of Example 1.10 .5 .
\end{mybox}

$\boxed{\textbf{Solution}}$ Consider the function:
$\sin (\pi x)$
Expand the function $\sin (\pi x)$ in a series of functions $\phi_{i}$ which are orthogonal. Write the function $\sin (\pi x)$ in a series of functions $\phi_{i}$ as,
$$
\sin (\pi x)=\sum_{i} \frac{\left\langle\phi_{i} \mid \sin \pi x\right\rangle}{\left\langle\phi_{i}, \phi_{i}\right\rangle} \phi_{i}(x)
$$
Here, $\phi_{0}=1, \phi_{1}=2 x-1, \phi_{2}=6 x^{2}-6 x+1, \phi_{3}=20 x^{3}-30 x^{2}+12 x-1$
The integrals are calculated as,
$$
\begin{aligned}
\left\langle\phi_{0} \mid \phi_{0}\right\rangle &=\int_{0}^{1} d x \\
&=(x)_{0}^{1} \\
&=1
\end{aligned}
$$
$$
\left\langle\phi_{1} \mid \phi_{1}\right\rangle=\int_{0}^{1}(2 x-1)^{2} d x
$$
$$\left\langle\phi_{1} \mid \phi_{1}\right\rangle=\int_{0}^{1}\left(4 x^{2}-4 x+1\right) d x$$
$$\left\langle\phi_{1} \mid \phi_{1}\right\rangle=\left(\frac{4 x^{3}}{3}-2 x^{2}+x\right)_{0}^{1}$$
$$\left\langle\phi_{1} \mid \phi_{1}\right\rangle=\left(\frac{4}{3}-2+1\right)$$
$$\left\langle\phi_{1} \mid \phi_{1}\right\rangle=\frac{1}{3}$$
$$
\begin{aligned}
\left\langle\phi_{2} \mid \phi_{2}\right\rangle &=\int_{0}^{1}\left(6 x^{2}-6 x+1\right)^{2} d x \\
&=\int_{0}^{1}\left(36 x^{4}-72 x^{3}+48 x^{2}-12 x+1\right) d x \\
&=\left(\frac{36 x^{5}}{5}-18 x^{4}+16 x^{3}-6 x^{2}+x\right)_{0}^{1} \\
&=\frac{36}{5}-18+16-6+1 \\
&=\frac{1}{5}
\end{aligned}
$$
$$
\begin{aligned}
\left\langle\phi_{3} \mid \phi_{3}\right\rangle &=\int_{0}^{1}\left(20 x^{3}-30 x^{2}+12 x-1\right)^{2} d x \\
&=\int_{0}^{1}\left(400 x^{6}-1200 x^{5}+1380 x^{4}-760 x^{3}+204 x^{2}-24 x+1\right) d x \\
&=\left(\frac{400 x^{7}}{7}-200 x^{6}+276 x^{5}-190 x^{4}+68 x^{3}-12 x^{2}+x\right)_{0}^{1} \\
&=\frac{400}{7}-200+276-190+68-12+1 \\
&=\frac{1}{7}
\end{aligned}
$$
$$
\begin{aligned}
\left\langle\phi_{0} \mid f\right\rangle &=\int_{0}^{1} \sin \pi x d x \\
&=\left(\frac{-\cos \pi x}{\pi}\right)_{0}^{1} \\
&=-\left(\frac{\cos \pi(1)}{\pi}-\frac{\cos \pi(0)}{\pi}\right) \\
&=-\left(\frac{-1}{\pi}-\frac{1}{\pi}\right) \\
&=\frac{2}{\pi}
\end{aligned}
$$
The value of $\left\langle\phi_{1} \mid f\right\rangle$ is,
$$\langle\phi \mid f\rangle=\int_{0}^{1}(2 x-1) \sin (\pi x) d x$$
$$\quad=\left(\frac{2 \sin (\pi x)+(\pi-2 \pi x) \cos (\pi x)}{\pi^{2}}\right)_{0}^{1}$$
Using $\int_{0}^{1}(2 x-1) \sin (\pi x) d x=\frac{2 \sin (\pi x)+(\pi-2 \pi x) \cos (\pi x)}{\pi^{2}}$
$$=\frac{2 \sin (\pi \cdot 1)+(\pi-2 \pi \cdot 1) \cos (\pi \cdot 1)}{\pi^{2}}-$$
$$=\frac{2 \sin (\pi \cdot 0)+(\pi-2 \pi \cdot 0) \cos (\pi \cdot 0)}{\pi^{2}}$$
$$=\frac{2(0)+(-\pi) \cdot 1}{\pi^{2}}-\left(\frac{2(0)+(-\pi) 1}{\pi^{2}}\right)$$
$$=0$$
$$
\left\langle\varphi_{2} \mid f\right\rangle=\frac{2}{\pi}-\frac{24}{\pi^{3}}
$$
$$
\left\langle\varphi_{3} \mid f\right\rangle=0
$$
$$
\sin \pi x=\frac{2 / \pi}{1} \varphi_{0}+\frac{2 / \pi-24 / \pi^{3}}{1 / 5} \varphi_{2}+\cdots
$$
$$\sin(\pi x) =0.6366-0.6871\left(6 x^{2}-6 x+1\right)+\cdots $$

\newpage

\begin{mybox}{5.1.9}
Expand the function $e^{-x}$ in Laguerre polynomials $L_{n}(x),$ which are orthonormal on the range $0 \leq x<\infty$ with scalar product
$$
\langle f \mid g\rangle=\int_{0}^{\infty} f^{*}(x) g(x) e^{-x} d x
$$
Keep the first four terms of the expansion. The first four $L_{n}(x)$ are
$$
L_{0}=1, \quad L_{1}=1-x, \quad L_{2}=\frac{2-4 x+x^{2}}{2}, \quad L_{3}=\frac{6-18 x+9 x^{2}-x^{3}}{6}
$$
\end{mybox}

$\boxed{\textbf{Solution}}$ The value of $a_0$ is
$$
\begin{aligned}
a_{0} &=\int_{0}^{\infty} L_{0}(x) e^{-2 x} d x \\
&=\int_{0}^{\infty} e^{-2 x} d x \\
&=\left(\frac{e^{-2 x}}{-2}\right)_{0}^{\infty} \\
&=\frac{-1}{2}\left(e^{-2(\alpha)}-e^{0}\right) \\
=& \frac{-1}{2}(0-1) \\
=& \frac{1}{2}
\end{aligned}
$$
The value of $a_1$ is
$$
\begin{aligned}
a_{1} &=\int_{0}^{\infty} L_{1}(x) e^{-2 x} d x \\
&=\int_{0}^{\infty}(1-x) e^{-2 x} d x \\
&=\left(\frac{1}{4} e^{-2 x}(2 x-1)\right)_{0}^{\infty}
\end{aligned}
$$
$$
=\frac{1}{4}\left(e^{-2(\infty)}(2(\infty)-1)-e^{0}(2(0)-1)\right)
$$
$$=\frac{1}{4}(0+1)$$
$$=\frac{1}{4}$$
The value of $a_2$ is
$$
\begin{aligned}
a_{2} &=\int_{0}^{\infty} L_{2}(x) e^{-2 x} d x \\
&=\int_{0}^{\infty}\left(\frac{2-4 x+x^{2}}{2}\right) e^{-2 x} d x \\
&=\left(\frac{-1}{8} e^{-2 x}\left(1-6 x+2 x^{2}\right)\right)_{0}^{\infty}
\end{aligned}
$$
The value of $a_{3}$ is,
$$
\begin{aligned}
a_{3} &=\int_{0}^{\infty} L_{3}(x) e^{-2 x} d x \\
&=\int_{0}^{\infty}\left(\frac{6-18 x+9 x^{2}-x^{3}}{6}\right) e^{-2 x} d x
\end{aligned}
$$
$$
=\left(\frac{1}{48} e^{-2 x}\left(4 x^{3}-30 x^{2}+42 x-3\right)\right)_{0}^{\infty}
$$
$$=\frac{3}{48}$$
$$=\frac{1}{16}$$
Thus, the expansion of $e^{-x}$ is
$$
e^{-x}=a_{0} L_{0}(x)+a_{1} L_{1}(x)+a_{2} L_{2}(x)+a_{3} L_{3}(x)+\cdots
$$
$$
=\frac{1}{2}(1)+\frac{1}{4}(1-x)+\frac{1}{8}\left(\frac{2-4 x+x^{2}}{2}\right)+\frac{1}{16}\left(\frac{6-18 x+9 x^{2}-x^{3}}{6}\right)+\cdots
$$

\newpage

\begin{mybox}{5.1.10}
The explicit form of a function $f$ is not known, but the coefficients $a_{n}$ of its expansion in the orthonormal set $\varphi_{n}$ are available. Assuming that the $\varphi_{n}$ and the members of another orthonormal set, $\chi_{n},$ are available, use Dirac notation to obtain a formula for the coefficients for the expansion of $f$ in the $\chi_{n}$ set.
\end{mybox}

$\boxed{\textbf{Solution}}$ The coefficients of $f$ in the $\phi$ basis are $a_{i}=\left\langle\phi_{i} \mid f\right\rangle,$ so the above equation is equivalent to,
$$
f=\sum_{j} b_{j} \chi_{j}
$$
Here, $b_{j}=\sum_{j}\left\langle\chi_{j} \mid \phi_{i}\right\rangle a_{i}$

\newpage

\begin{mybox}{5.1.11}
Using conventional vector notation, evaluate $\sum_{j}\left|\hat{\mathbf{e}}_{j}\right\rangle\left\langle\hat{\mathbf{e}}_{j} \mid \mathbf{a}\right\rangle,$ where a is an arbitrary vector in the space spanned by the $\hat{\mathbf{e}}_{j}$
\end{mybox}

$\boxed{\textbf{Solution}}$  We assume the unit vectors are orthogonal. Then,
$$
\sum_{j}\left|\hat{\mathbf{e}}_{j}\right\rangle\left\langle\hat{\mathbf{e}}_{j} \mid \mathbf{a}\right\rangle=\sum_{j}\left(\hat{\mathbf{e}}_{j} \cdot \mathbf{a}\right) \hat{\mathbf{e}}_{j}
$$
This expression is a component decomposition of a.

\newpage

\begin{mybox}{5.1.12}
Letting a $=a_{1} \hat{\mathrm{e}}_{1}+a_{2} \hat{\mathrm{e}}_{2}$ and $\mathbf{b}=b_{1} \hat{\mathrm{e}}_{1}+b_{2} \hat{\mathrm{e}}_{2}$ be vectors in $\mathbb{R}^{2},$ for what values of $k,$ if
any, is
$$
\langle\mathbf{a} \mid \mathbf{b}\rangle=a_{1} b_{1}-a_{1} b_{2}-a_{2} b_{1}+k a_{2} b_{2}
$$
a valid definition of a scalar product?
\end{mybox}

$\boxed{\textbf{Solution}}$ Consider the two vectors:
$$\mathbf{a}=a_{1} \mathbf{e}_{1}+a_{2} \mathbf{e}_{2}$$ 
and 
$$b=b_{1} \mathbf{e}_{1}+b_{2} \mathbf{e}_{2}$$
The objective is to for what values of $k$ the scalar product 
$$\langle\mathbf{a} \mid \mathbf{b}\rangle=a_{1} b_{1}-a_{1} b_{2}-a_{2} b_{1}+k a_{2} b_{2}$$
is valid. The scalar product $\langle\mathbf{a} \mid \mathbf{a}\rangle$ must be positive for every non-zero vector in the space. If we write
$\langle\mathbf{a} \mid \mathbf{a}\rangle$ in the form,
$$
\begin{aligned}
\langle\mathbf{a} \mid \mathbf{a}\rangle &=a_{1} a_{1}-a_{1} a_{2}-a_{2} a_{1}+k a_{2} a_{2} \\
&=a_{1}^{2}-2 a_{1} a_{2}+k a_{2}^{2} \\
&=\left(a_{1}^{2}-2 a_{1} a_{2}+a_{2}^{2}\right)-a_{2}^{2}+k a_{2}^{2} \\
&=\left(a_{1}-a_{2}\right)^{2}-a_{2}^{2}+k a_{2}^{2}
\end{aligned}
$$
$$
=\left(a_{1}-a_{2}\right)^{2}+(k-1) a_{2}^{2}
$$
This condition is violated for some non-zero vector a unless $k>1$.
Therefore, the scalar product is valid when $k>1$.






\end{flushleft}
\end{document}
