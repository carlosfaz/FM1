

$\vspace{7cm}$

 
 
 % Suppresses headers and footers on the title page

	{\centering % Centre everything on the title page
	
	 % Use small caps for all text on the title page
	
	\vspace*{\baselineskip} % White space at the top of the page
	
	%------------------------------------------------
	%	Title
	%------------------------------------------------
	
	\rule{\textwidth}{1.6pt}\vspace*{-\baselineskip}\vspace*{2pt} % Thick horizontal rule
	\rule{\textwidth}{0.4pt} % Thin horizontal rule
	
	\vspace{0.75\baselineskip} % Whitespace above the title
	
	{\LARGE Chapter 1 \\ Mathematical \\ Preliminaries \\} % Title
	
	\vspace{0.75\baselineskip} % Whitespace below the title
	
	\rule{\textwidth}{0.4pt}\vspace*{-\baselineskip}\vspace{3.2pt} % Thin horizontal rule
	\rule{\textwidth}{1.6pt} % Thick horizontal rule
	
	\vspace{2\baselineskip} % Whitespace after the title block
	
	%------------------------------------------------
	%	Subtitle
	%------------------------------------------------
	

	
	\vspace*{3\baselineskip} % Whitespace under the subtitle
	

}


\newpage

\begin{center}
{\Large\text{Chapter 1.1: Infinite Series}}
\end{center}



\begin{mybox}{1.1.1}
\begin{enumerate}[$(a)$]
\item Prove that if $\lim _{n \rightarrow \infty} n^{p} u_{n}=A<\infty, p>1,$ the series $\sum_{n=1}^{\infty} u_{n}$ converges.
\item Prove that if $\lim _{n \rightarrow \infty} n u_{n}=A>0,$ the series diverges. (The test fails for $\left.A=0 .\right)$ These two tests, known as limit tests, are often convenient for establishing the convergence of a series. They may be treated as comparison tests, comparing with
$$
\sum_{n} n^{-q}, \quad 1 \leq q<p
$$
\end{enumerate}
\end{mybox}


$\boxed{\textbf{Solution}}$ For $(a)$ Let 
$$\lim _{n \rightarrow \infty} n^{p} u_{n}=A<\infty, \quad p>1$$
Define 
$$
u_{n}=\frac{1}{n^{p}}, \quad  p>1
$$
be a series. Let 
$$
\lim _{m \rightarrow \infty} n^{p} u_{n}=A<\infty, \quad p>1
$$
then $\sum_{n=1}^{\infty} u_{n}$ is convergent by limit comparision test as 
$$
\lim _{n \rightarrow \infty} \frac{u_{n}}{u_{n}}=\lim _{n \rightarrow \infty} \frac{u_{n}}{1 / n^{p}}
$$
$$
=\lim _{n \rightarrow \infty} n^{p} u_{n}=A<\infty
$$
If $A\neq 0$ both, the series $\sum_{n=1}^{\infty} u_{n}$ and $\sum_{n=1}^{\infty} u_{n}$ behave alike as 
$$\sum_{n=1}^{\infty} u_{n} = \sum_{n=1}^{\infty} \frac{1}{n^{p}}$$ is convergent series. If $A\neq 0$ then by limit comparision test, if $
\sum_{n=1}^{\infty} u_{n}$ is convergent then $$\sum_{n=1}^{\infty} \frac{1}{n^p}, \quad p>1$$
is convergent by $p-$test. Therefore $\sum_{n=1}^{\infty} u_{n}$ is convergent series. 


$\vspace{3mm}$


$\boxed{\textbf{Solution}}$ For $(b)$ let $\lim _{n \rightarrow \infty} n u_{n}=A>0$ and define
$$u_n = \frac{1}{n}$$
then $\sum u_n$ is divergent series by limit comparision test. Because  
$$
\lim _{n \rightarrow \infty} \frac{u _n}{u_n}=\lim _{n \rightarrow \infty} \frac{u_n}{\frac{1}{n}}=\lim _{n \rightarrow \infty} n u_{n}=A \neq 0
$$
then by limit comparision test, if $A\neq 0$ and finite then $\sum_{n=1}^{\infty}u_n, \sum_{n=1}^{\infty} u_n$ both behave alike since 
$$
\sum_{n=1}^{\infty} u_{n}=\sum_{m=1}^{\infty} \frac{1}{n}
$$
is a divergent series by $p-$test, $\sum_{n=1}^{\infty} u_{n}$ is divergent series. If $A\neq 0$ and infinite, then $\sum_{n=1}^{\infty} u_n$ is divergent and $\sum_{n=1}^{\infty} u_n$ is also divergent.


\newpage






\begin{mybox}{1.1.2}
If $\displaystyle \lim _{n \rightarrow \infty} \frac{b_{n}}{a_{n}}=K,$ a constant with $0<K<\infty,$ show that $\Sigma_{n} b_{n}$ converges or diverges with $\Sigma a_{n}$

Hint. If $\Sigma a_{n}$ converges, rescale $b_{n}$ to $b_{n}^{\prime}=\dfrac{b_{n}}{2 K} .$ If $\Sigma_{n} a_{n}$ diverges, rescale to $b_{n}^{\prime \prime}=\dfrac{2 b_{n}}{K}$
\end{mybox}


$\boxed{\textbf{Solution}}$ The objective is to prove that the given series converges or diverges with $\sum_{n=1}^{\infty} a_{n}$
It is given that 
$$\lim_{n\rightarrow \infty} \frac{b_{n}}{a_{n}}=K, \quad 0<K<\infty$$
By the definition of limit of a sequence, for any given $\varepsilon>0$ there exists a positive integer $N$ such that for all 
$$n \geq N,\left|\frac{b_{n}}{a_{n}}-K\right|<\varepsilon$$
This implies, 
$$-\varepsilon<\frac{b_{n}}{a_{n}}-K<\varepsilon$$ for all values of $n \geq N$.
From this the inequality implies, 
$$K-\varepsilon<\frac{b_{n}}{a_{n}}<K+\varepsilon$$
for all $n \geq N$. From the above inequality, 
$$(K-\varepsilon) a_{n}<b_{n}<(K+\varepsilon) a_{n}$$ 
for all $n \geq N$. So, $(K-\varepsilon) a_{n}<b_{n}$ and $b_{n}<(K+\varepsilon) a_{n}$ for all $n \geq N$. Suppose that $\sum_{n=1}^{\infty} a_{n}$ converges. Then, 
$$\sum_{n=1}^{\infty}(K+\varepsilon) a_{n}$$ 
also converges. By the equation above, $b_{n}<(K+\varepsilon) a_{n}$ for all $n \geq N$
By comparison test the series $\sum_{n=1}^{\infty} b_{n}$ also converges. Suppose that $\sum_{n=1}^{\infty} a_{n}$ diverges. Then 
$$\sum_{n=1}^{\infty}(K-\varepsilon) a_{n}$$ 
also diverges. By the above equation 
$$(K-\varepsilon) a_{n}<b_{n}$$
for all $n \geq N$. By the condition $(K-\varepsilon) a_{n}<b_{n}$ for all $n \geq N$. Hence, by comparison test the series $\sum_{n=1}^{\infty} b_{n}$ also diverges.


\newpage



\begin{mybox}{1.1.3}
\begin{enumerate}[$(a)$]
\item Show that the series 
$$\sum_{n=2}^{\infty} \frac{1}{n(\ln n)^{2}}$$
converges.
\item By direct addition $$\sum_{n=2}^{100,000}\left[n(\ln n)^{2}\right]^{-1}=2.02288$$ Use Eq. (1.9) to make a fivesignificant-figure estimate of the sum of this series.	
$$
\int_{N+1}^{\infty} f(x) d x \leq \sum_{n=N+1}^{\infty} a_{n} \leq \int_{N+1}^{\infty} f(x) d x+a_{N+1} \quad (1.9)
$$
\end{enumerate}
\end{mybox}

$\boxed{\textbf{Solution}}$ For $(a)$ we check the convergence it is required to use the integral test. Put 
$$f(x)=\frac{1}{x(\ln x)^{2}} \Rightarrow f(n)=\frac{1}{n(\ln n)^{2}}$$
Then, $f(x)$ is a continuous function and monotonically decreasing.
To apply the test, the integral 
$$\int_{2}^{\infty} f(x) d x=\int_{2}^{\infty} \frac{1}{x(\ln x)^{2}} d x$$ 
should be evaluated. Let $\ln x=t \Rightarrow \dfrac{1}{x} d x=d t$
$$
\begin{aligned}
\int f(x) d x &=\int \frac{1}{t^{2}} d t \\
&=\frac{-1}{t} \\
&=\frac{-1}{\ln x}
\end{aligned}
$$
$$
\begin{aligned}
\int_{2}^{\infty} f(x) d x &=\left[\frac{-1}{\ln x}\right]_{2}^{\infty} \\
&=0-\left(\frac{-1}{\ln 2}\right) \\
&=\frac{1}{\ln 2}
\end{aligned}
$$
Therefore the integral of the function is a finite number.
By integral test, $\sum_{n=2}^{\infty} f(n)$ is convergent if the integral is a finite number.
since, the integral is a finite number, 
$$\sum_{n=2}^{\infty} \frac{1}{n(\ln n)^{2}}$$ converges by integral test.

$\vspace{3mm}$

$\boxed{\textbf{Solution}}$ For $(b)$ by addition, it is given that 
$$\sum_{n=2}^{1,00,000}\left[n(\ln n)^{2}\right]^{-1}=2 \cdot 02288$$
It is required to find the sum of the series, 
$$\sum_{n=2}^{\infty}\left[n(\ln n)^{2}\right]^{-1}$$
But, 
$$\sum_{n=2}^{\infty}\left[n(\ln n)^{2}\right]^{-1}=\sum_{n=2}^{N}\left[n(\ln n)^{2}\right]^{-1}+\sum_{n=N+1}^{\infty}\left[n(\ln n)^{2}\right]^{-1}$$
By the inequality, 
$$\int_{N+1}^{\infty} f(x) d x \leq \sum_{n=N+1}^{\infty} a_{n} \leq \int_{N+1}^{\infty} f(x) d x+a_{N+1}$$ 
where 
$$a_{n}=f(n)=\frac{1}{n(\ln n)^{2}}$$
$$
\int_{N+1}^{\infty} \frac{1}{x(\ln x)^{2}} d x \leq \sum_{n=N+1}^{\infty} \frac{1}{n(\ln n)^{2}} \leq \int_{N+1}^{\infty} \frac{1}{x(\ln x)^{2}} d x+a_{N+1}
$$
Put, the value $N=1,00,000$ in the inequality.
$$
\int_{1,00,001}^{\infty} \frac{1}{x(\ln x)^{2}} d x \leq \sum_{n=1,00,001}^{\infty} \frac{1}{n(\ln n)^{2}} \leq \int_{1,00,001}^{\infty} \frac{1}{x(\ln x)^{2}} d x+\frac{1}{1,00,001 \times(\ln 1,00,001)}
$$
$$
\left[\frac{-1}{\ln x}\right]_{1,00,001}^{\infty} \leq \sum_{n=1,00,001}^{\infty} \frac{1}{n(\ln n)^{2}} \leq\left[\frac{-1}{\ln x}\right]_{1,00,001}^{\infty}+\frac{1}{1,00,001 \times(\ln 1,00,001)^{2}}
$$
$$
\frac{1}{\ln 1,00,001} \leq \sum_{n=1,00,001}^{\infty} \frac{1}{n(\ln n)^{2}} \leq \frac{1}{\ln 1,00,001}+0 \cdot 0868500
$$
$$
0.0868588 \leq \sum_{n=1,00,001}^{\infty} \frac{1}{n(\ln n)^{2}} \leq 0 \cdot 0868588+0 \cdot 0868500
$$
$$
0.0868588\leq \sum_{n=1,00,001}^{\infty} \frac{1}{n(\ln n)^{2}} \leq 0 \cdot 0868588754
$$
From the above the value of the sum is 
$$\sum_{n=1,00,001}^{\infty} \frac{1}{n(\ln n)^{2}}=0 \cdot 08686$$
Now, the required value of the sum of the series is
$$
\begin{aligned}
\sum_{n=2}^{\infty}\left[n(\ln n)^{2}\right]^{-1} &=\sum_{n=2}^{1,00,000}\left[n(\ln n)^{2}\right]^{-1}+\sum_{n=1,00,001}^{\infty}\left[n(\ln n)^{2}\right]^{-1} \\
&=2 \cdot 02288+0 \cdot 08686 \\
&=2 \cdot 10974
\end{aligned}
$$
Hence, the required sum is 
$$\sum_{n=2}^{\infty}\left[n(\ln n)^{2}\right]^{-1}=2 \cdot 10974$$


\newpage


\begin{mybox}{1.1.4}
Gauss' test is often given in the form of a test of the ratio
$$
\frac{u_{n}}{u_{n+1}}=\frac{n^{2}+a_{1} n+a_{0}}{n^{2}+b_{1} n+b_{0}}
$$
For what values of the parameters $a_{1}$ and $b_{1}$ is there convergence? divergence?

ANS: Convergent for $a_{1}-b_{1}>1$, divergent for $a_{1}-b_{1} \leq 1$
\end{mybox}



$\boxed{\textbf{Solution}}$ Using the division algorithm
$$
\begin{aligned}
\frac{u_{n}}{u_{n+1}} &=1+\frac{\left(a_{1}-b_{1}\right) n+\left(a_{0}-b_{0}\right)}{n^{2}+b_{1} n+b_{0}} \\
&=1+\frac{\left(a_{1}-b_{1}\right) n}{n^{2}+b_{1} n+b_{0}}+\frac{\left(a_{0}-b_{0}\right)}{n^{2}+b_{1} n+b_{0}} \\
&=1+\frac{\left(a_{1}-b_{1}\right) n}{n^{2}\left(1+\frac{b_{1}}{n}+\frac{b_{0}}{n^{2}}\right)}+\frac{\left(a_{0}-b_{0}\right)}{n^{2}\left(1+\frac{b_{1}}{n}+\frac{b_{0}}{n^{2}}\right)} \\
&=1+\frac{\left(a_{1}-b_{1}\right)}{n\left(1+\frac{b_{1}}{n}+\frac{b_{0}}{n^{2}}\right)}+\frac{\left(a_{0}-b_{0}\right)}{n^{2}\left(1+\frac{b_{1}}{n}+\frac{b_{0}}{n^{2}}\right)}
\end{aligned}
$$
Compare the expression with 
$$\frac{u_{n}}{u_{n+1}}=1+\frac{h}{n}+\frac{B(n)}{n^{2}}$$
Then, 
$$h=\frac{a_{1}-b_{1}}{\left(1+\frac{b_{1}}{n}+\frac{b_{0}}{n^{2}}\right)}$$
$$B(n)=\frac{a_{0}-b_{0}}{\left(1+\frac{b_{1}}{n}+\frac{b_{0}}{n^{2}}\right)}$$
For larger values of $n,$
$$\left(1+\frac{b_{1}}{n}+\frac{b_{0}}{n^{2}}\right) \rightarrow 1$$
Hence, for larger values of $n$, $B(n)$ is bounded. By Gauss's test, the series converges if $h>1$ and diverges if $h \leq 1$.
Therefore the series converges if $a_{1}-b_{1}>0$ and diverges if $a_{1}-b_{1} \leq 1$





\newpage



\begin{mybox}{1.1.5}
Test for convergence

\begin{multicols}{2}
\begin{enumerate}[$(a)$]
\item $\displaystyle \sum_{n=2}^{\infty}(\ln n)^{-1}$
\item $\displaystyle \sum_{n=1}^{\infty} \frac{n !}{10^{n}}$
\item $\displaystyle \sum_{n=1}^{\infty} \frac{1}{2 n(2 n+1)}$
\item $\displaystyle \sum_{n=1}^{\infty}[n(n+1)]^{-1 / 2}$
\item $\displaystyle \sum_{n=0}^{\infty} \frac{1}{2 n+1}$
\end{enumerate}
\end{multicols}
\end{mybox}


$\boxed{\textbf{Solution}}$ For $(a)$ As in all these convergence tests, it is good to first have a general idea of whether we expect this to converge or not, and then find an appropriate test to confirm our hunch. For this one, we can imagine that $\ln n$ grows very slowly, so that its inverse goes to zero very slowly - too slowly, in fact, to converge. To prove this, we can perform a simple comparison test. since $\ln n<n$ for $n \geq 2,$ we see that
$$
a_{n}=(\ln n)^{-1}>n^{-1}
$$
since the harmonic series diverges, and each term is larger than the corresponding harmonic series term, this series must diverge.


$\vspace{3mm}$

$\boxed{\textbf{Solution}}$ For $(b)$ the factorial in the numerator will start to dominate over the power in the denominator. So we expect this to diverge. As a proof, we can perform a simple ratio test.
$$
a_{n}=\frac{n !}{10^{n}} \Rightarrow \frac{a_{n}}{a_{n+1}}=\frac{10}{n+1}
$$
Taking the limit, we obtain
$$
\lim _{n \rightarrow \infty} \frac{a_{n}}{a_{n+1}}=0
$$
hence the series diverges by the ratio test.



$\vspace{3mm}$


$\boxed{\textbf{Solution}}$ For $(c)$ We first note that this series behaves like $1 / 4 n^{2}$ for large $n .$ As a result, we expect it to converge. To see this, we may consider a simple comparison test
$$
a_{n}=\frac{1}{2 n(2 n+1)}<\frac{1}{2 n \cdot 2 n}=\frac{1}{4}\left(\frac{1}{n^{2}}\right)
$$
since the series $\zeta(2)=\sum_{n=1}^{\infty}\left(1 / n^{2}\right)$ converges, this series converges as well.


$\vspace{3mm}$

$\boxed{\textbf{Solution}}$ For $(d)$ we expect to diverges
$$
a_{n}=\frac{1}{\sqrt{n(n+1)}}>\frac{1}{\sqrt{(n+1)(n+1)}}=\frac{1}{n+1}
$$
Because the harmonic series diverges

$\vspace{3mm}$



$\boxed{\textbf{Solution}}$ For $(e)$ since this behaves as $1 / 2 n$ for large $n,$ the series ought to diverge. We may either compare this with the harmonic series or perform an integral test. Consider the integral test
$$
\int_{0}^{\infty} \frac{d x}{2 x+1}=\left.\frac{1}{2} \ln (2 x+1)\right|_{0} ^{\infty}=\infty
$$
Thus the series diverges

\newpage



\begin{mybox}{1.1.6}
Test for convergence
\begin{multicols}{2}
\begin{enumerate}[$(a)$]
\item $\displaystyle \sum_{n=1}^{\infty} \frac{1}{n(n+1)}$
\item $\displaystyle \sum_{n=2}^{\infty} \frac{1}{n \ln n}$
\item $\displaystyle \sum_{n=1}^{\infty} \frac{1}{n 2^{n}}$
\item $\displaystyle \sum_{n=1}^{\infty} \ln \left(1+\frac{1}{n}\right)$
\item $\displaystyle \sum_{n=1}^{\infty} \frac{1}{n \cdot n^{1 / n}}$
\end{enumerate}
\end{multicols}
\end{mybox}

$\boxed{\textbf{Solution}}$ For $(a)$ let the series be $\sum_{n=1}^{\infty} u_{n}$ where 
$$u_{n}=\frac{1}{n(n+1)}$$
If we put, 
$$f(x)=\frac{1}{x(x+1)}$$ 
then the function is a decreasing function. This is because the derivative of the function is 
$$f^{\prime}(x)=\frac{-2 x-1}{\left(x^{2}+x\right)^{2}}<0$$
Also, if 
$$f(n)=\frac{1}{n(n+1)}$$ 
then the series is a series of positive terms. we apply integral test, it is required to find the integral of the function from 1 to infinity.

$$
\begin{aligned}
\int_{1}^{\infty} f(x) d x &=\int_{1}^{\infty} \frac{1}{x(x+1)} d x \\
&=\int_{1}^{\infty}\left[\frac{1}{x}-\frac{1}{x+1}\right] d x \\
&=[\ln x-\ln (x+1)]_{1}^{\infty} \\
&=\left[\ln \left(\frac{x}{x+1}\right)\right]_{1}^{\infty}
\end{aligned}
$$
But the value of 
$$\lim_{x\rightarrow \infty} \ln \left(\frac{x}{x+1}\right)=\ln 1=0$$ 
and 
$$\lim_{x\rightarrow 1} \ln \left(\frac{x}{x+1}\right)=\ln \frac{1}{2}$$
Substituting the upper and lower limits, 
$$\int_{1}^{\infty} f(x) d x=0-\ln \left(\frac{1}{2}\right)=-\ln \left(\frac{1}{2}\right)$$
This is a finite number. So, by integral test, the given series converges.
Therefore, the series 
$$\sum_{n=1}^{\infty} \frac{1}{n(n+1)}$$ 
converges.

$\vspace{3mm}$

$\boxed{\textbf{Solution}}$ For $(b)$ Let the series be $\sum_{n=2}^{\infty} u_{n}$ where 
$$u_{n}=\frac{1}{n \ln n}$$ 
The series contains all positive terms. If we put, 
$$f(x)=\frac{1}{x \ln x}$$ 
then the function is a decreasing function. This is because the derivative of the function is 
$$f^{\prime}(x)=\frac{-(1+\ln x)}{(x \ln x)^{2}}<0$$
Also, if 
$$f(n)=\frac{1}{n \ln n}$$ 
then the series is a series of positive terms. To apply Integral test, it is required to find the integral of the function from 2 to infinity.
$$
\begin{aligned}
\int_{2}^{\infty} f(x) d x &=\int_{2}^{\infty} \frac{1}{x \ln x} d x \\
&=\int_{2}^{\infty} \frac{1}{x} \times \frac{1}{\ln x} d x \\
&=[\ln (\ln x)]_{2}^{\infty} \\
&=\ln (\ln \infty)-\ln (\ln 2)
\end{aligned}
$$
But the value of $\ln (\ln \infty)$ is infinite. So, the integral diverges
So, by integral test, the given series diverges.
Therefore, the series $\sum_{n=2}^{\infty} \frac{1}{n \ln n}$ diverges.

$\vspace{3mm}$

$\boxed{\textbf{Solution}}$ For $(c)$ let the series be $\sum_{n=1}^{\infty} u_{n}$ where 
$$u_{n}=\frac{1}{n 2^{n}}$$
The series contains all positive terms. It is required to apply the ratio test. For this evaluate 
$$\lim_{n\rightarrow \infty} \frac{u_{n+1}}{u_{n}}$$
The value of 
$$u_{n+1}=\frac{1}{(n+1) 2^{n+1}}$$
$$
\begin{aligned}
\lim_{n\rightarrow \infty} \frac{u_{n+1}}{u_{n}} &=\lim_{n\rightarrow \infty} \frac{1}{(n+1) 2^{n+1}} \times \frac{n 2^{n}}{1} \\
&=\lim_{n\rightarrow \infty} \frac{n}{(n+1) 2} \\
&=\frac{1}{2}<1
\end{aligned}
$$
By ratio test, the given series $\sum_{n=1}^{\infty} \frac{1}{n 2^{n}}$ converges.

$\vspace{3mm}$

$\boxed{\textbf{Solution}}$ For $(d)$ let the series be $\sum_{n=1}^{\infty} u_{n}$ where $u_{n}=\ln \left(1+\frac{1}{n}\right) .$ Le the sum of first $n$ terms be $S_{n}$
$$
\begin{aligned}
S_{n} &=\sum_{k=1}^{n} u_{k} \\
&=u_{1}+u_{2}+\cdots s+u_{n} \\
&=\ln 2+\ln \frac{3}{2}+\ln \frac{4}{3}+\cdots+\ln \frac{n+1}{n} \\
&=\ln \left(2 \times \frac{3}{2} \times \frac{4}{3} \times \cdots \times \frac{n+1}{n}\right) \\
&=\ln (n+1)
\end{aligned}
$$
From the above equation, it is clear that $S_{n}=\ln (n+1)$ So, by the definition of the sum of the series,
$$
\begin{aligned}
\sum_{n=1}^{\infty} u_{n} &=\lim_{n\rightarrow \infty} S_{n} \\
&=\lim_{n\rightarrow\infty} \ln (n+1) \\
&=\infty
\end{aligned}
$$
So, the given series $\sum_{n=1}^{\infty} \ln \left(1+\frac{1}{n}\right)$ diverges.

$\vspace{3mm}$

$\boxed{\textbf{Solution}}$ For $(e)$ it is required to use a test called limit comparison test. If $\sum_{n=1}^{\infty} u_{n}$ is a series of positive terms such that 
$$\lim_{n\rightarrow \infty} \frac{u_{n}}{v_{n}} \neq 0$$ 
then both the series converge or diverge together. Here, $u_{n}=\dfrac{1}{n n^{1/n}}$. Consider the series, $v_{n}=\dfrac{1}{n}$.
$$\displaystyle \lim_{n\rightarrow \infty} \frac{u_{n}}{v_{n}} = \lim_{n\rightarrow \infty} \frac{\frac{1}{n n n}}{\frac{1}{n}} = 1$$
By limit comparison test, $\sum_{n=1}^{\infty} u_{n}$ converges if $\sum_{n=1}^{\infty} v_{n}$ converges and $\sum_{n=1}^{\infty} u_{n}$ diverges if $\sum_{n=1}^{\infty} v_{n}$
diverges.
But, the series 
$$\sum_{n=1}^{\infty} v_{n}=\sum_{n=1}^{\infty} \frac{1}{n}$$ 
diverges. So, by the test 
$$\sum_{n=1}^{\infty} u_{n}$$ 
also diverges. Hence, the series 
$$\sum_{n=1}^{\infty} \frac{1}{n n^{\frac{1}{n}}}$$
is divergent


\newpage



\begin{mybox}{1.1.7}
For what values of $p$ and $q$ will $\displaystyle\sum_{n=2}^{\infty} \frac{1}{n^{p}(\ln n)^{q}}$ converge?

ANS. Convergent for $\left\{\begin{array}{ll}p>1, & \text { all } q, \vspace{2mm}\\ p=1, & q>1,\end{array} \quad\right.$ divergent for $\left\{\begin{array}{ll}p<1, & \text { all } q \vspace{2mm} \\ p=1, & q \leq 1\end{array}\right.$
\end{mybox}



$\boxed{\textbf{Solution}}$ since the $\ln n$ term is not as dominant as the power term $n^{p},$ we may have some idea that the series ought to converge or diverge as the $1 / n^{p}$ series. To make this more precise, we can use Raabe's test
$$
\begin{aligned}
a_{n}=\frac{1}{n^{p}(\ln n)^{q}} \Rightarrow \frac{a_{n}}{a_{n+1}} &=\frac{(n+1)^{p}(\ln (n+1))^{q}}{n^{p}(\ln n)^{q}} \\
&=\left(1+\frac{1}{n}\right)^{p}\left(1+\frac{\ln \left(1+\frac{1}{n}\right)}{\ln n}\right)^{q} \\
&=\left(1+\frac{1}{n}\right)^{p}\left(1+\frac{1}{n \ln n}+\cdots\right)^{q} \\
&=\left(1+\frac{p}{n}+\cdots\right)\left(1+\frac{q}{n \ln n}+\cdots\right) \\
&=\left(1+\frac{p}{n}+\frac{q}{n \ln n}+\cdots\right)
\end{aligned}
$$

$$
\lim _{n \rightarrow \infty} n\left(\frac{a_{n}}{a_{n+1}}-1\right)=\lim _{n \rightarrow \infty}\left(p+\frac{q}{\ln n}+\cdots\right)=p
$$

This gives convergence for $p>1$ and divergence for $p<1$. For $p=1,$ Raabe's test is ambiguous. However, in this case we can perform an integral test. since
$$
p=1 \quad \Rightarrow \quad a_{n}=\frac{1}{n(\ln n)^{q}}
$$
we evaluate
$$
\int_{2}^{\infty} \frac{d x}{x(\ln x)^{q}}=\int_{\ln 2}^{\infty} \frac{d u}{u^{q}}
$$
where we have used the substitution $u=\ln x .$ This converges for $q>1$ and diverges otherwise.  Hence the final result is
$$
\begin{array}{l}
p>1, \text{any} \ q \quad \text{ converge }\\
p=1, q>1 \quad \text { converge } \\
p=1, q \leq 1 \quad \text { diverge } \\
p<1, \text{any} \ q \quad \text{ diverge }
\end{array}
$$


\newpage


\begin{mybox}{1.1.8}
Given $\sum_{n=1}^{1,000} n^{-1}=7.485470 \ldots$ set upper and lower bounds on the Euler-Mascheroni constant. 

ANS: $0.5767<\gamma<0.5778$
\end{mybox}


$\boxed{\textbf{Solution}}$ No solution yet.


$\vspace{3mm}$

\begin{mybox}{1.1.9}
(From Olbers' paradox.) Assume a static universe in which the stars are uniformly distributed. Divide all space into shells of constant thickness; the stars in any one shell by themselves subtend a solid angle of $\omega_{0}$. Allowing for the blocking out of distant stars by nearer stars, show that the total net solid angle subtended by all stars, shells extending to infinity, is exactly $4 \pi$. [Therefore the night sky should be ablaze with light. For more details, see E. Harrison, Darkness at Night: A Riddle of the Universe. Cambridge, MA: Harvard University Press (1987).]
\end{mybox}


$\boxed{\textbf{Solution}}$ No solution yet

\newpage



\begin{mybox}{1.1.10}
Test for convergence
$$
\sum_{n=1}^{\infty}\left[\frac{1 \cdot 3 \cdot 5 \cdots(2 n-1)}{2 \cdot 4 \cdot 6 \cdots(2 n)}\right]^{2}=\frac{1}{4}+\frac{9}{64}+\frac{25}{256}+\cdots
$$
\end{mybox}



$\boxed{\textbf{Solution}}$ Let the series be $\sum_{n=1}^{\infty} u_{n}$ where $u_{n}=\displaystyle \left[\frac{1 \cdot 3 \cdot 5 \cdots (2 n-1)}{2 \cdot 4 \cdot 6 \cdots  \cdot(2 n)}\right]^{2}$. Take $a_{n}=n,$ then $\sum_{n=1}^{\infty} a_{n}^{-1}$ diverges
Consider the following ratio.

$$
\frac{u_{n}}{u_{n+1}}=\frac{\left[\frac{1 \cdot 3 \cdot 5 \cdots (2 n-1)}{2 \cdot 4 \cdot 6 \cdots (2 n)}\right]^{2}}{\left[\frac{1 \cdot 3 \cdot 5 \cdots (2(n+1)-1)}{2 \cdot 4 \cdot 6 \cdots (2(n+1))}\right]^{2}}
$$

$$
\frac{u_{n}}{u_{n+1}}=\left[\frac{1 \cdot 3 \cdot 5 \cdots (2 n-1)}{2 \cdot 4 \cdot 6 \cdots (2 n)}\right]^{2} \times\left[\frac{2 \cdot 4 \cdot 6 \cdots  \cdot(2 n+2)}{1 \cdot 3 \cdot 5 \cdots (2 n+1)}\right]^{2}
$$
$$\frac{u_{n}}{u_{n+1}}=\left[\frac{(2 n+2)}{(2 n+1)}\right]^{2}$$
Now, consider the expression
$$
\lim_{n\rightarrow \infty}\left[a_{n} \frac{u_{n}}{u_{n+1}}-a_{n+1}\right]= \lim_{n\rightarrow \infty} \left[n\left[\frac{(2 n+2)}{(2 n+1)}\right]^{2}-(n+1)\right]
$$

$$
\lim_{n\rightarrow \infty}\left[a_{n} \frac{u_{n}}{u_{n+1}}-a_{n+1}\right]= \lim_{n\rightarrow \infty} \left[\frac{n\left(4 n^{2}+4+8 n\right)-(n+1)\left(4 n^{2}+1+4 n\right)}{4 n^{2}+1+4 n}\right]
$$

$$
\lim_{n\rightarrow \infty}\left[a_{n} \frac{u_{n}}{u_{n+1}}-a_{n+1}\right]= \lim_{n\rightarrow \infty} \left[\frac{4 n^{3}+4 n+8 n^{2}-4 n^{3}-n-4 n^{2}-4 n^{2}-1-4 n}{4 n^{2}+1+4 n}\right]
$$

$$
\lim_{n\rightarrow \infty}\left[a_{n} \frac{u_{n}}{u_{n+1}}-a_{n+1}\right]= \lim_{n\rightarrow \infty} \left[\frac{-n-1}{4 n^{2}+1+4 n}\right]
$$
But the degree of the polynomial in the denominator is greater than the degree of the polynomial in the numerator.
$$\lim_{n\rightarrow \infty} \left[\frac{-n-1}{4 n^{2}+1+4 n}\right] = 0$$
This implies that 
$$
\lim_{n\rightarrow \infty}\left[a_{n} \frac{u_{n}}{u_{n+1}}-a_{n+1}\right]= 0$$

By Kummer's theorem, the given series 
$$\sum_{n=1}^{\infty}\left[\frac{1 \cdot 3 \cdot 5 \cdots (2 n-1)}{2 \cdot 4 \cdot 6 \cdots (2 n)}\right]^{2}$$ 
diverges.








\newpage



\begin{mybox}{1.1.11}
Determine whether each of these series is convergent, and if so, whether it is absolutely convergent:

\begin{enumerate}[$(a)$]
\item $\displaystyle \frac{\ln 2}{2}-\frac{\ln 3}{3}+\frac{\ln 4}{4}-\frac{\ln 5}{5}+\frac{\ln 6}{6}-\cdots$
\item $\displaystyle \frac{1}{1}+\frac{1}{2}-\frac{1}{3}-\frac{1}{4}+\frac{1}{5}+\frac{1}{6}-\frac{1}{7}-\frac{1}{8}+\cdots$
\item $\displaystyle 1-\frac{1}{2}-\frac{1}{3}+\frac{1}{4}+\frac{1}{5}+\frac{1}{6}-\frac{1}{7}-\frac{1}{8}-\frac{1}{9}-\frac{1}{10}+\frac{1}{11} \cdots+\frac{1}{15}-\frac{1}{16} \cdots-\frac{1}{21}+\cdots$
\end{enumerate}
\end{mybox}


$\boxed{\textbf{Solution}}$ No solution yet.

$\vspace{3mm}$




\begin{mybox}{1.1.12}
Catalan's constant $\beta(2)$ is defined by
$$
\beta(2)=\sum_{k=0}^{\infty}(-1)^{k}(2 k+1)^{-2}=\frac{1}{1^{2}}-\frac{1}{3^{2}}+\frac{1}{5^{2}} \cdots
$$
Calculate $\beta(2)$ to six-digit accuracy.

Hint. The rate of convergence is enhanced by pairing the terms,
$$
(4 k-1)^{-2}-(4 k+1)^{-2}=\frac{16 k}{\left(16 k^{2}-1\right)^{2}}
$$
If you have carried enough digits in your summation, $\sum_{1 \leq k \leq N} 16 k /\left(16 k^{2}-1\right)^{2},$ additional significant figures may be obtained by setting upper and lower bounds on the tail of the series, $\sum_{k=N+1}^{\infty} .$ These bounds may be set by comparison with integrals, as in the Maclaurin integral test.

ANS. $\beta(2)=0.915965594177 \cdots$
\end{mybox}



$\boxed{\textbf{Solution}}$ No solution yet.

$\vspace{3mm}$





\begin{mybox}{1.1.13}
Show how to combine $\zeta(2)=\sum_{n=1}^{\infty} n^{-2}$ with $\alpha_{1}$ and $\alpha_{2}$ to obtain a series converging as $n^{-4}$

Note. $\zeta(2)$ has the known value $\pi^{2} / 6 .$ See Eq. (12.66).
\end{mybox}


$\boxed{\textbf{Solution}}$ No solution yet.

$\vspace{3mm}$





\begin{mybox}{1.1.14}
Give a method of computing
$$
\lambda(3)=\sum_{n=0}^{\infty} \frac{1}{(2 n+1)^{3}}
$$
that converges at least as fast as $n^{-8}$ and obtain a result good to six decimal places.

ANS. $\quad \lambda(3)=1.051800$
\end{mybox}


$\boxed{\textbf{Solution}}$ No solution yet.

$\vspace{3mm}$




 
\begin{mybox}{1.1.15}
Show that (a) $\sum_{n=2}^{\infty}[\zeta(n)-1]=1$,
(b) $\sum_{n=2}^{\infty}(-1)^{n}[\zeta(n)-1]=\frac{1}{2}$,
where $\zeta(n)$ is the Riemann zeta function.
\end{mybox}


$\boxed{\textbf{Solution}}$ No solution yet.

$\vspace{3mm}$





\begin{mybox}{1.1.16}
The convergence improvement of 1.1.11 may be carried out more expediently (in this special case) by putting $\alpha_{2},$ from Eq. (1.26), into a more symmetric form: Replacing $n$ by $n-1,$ we have
$$
\alpha_{2}^{\prime}=\sum_{n=2}^{\infty} \frac{1}{(n-1) n(n+1)}=\frac{1}{4}
$$

\begin{enumerate}[$(a)$]
\item Combine $\zeta(3)$ and $\alpha_{2}^{\prime}$ to obtain convergence as $n^{-5}$ 
\item Let $\alpha_{4}^{\prime}$ be $\alpha_{4}$ with $n \rightarrow n-2 .$ Combine $\zeta(3), \alpha_{2}^{\prime},$ and $\alpha_{4}^{\prime}$ to obtain convergence as $n^{-7}$
\item If $\zeta(3)$ is to be calculated to six-decimal place accuracy (error $\left.5 \times 10^{-7}\right),$ how many terms are required for $\zeta(3)$ alone? combined as in part (a)? combined as in part (b)?
\end{enumerate}
Note. The error may be estimated using the corresponding integral.

ANS.
$$(a)\quad \zeta(3)=\frac{5}{4}-\sum_{n=2}^{\infty} \frac{1}{n^{3}\left(n^{2}-1\right)}$$
\end{mybox}



$\boxed{\textbf{Solution}}$ No solution yet.

$\vspace{3mm}$






\newpage









































































































